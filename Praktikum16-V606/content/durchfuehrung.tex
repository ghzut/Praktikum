
\section{Durchführung}
\label{sec:Durchführung}

Zunächst wird die Filterkurve des Selektiv-Verstärkers bei einer Güte von $100$ aufgenommen. Hierfür wird der Aufbau in Abbildung \ref{fig:Schaltskizze}, jedoch ohne Brückenschaltung verwendet. Es wird zunächst die Spannung direkt am Synthesizer gemessen. Danach wird in einem Bereich von $\SI{30}{\kilo\hertz}$ bis $\SI{40}{\kilo\hertz}$ die jeweiligen Spannungen am Selektiv-Verstärker gemessen. Hierbei wird in besonders kleinen Abständen um die Durchlassfrequenz herum gemessen. Um nachher die Suszeptibilität von Oxiden bestimmen zu können, wird der Aufbau gemäß Abbildung \ref{fig:Schaltskizze} aufgebaut. Zunächst wird die Frequenz am Sinusgenerator auf die Durchlassfrequenz des Selektivverstärkers eingestellt. Danach wird die Brückenspannung auf null abgeglichen und der Widerstand $R_3$ und die Brückenspannung $U_\text{Br}$ notiert. Nun wird das Proberöhrchen eingeführt und die neue Spannung $U_\text{Br}$ notiert. Danach wird die Brücke wieder auf Null abgeglichen und der neue Widerstand $R_3$ notiert. Dies wird wiederholt bis alle Werte mindestens dreimal gemessen und notiert wurden. 