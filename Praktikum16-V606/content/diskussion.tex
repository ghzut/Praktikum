
\section{Diskussion}
\label{sec:Diskussion}
%Die bestimmten Suszeptibilitäten zeigen große Abweichungen, %sowohl unter den
%verschiedenen Bestimmungsmethoden als auch insbesondere in %Relation zu den
%bestimmten Theoriewerten.


Die bestimmten Suszeptibilitäten zeigen Abweichungen bezüglich der ermittelten Theoriewerte, welche zu diskutieren sind. Es wird dabei deutlich, dass die per Spannungsdifferenz 
bestimmten Werte wesentlich größere Differenzen aufzeigen, als die per Widerstandsdifferenz bestimmten. Bei den Letzteren trifft der Wert von $Gd_2O_3$ mit seiner $1 \sigma$-Standardabweichung sogar den Theoriewert. Die Werte der anderen beiden Stoffe liegen zumindest nur leicht außerhalb der Fehlertoleranz. Die mit der Spannungsdifferenz bestimmten Werte sind jedoch weitaus kleiner als die Theoriewerte. Diese sind zwischen zwei und fünf mal größer.
 Es gibt mehrere Fehlerquellen, weswegen die Spannungsdifferenzmethode schlechter abschneidet. Zum einen ist der Verstärkungsfaktor
des Selektiv-Verstärkers nur geringfügig größer als 1. Da die auftretende Filterkurve nach
Abb. \ref{fig:GraphSelektiv} eine große Güte besitzt und der für die spätere
Messung verwendete Sinusgenerator nur sehr grobe Einstellmöglichkeiten besitzt,
liegt der echte Verstärkungsfaktor vermutlich weit unter 1. Des Weiteren konnte
die verwendete Brückenschaltung nicht komplett abgeglichen werden. Die Spannungsmessungen werden dadurch verfälscht. Außerdem
konnten bei vielen Messungen keine erkennbaren Unterschiede festgestellt werden, weswegen einige Spannungsdifferenzen allein auf Rundungen basieren.
Die Methode mithilfe von Widerstandsdifferenzen zeigt sich in dem Sinne
unempfindlicher, da sich die bestimmten $R_3$ Werte genauer bestimmen ließen und das nicht vollständige Abgleichen der Brückenschaltung weniger Einfluss besitzt. Die Abhängigkeit der Suszeptibilitäten von der Temperatur konnte in beiden Fällen nicht berücksichtigt werden.
Schließlich lässt sich vermuten, dass die Ergebnisse der Spannungsdifferenzmethode nicht aussagekräftig sind. 