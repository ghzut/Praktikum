
\section{Diskussion}
\label{sec:Diskussion}
Die bestimmten Suszebtibilitäten zeigen große Abweichungen, sowohl unter den
verschiedenen Bestimmungsmethoden als auch insbesondere in Relation zu den
bestimmten Theoriewerten. Hierbei sind die Theoriewerte teilweise 2
Zehnerpotenzen gößer als die praktisch ermittelten. Die per Widerstandsdifferenz
bestimmten Größen sind ca. 2-10 mal größer als die per Spannungsdifferenz
ermittelten. Es gibt mehrere Fehlerquellen. Zum einen ist der Verstärkungsfaktor
des Selektiv-Verstärkers nur geringfügig größer als 1. Da auftretende Filterkurve nach
Abb. \ref{fig:GraphSelektiv} eine große Güte besitzt und der für die spätere
Messung verwendete Sinusgenerator nur sehr grobe Einstellmöglichkeiten besitzt,
liegt der echte Verstärkungsfaktor vermutlich weit unter 1. Desweiteren konnte
die verwendete Brückenschaltung nicht komplett abgeglichen werden. Außerdem
konnten bei vielen Messungen keine erkennbaren Unterschiede festgestellt werden,
weshalb die Widerstandsänderungen in erster Linie dazu sind einen Fehler zu
generieren. Zusätzlich basieren einige Spannungsdifferenzen allein auf Rundungen.
Die Methode mithilfe von Widerstandsdifferenzen zeigt sich in dem Sinne
unempfindlicher und liefert bessere Werte als die mit Spannungsdifferenzen.
 Final sind die bestimmten Suszeptibilitäten jedoch unter beiden Methoden nicht
aussagekräftig.
