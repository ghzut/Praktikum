\section{Auswertung}
\label{sec:Auswertung}

Die Graphen wurden sowohl mit Matplotlib \cite{matplotlib} als auch NumPy \cite{numpy} erstellt. Die
 Fehlerrechnung wurde mithilfe von Uncertainties \cite{uncertainties} durchgeführt.
Die Konstanten $k$, $\hslash$, $e_0$, $m_0$, $u_0$ und $N_\text{A}$ sind vom NIST \cite{nistgov}.
\subsection{Untersuchung der Filterkurve des Selektiv-Verstärkers}

\begin{table}
 \centering
 \caption{Die an dem Selektiv-Verstärker gemessenen Spannungen in Abhängigkeit der Eingangsspannungsfrequenz.}
 \input{build/tabKurvenergebnisse1.tex}
 \input{build/tabKurvenergebnisse2.tex}
\end{table}


\begin{figure}
 \centering
 \caption{Die Filterkurve des Selektiv-Verstärkers.}
 \includegraphics[width=\linewidth-70pt,height=\textheight-70pt,keepaspectratio]{build/Resonanzkurve.pdf}
 \label{fig:GraphSelektiv}
\end{figure}

Zunächst wird der Spannungsverlauf des Selektiv-Verstärkers betrachtet. Hierbei
zeigt der Graph in Abb. \ref{fig:GraphSelektiv} einen deutlichen, symmetrischen
Peak bei ca. $\SI{35.17}{\kilo\hertz}$. Mithilfe eines exponentiellen Fits der Form:

\begin{equation}
  y = a \cdot \exp(-b|x-c|)+d
  \end{equation}

lässt sich der interessante Bereich um das Maximum geeignet betrachten. Hierbei
wurden der Streckungsfaktor $a = 900$ und das Maximum des Verlaufes $c = 35.17$ manuell gesetzt. Die Parameter $b$
und $d$ wurden mithilfe von Curvefit bestimmt und liegen bei $b = 2.29$ und $d = 65.26$.
Die Eingangsfrequenz der
Wechselspannungsquelle wird daher für die folgenden Versuche auf ca. $\SI{35.17}{\kilo\hertz}$
gestellt. Bei einer Eingangsspannung von $\SI{0.95}{\volt}$ entsteht außerdem ein
Verstärkungsfaktor von ca. 1.02.
Über den Fit lässt sich die Güte experimentell zu ca. $Q = 107$ bestimmen. Diese
liegt in unmittelbarer Nähe zur eingestellten Güte von $Q = 100$. Der Unterschied
ist auf die Form des Fittes zurückzuführen. Aufgrund der groben Frequenzeinstellungsmöglichkeiten der Quelle und der hohen Güte
ist jedoch nicht die volle Verstärkung zu erwarten,
 weshalb der Verstärkungsfaktor nicht mit in die Rechnung einbezogen wird.

\subsection{Die aufgenommenen Daten für die Suszeptibilitätsbestimmung}

\begin{table}
 \centering
 \caption{Die Abmessungen der verwendeten Proben.}
 \input{build/tabbasis.tex}
 \label{tab:basis}
\end{table}




\begin{table}
 \centering
 \caption{Die gemessenen Werte für $Nd_2O_3$.}
 \input{build/tabNd2O3.tex}
 \label{tab:Nd2O3}
\end{table}

\begin{table}
 \centering
 \caption{Die gemessenen Werte für $Gd_2O_3$.}
 \input{build/tabGd2O3.tex}
 \label{tab:Gd2O3}
\end{table}


\begin{table}
 \centering
 \caption{Die gemessenen Werte für $Dy_2O_3$.}
 \input{build/tabDy2O3.tex}
 \label{tab:Dy2O3}
\end{table}

\subsection{experimentelle Bestimmung der Suszeptibilität über die Spannungsdifferenz}

\begin{table}
 \centering
 \caption{Die mit der Spannungsdifferenz bestimmten Suszeptibilitäten.}
 \input{build/SusU.tex}
 \label{tab:SusU}
\end{table}

Da die eingestellte Wechselspannungsfrequenz hinreichend groß ist folgt die
Suszibilität mit Formel \ref{eq:SusU}. Es wird eine Querschnittsfläche $F$
der Messspule von $\SI{86.6}{\milli\meter\squared}$ angenommen. Der reale Probenquerschnitt $Q_\text{real}$,
unterscheidet sich von dem eines Einkristalls, da das Probenmaterial staubförmig ist. Für ihn gilt:
\begin{equation}
  Q_\text{real} = \frac{M}{l \rho}\text{.}
\end{equation}
Hierbei bezeichnet $M$ die Masse, $l$ die Länge und $\rho$ die Einkristalldichte der Probe.
Mithilfer oben genannter Überlegegungen und den Spannungsdaten der Tabellen \ref{tab:Nd2O3},\ref{tab:Gd2O3},\ref{tab:Dy2O3}
ergeben sich die Suszeptibilitäten in Tabelle \ref{tab:SusU}.



\subsection{experimentelle Bestimmung der Suszeptibilität über die Widerstandsdifferenz}


\begin{table}
 \centering
 \caption{Die mit der Widerstandsdifferenz bestimmten Suszeptibilitäten.}
 \input{build/SusR.tex}
 \label{tab:SusR}
\end{table}


Ähnlich folgen sich die Suszeptibilitäten über die Widerstandsdifferenz. Diese ergeben sich mit Formel
\ref{eq:SusR}. Hierzu werden die bekannten $F$ und $Q_\text{real}$ sowie die Widerstandsmesswerte
der Tabellen \ref{tab:Nd2O3},\ref{tab:Gd2O3},\ref{tab:Dy2O3} verwendet.
Bei den Messwerten von $R_3$ muss zusätzlich beachtet werden, dass es sich nur
um Schrittzahlen handelt und eine Schrittweite $\SI{0.005}{\ohm}$ beträgt.
Zusätzlich besteht $R_3$ auch noch aus einem weiteren, bereits integriertem Widerstand
der Größe $\SI{998}{\ohm}$, welcher addiert werden muss. Mit diesen Angaben folgen die Suszeptibilitäten in Tabelle \ref{tab:SusR}.

%für uB in theorie noch formelnummer ergänzen

\subsection{Bestimmung der theoretischen Suszeptibilitäten}

\begin{table}
 \centering
 \caption{Die Drehimpulse der seltenen Erden.}
 \input{build/tabspins.tex}
 \label{tab:spins}
\end{table}


Auf Basis der in der Theorie dargestellten Hundschen Regeln ergeben sich die
benötigten Parameter Spin $S$, Gesamtbahndrehimpuls $L$ und Gesamtdrehimpuls $J$. Über Formel
\ref{eq:g_J} reultiert zusätzlich der Landefaktor. Die äußere Elektronenhülle von $Nd^+ $
besitzt 3 4f-Elektronen, die von $Gd^+ 7$ und die von
$Dy^+ 9$. Damit ergeben sich die Parameter in Tabelle \ref{tab:spins}.
Die Anzahl der magnetischen Momente pro Volumen $N$ ist gleich der Ionenzahl pro
Volumen und berechnet sich über:
\begin{equation}
  N = N_\text{A}\frac{m}{M l Q_\text{real}}\text{,}
\end{equation}
mit der molaren Masse $M$.
Über die Formeln \ref{eq:muB} und \ref{eq:chitheo} und den Probeninformationen in
Tabelle \ref{tab:basis} resultieren schließlich die Suszeptibilitäten in Tabelle \ref{tab:SusT}.
Es wird von einer Materialtemperatur von $\SI{20}{\degreeCelsius}$ ausgegangen. Die
molaren Massen der verwendeten Stoffe aus Tabelle \ref{tab:basis} stammen aus in
absteigender Reihenfolge aus den Quellen \cite{MNd2O3}, \cite{MGd2O3} und \cite{MDy2O3}, die Dichten aus \cite{V606}.

\begin{table}
 \centering
 \caption{Die theoretischen Suszeptibilitäten.}
 \input{build/SusT.tex}
 \label{tab:SusT}
\end{table}
