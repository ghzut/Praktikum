
\section{Durchführung}
\label{sec:Durchführung}

\renewcommand{\labelenumi}{\alph{enumi})}
Zunächst werden die tatsächlichen Geschwindigkeiten des Wagens ermittelt, welche
 hinter den Markierungen 6 - 60 stehen. Es werden Aufbau und Schaltung Abbbbb bzw. aaaaaabbbb verwendet. Hierzu wird eine Messstrecke mit den
  beiden Sensoren abgesteckt und ihre Länge abgemessen. Anschließend werden
   Durchfahrtszeiten zu allen Motoreinstellungen vorgenommen. Pro Motoreinstellung
    werden die Zeiten von 5 Vowärtsfahrten und 5 Rückwärtsfahrten notiert. Danach
     wird die Wellenlänge der zu betrachtenden Schallwelle ermittelt. Es wird der Aufbau nach AAAAABBB verwendet.
      Das Mikrofon wird mit der Mikrometerschraube bewegt und am Oszilloskop werden
       die auftretenden Lessajous-Figuren beobachtet. Der Abstand zwischen zwei
        Linien auf dem Oszilloskop beträgt eine halbe Wellenlänge.
