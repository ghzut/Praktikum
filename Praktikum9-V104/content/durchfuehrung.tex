
\section{Durchführung}
\label{sec:Durchführung}

\renewcommand{\labelenumi}{\alph{enumi})}
\begin{enumerate}
  \item Es wird die Durchbiegung eines einseitig eingespannten, runden Stabes in
   Abhängigkeit von x gemessen. Hierzu wird der Stab an der rechten Seite in die klemmvorrichtung eingespannt.
   Da nicht auzuschließen ist, dass der Stab bereits gebogen ist, wird zunächst
    eine Messung der Auslenkung ohne angehängte Masse durchgeführt. Es ist beim
     Messvorgang darauf zu achten, dass das Rädchen der Messuhr auf der Mitte des Stabes aufliegt.
     Der Abstand zwischen der einzelnen Messstellen beträgt
     $\SI{1}{\centi\meter}$. Anschließend wird eine Masse angehängt, sodass
      die maximale Durchbiegung ca. $5-\SI{7}{\milli\meter}$ beträgt. Die Auslenkungsmessung
       wird nun mit denselben Messtellen wiederholt. Die Durchbiegung $D(x)$
       berechnet sich über die Differenz der jeweiligen Werte.

       \item Das oben beschriebene Verfahren wird nochmals mit einem
        quadratischen Stab durchgeführt.

        \item Nun wird die Durchbiegung des quadratischen Stabes nochmals untersucht, diesmal liegt er jedoch an beiden Enden frei auf.
         Die Normalspannung soll nun in der Stabmitte wirken und wird durch eine
          Masse realisiert, welche in dieser befestigt wird.
          Da sich die Messuhr bei einer in der Mitte angehängten Masse nicht
           über die gesamte Stablänge schieben lässt, kommen zwei Messuhren zum Einsatz.
           Auch wird zunächst die Auslenkung ohne Last gemessen.
            Um systematische Messfehler aufgrund von verstellten Messuhren zu vermeiden werden bereits hier
             beide Uhren für die Linke bzw. Rechte Hälfte verwendet. Im Anschluss
              wird wieder eine Masse angehängt und nochmals gemessen. Auch hier
               berechnet sich $D(x)$ über die Differenz.

\end{enumerate}
