
\section{Durchführung}
\label{sec:Durchführung}

\renewcommand{\labelenumi}{\alph{enumi})}
Zunächst werden die tatsächlichen Geschwindigkeiten des Wagens ermittelt, welche
 hinter den Markierungen 6 - 60 stehen. Es werden Aufbau und Schaltung nach Abb.\ref{fig:Aufbau} bzw. Abb. \ref{fig:Aufbausteve} verwendet. Hierzu wird eine Messstrecke mit den
  beiden Sensoren abgesteckt und ihre Länge abgemessen. Der Zeitgenerator wird auf eine Genauigkeit von $\SI{1}{\milli\second}$ gesetzt. Anschließend werden
   Durchfahrtszeiten zu allen Motoreinstellungen vorgenommen. Pro Motoreinstellung
    werden die Zeiten von 5 Vowärtsfahrten und 5 Rückwärtsfahrten notiert. Danach
     wird die Wellenlänge der zu betrachtenden Schallwelle ermittelt. Es wird der Aufbau nach Abb. \ref{fig:lamb} verwendet.
      Das Mikrofon wird mit der Mikrometerschraube bewegt und am Oszilloskop werden
       die auftretenden Lessajous-Figuren beobachtet. Der Abstand zwischen zwei
        auf dem Oszilloskop dargestellten Linien entspricht einer halben Wellenlänge.
         Zuletzt wird die Frequenz bei unterschiedlichen Geschwindigkeiten ermittelt.
          Dazu findet wieder ein Aufbau nach \ref{fig:Aufbau2} Verwendung. Der
          Untersetzer wird auf eine Zählung von 1000 eingestellt, sodass die Zählzeit bei $\SI{1}{\second}$ liegt.
          Auch dieses mal werden jeweils Frequenzzählungen von 5 Vorwärtsfahrten und 5 Rückfährtsfahrten notiert.
