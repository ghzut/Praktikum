
\section{Diskussion}
\label{sec:Diskussion}

%Was alles rein muss:
%\begin{itemize}
%	\item Warum systematische Abweichung im Graph \ref{fig:Graph1}?
%	\item Grober Fehler oder systematischer zwischen den Proportionalitätsfaktoren?
%	\item Warum Fehler zwischen den Proportionalitätsfaktoren? (Vielleicht Änderung von $f_0$)
%	\item Warum der zweite Wert besser ist (Vergleich der Schallgeschwindigkeit mit Literaturwert!)? normaldruck20 grad \SI{343.2}{\meter\per\second} \cite{c}

%\end{itemize}

Im Graph \ref{fig:Graph1} liegt eine systematische Abweichung zum Fit vor. Diese Abweichung entsteht vermutlich dadurch, dass nur ganzzahlige Frequenzen gemessen werden können. Die gemessene Frequenzen sind immer nach unten gerundet. Dadurch wird der Abstand zu dem gemessenen $f_0$ für die Frequenzen $f_\text{r}$ größer und für $f_\text{v}$ kleiner und somit kommt es zu einer systematischen Abweichung.


Bei dem Vergleich mithilfe des Studentschen t-Testes ist aufgefallen, dass sich die Erwartungsgrößen mit einer Wahrscheinlichkeit von $\SI{97.5}{\percent}$ unterscheiden und somit entweder ein systematischer oder grober Fehler vorliegt. Das ein systematischer Fehler vorliegt war bereits vor der Auswertung bekannt und sollte durch die Verwendung eines Gerätefaktors von $\frac{5}{4}$ ausgeglichen werden, indem die gemessene Frequenz mit diesem multipliziert wird. Der systematische Fehler könnte also durch eine Ungenauigkeit des Gerätefaktors zustande kommen, jedoch auch durch ein systematischen Fehler des Präzisionsschlittens.
Die aus dem Proportionalitätsfaktor aus der Ausgleichsrechnung berechnete Schallgeschwindigkeit $c$ von $\SI{348(3)}{\meter\per\second}$ liegt in der Nähe des Literaturwertes für die Schallgeschwindigkeit bei Normaldruck und $\SI{20}{\celsius}$ von $\SI{343.2}{\meter\per\second}$ \cite{c}. Auch die aus dem Proportionalitätsfaktor aus der Wellenlängenmessung berechnete Schallgeschwindigkeit $c$ von $\SI{351(1)}{\meter\per\second}$ liegt in der Nähe des Literaturwertes. Damit ist zu vermuten, dass der Gerätefaktor in dem gemessenem Bereich tatsächlich ungefähr $\frac{5}{4}$ beträgt.





%  Es wird vermutet, dass der Proportionalitätsfaktor aus der Ausgleichsrechnung wahrscheinlich näher am tatsächlichen Wert des Proportionalitätsfaktors liegt. Dies folgt, da die aus diesem berechnete Schallgeschwindigkeit $c$ von $\SI{349(3)}{\meter\per\second}$ in der Nähe des Literaturwertes für die Schallgeschwindigkeit bei Normaldruck und $\SI{20}{\celsius}$ von $\SI{343.2}{\meter\per\second}$ \cite{c} liegt. Die aus dem Proportionalitätsfaktor aus der Wellenlängenmessung berechnete Schallgeschwindigkeit $c$ von $\SI{281(1)}{\meter\per\second}$ hingegen weicht von diesem Literaturwert deutlich nach unten ab. Damit wird als Ursache des Fehlers vermutet, dass sich die Frequenz $f_0$ bei der Wellenlängenmessung von dem nachher gemessenen $f_0$ unterscheidet.
