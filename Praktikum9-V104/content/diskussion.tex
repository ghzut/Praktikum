
\section{Diskussion}
\label{sec:Diskussion}
Die Bestimmung des Elastizitätsmoduls des runden Stabes liefert einen Wert von
 $E = \SI{91.0 \pm 0.4}{\giga\pascal}$. In Anbetracht der leicht goldenen Färbung
  des Stabes besteht der Stab daher vermutlich aus einer Messinglegierung. Der
  E-Modul einer solchen Legierung liegt bei ca. $78 - \SI{123}{\giga\pascal}$\cite{Elastizitätsmodul},
   womit die Ergebnisse hierzu passen.
  Da jedoch keine weiteren Angaben bezüglich des Stabes vorliegen und der
  angegebene Literaturbereich die Möglichkeit eines anderen Stoffes offenlegt,
  ist keine eindeutige Einordnung möglich. Die Bestimmung des Elastizitätsmoduls
   des quadratischen, einseitig eingespannten Stabes liefert hingegen einen Wert
    von $\SI{87.38 \pm 0.26}{\giga\pascal}$. Dieser ist damit ca. $\SI{4}{\percent}$ kleiner als der des Runden, 
    was sich nicht allein durch die Messunsicherheit erklären lässt. Da beide Stäbe jedoch den gleichen Farbton
     besitzen, ist trotzdem davon auszugehen, dass es sich um den gleichen Stoff handelt. Dies bestätigt auch
      die nachfolgende Messung desselben Stabes bei beidseitiger Auflage. Deren Ergebnis liegt mit $\SI{91.63 \pm 0.25}{\giga\pascal}$ unter Beachtung
       der Messunsicherheit in der Sigma-Umgebung des runden Stabes. Mögliche Ursachen
         für die Abweichung der zweiten Messung ist ein falscher Wert der
         effektiven Stablänge.
