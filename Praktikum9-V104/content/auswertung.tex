\section{Auswertung}
\label{sec:Auswertung}

Der Graph in Abbildung \ref{fig:Graph1} wurde sowohl mit Matplotlib \cite{matplotlib} als auch NumPy \cite{numpy} erstellt. Die Fehlerrechnung wurde mit Unterstützung von Uncertainties \cite{uncertainties} durchgeführt.

\subsection{Bestimmung des Proportionalitätsfaktors mithilfe der Wellenlängenmessungen}
\begin{table}
	\centering
	\caption{Die Stellen $x$ an denen eine Phasenverschiebung von dem Vielfachen von $\pi$ gemessen wurde.}
	\input{build/tabwelle}
\end{table}
Aus Tabelle \ref{tab:tabwelle} berechnet sich die Wellenlänge $\lambda_0$ zu $\SI{16.95(7)}{\milli\meter}$. Mit Formel \eqref{eq:c} ergibt sich für den Proportionalitätsfaktor somit:
\begin{displaymath}
	\frac{f_0}{c} = \frac{1}{\lambda_0} = \SI{59.0(2)}{\per\meter}\text{.}
\end{displaymath}
Mit der zuvor gemessenen Frequenz $f_0$ von $\SI{16594}{\hertz}$ berechnet sich aus dem Proportionalitätsfaktor die Schallgeschwindigkeit $c$ zu $\SI{281(1)}{\meter\per\second}$.

\subsection{Bestimmung des Proportionalitätsfaktors  mithilfe einer linearen Ausgleichsrechnung}
Aus Tabelle \ref{tab:tabv1} berechnet sich die Geschwindigkeiten in den verschiedenen Gängen und Richtungen zu den Werten in Tabelle \ref{tab:tabges}.
\begin{table}
	\caption{Die gemessene benötigte Zeit $t_\text{v}$ vom linkem Sensor zum rechtem und die benötigte Zeit $t_\text{r}$ vom rechtem zum linkem in verschiedenen Gängen.}
	\begin{minipage}{0.5\textwidth}
		\centering
		\input{build/tabv1.tex}
	\end{minipage}
	\begin{minipage}{0.5\textwidth}
		\centering
		\input{build/tabv2.tex}
	\end{minipage}
\end{table}
\begin{table}
	\centering
	\caption{Die aus Tabelle \ref{tab:tabv1} berechneten Geschwindigkeiten in den verschiedenen Gängen und Richtungen}
	\input{build/tabges}
\end{table}

In Tabelle \ref{tab:tabges} kann die maximale Geschwindigkeit $v_{\text{max}}$, mit der sich der Wagen bewegt, von $\SI{51,43(2)}{\centi\meter\per\second}$ entnommen werden. Es wurde eine Frequenz $f_0$ von $\SI{16594}{\hertz}$ gemessen. 
Da $v_{\text{max}} \cdot f_0 > c$ gilt lässt sich nach Formel \eqref{eq:bewegS} $\Delta f$ messen, da ganzzahlige Unterschiede in diesem Versuchsaufbau gemessen werden können.
Es muss nicht unterschieden werden zwischen den Formeln \eqref{eq:bewegE} und \eqref{eq:bewegS}, da $|v_\text{max}| \ll c$ gilt. Dies ist auch in Abbildung \ref{fig:Graph1} daran zu erkennen, dass alle Wertepaare durch eine Gerade genähert werden können.

Der Fit in Abbildung \ref{fig:Graph1} besitzt die Form $y=a x$. Eine lineare Ausgleichsrechnung der Form $y=a x$ liefert mit den Wertepaaren aus Tabelle \ref{tab:tab1} und Tabelle \ref{tab:tabv1} nach Formel \eqref{eq:bewegE}:
\begin{displaymath}
a = \frac{f_0}{c} = \SI{47.6(4)}{\per\meter}\text{.}
\end{displaymath}
Dies ist der gesuchte Proportionalitätsfaktor. Mit der zuvor gemessenen Frequenz $f_0$ von $\SI{16594}{\hertz}$ berechnet sich aus $a$ die Schallgeschwindigkeit $c$ zu $\SI{349(3)}{\meter\per\second}$.
Es ist in Abbildung \ref{fig:Graph1} zu erkennen, dass die Messwerte durch den Fit angenähert werden können und dass es eine systematische Abweichung gibt, die in der Diskussion geklärt werden muss.
\begin{figure}
	\centering
	\caption{Die Differenz $\Delta f $ zwischen $f_0$ und der gemessenen Frequenz $f$ gegen die Geschwindigkeit $v$ aufgetragen.}
	\includegraphics[width=\linewidth-70pt,height=\textheight-70pt,keepaspectratio]{build/VgegenDeltaV.pdf}
	\label{fig:Graph1}
\end{figure}
\begin{table}
	\caption{Die gemessene Frequenz $f_\text{v}$ beim auf das Mikrofon zufahren und die gemessene Frequenz $f_\text{r}$ beim fahren in die entgegengesetzte Richtung in verschiedenen Gängen.}
	\begin{minipage}{0.5\textwidth}
		\centering
		\input{build/tab1.tex}
	\end{minipage}
	\begin{minipage}{0.5\textwidth}
		\centering
		\input{build/tab2.tex}
	\end{minipage}
\end{table}

\subsection{Vergleich zwischen den Ergebnissen mithilfe eines Studentschen t-Testes}

$t-Wert = 45.7$ -> 100 Prozent Wahrscheinlichkeit eines systematischen Fehlers. Was für einer ? Zweites Ergebnis nah am Theoriewert erstes nicht.

