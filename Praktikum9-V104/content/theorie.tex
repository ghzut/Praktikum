
\section{Theorie}
\label{sec:Theorie}
Eine ausbreitende Welle beschreibt den Limes einer Schwingung von N-Teilchen,
 welche sich von einem Sender ausgehend im Raum ausbreitet. Sie besitzt eine Frequenz $f_0$ und eine Wellenlänge $\lambda_0$. Über
 \begin{equation}
   c = f_0 \lambda_0 \label{eq:c}
   \end{equation}
   folgt die Ausbreitungsgeschwindigkeit $c$ der Welle.
Der Dopplereffekt ist nun ein Phänomen, welches auftritt,
 wenn sich Sender und Empfänger relativ zueinander bewegen. Er äußert sich in
  einer für den Empfänger veränderten Wellenfrequenz gegenüber der ursprünglich
   Ausgesendeten. An diesem Punkt muss zwischen Wellenausbreitungen im Vakuum
    und denen in Materie unterschieden werden. Im folgenden wird nur der letzte
     Fall behandelt. Breitet sich die Welle in einem
     Medium aus, wird zwischen einem bewegten Empfänger oder einem bewegten
      Sender separiert.
\begin{itemize}
\item Bewegt sich der Empfänger relativ zu einem ruhenden Sender mit der
 Geschwindigkeit $v$ folgt für die Frequenz $f_\text{E}$ am Empfänger:
 \begin{equation}
   f_\text{E} = f_0\left(1+\frac{v}{c}\right) \text{ Mit der Frequenzänderung }\Delta f = f_0\frac{v}{c}\text{.}\label{eq:bewegE}
   \end{equation}
   \item Bewegt sich hingegen der Sender relativ zu einem ruhenden Empfänger mit der
    Geschwindigkeit $v$ folgt in Näherung für die Frequenz $f_\text{S}$ am Sender:
    \begin{equation}
      f_\text{S} = f_\text{E} + f_0\left(\frac{v}{c}\right)²\text{.}\label{eq:bewegS}
      \end{equation}
      Für die Frequenzen bei einem $v > 0$ gilt daher $f_\text{S} > f_\text{E} > f_0$.
\end{itemize}
Ist die Geschwindigkeit des Senders betragsmäßig jedoch viel kleiner als
 die Ausbreitungsgeschwindigkeit der Welle nähern sich die Ergebnisse von
  \eqref{eq:bewegE} und \eqref{eq:bewegS} beliebig nah an.
