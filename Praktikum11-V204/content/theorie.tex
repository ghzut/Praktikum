
\section{Theorie}
\label{sec:Theorie}
Eine Materialeigenschaft eines Körpers ist die Speicherung von Wärme bzw. Abgabe
die von bereits gespeicherter Wärme. Existiert nun auf auf einem solchen Körper
 ein Wärmeungleichgewicht zwischen verschiedenen Stellen, kommt es zum Ausgleich
 und dem hierzu nötigen Wärmetransport.
mögliche Vorgänge des Wärmetransportes sind Wärmestrahlung, Konvektion oder
  Wärmeleitung. Im Folgenden wird sich auf letzteren Vorgang beschränkt. Im
  Allgemeinen basiert diese in festen Material zu großen Teilen auf Phonomen und
   frei beweglichen Elektronen. Wird ein Metall betrachtet können Effekte der
    bestehenden Gitterstrukturen vernachlässigt werden. Im Folgenden ein Stab der
    Länge $L$, sowie der Querschnittfläche $A$ betrachtet. Er besitzt die
    gleichmäßige Dichte $\rho$ und die spezifische Wärme $c$. Existiert nun ein
    Temperaturungleichgewicht $T$ zwischen den beiden Stabenden, folgt für den Fluss
     der Wärmeenergie $\text{d}Q$ durch $A$ in der Zeit $\text{d}t$ :
     \begin{equation}
       \text{d}Q = -\kappa A \frac{\partial T}{\partial x} \text{d}t
       \end{equation}
       Der Faktor \kappa beschreibt dabei die Wärmeleitfähigkeit und ist eine
        Konstante des Materials. Nach Konvention fließt die Wärmeenergie vom
         energiereicheren zum energieärmeren Stabende. Analog folgt für die Wärmestromdichte $j_\text{w}$:
         \begin{equation}
           j_\text{w} = -\kappa \frac{\partial T}{\partial x}
           \end{equation}
           Auf Basis beider Gleichungen, sowie die Kontinuitätsgleichung folgt
            die eindimensionale Wärmeleitungsgleichung
            \begin{equation}
              \frac{\partial T}{\partial t} = \frac{\kappa}{\rho c}\frac{\partial² T}{\partial x²}
              \end{equation}
              mit der materialspezifischen Temperaturleitfähigkeit
              \begin{equation}
                \sigma_\text{T} = \frac{\kappa}{\rho c}
                \end{equation}
               definieren.
             Diese beschreibt
             die zeitliche und räumliche Abhängigkeit des Temperaturverlaufes im Stab.
             Die Schnelligkeit, mit welcher sich die Temperaturausgleich abläuft wird
             über $\sigma_\text{T}$ definiert. Die Lösungsfunktion ist abhängig
             von der Geometrie des Stabes und den angenommenen Anfangsbedingungen.

In der konfiguration wird der Stab nun abwechselnd erwärmt und  danach wieder
abgekühlt. Lässt sich diesem Vorgang eine gleichbleibende Periodendauer zuordnen und
 ist der Stab ziemlich lang, bildet sich eine räumliche und zeitliche Temperaturwelle
  innerhalb des Stabes aus. Diese besitzt die Form:
  \begin{equation}
    T(x,t) = T_\text{max}e^{-\sqrt{{\frac{\omega \rho c}{2\kappa}}x}}\cos \left( \omega t - \sqrt{\frac{\omega \rho c}{2\kappa}}x \right)
  \end{equation}
mit der Phasengeschwindigkeit
\begin{equation}
  v = sqrt{\frac{2 \kappa \omega}{\rho c}}\text{.}
\end{equation}
Sind zwei Amplituden $A_\text{nah}$ und $A_\text{fern}$ zu zwei Messstellen
$x_\text{nah}$ und $x_\text{fern}$ und die dazwischenliegende Phasenverschiebung $\Delta t$
 der Temperaturwelle bekannt, lässt sich $\kappa$ mithilfe dieser bestimmen. Für $\kappa$ folgt:
 \begin{equation}
   \kappa = \frac{\rho c (\Delta x)²}{2 \Delta t (\ln(A_\text{nah})-\ln(A_\text{fern})}
 \end{equation}
