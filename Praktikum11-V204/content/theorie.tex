
\section{Theorie}
\label{sec:Theorie}
Eine Materialeigenschaft eines Körpers ist die Speicherung von Wärme bzw. die Abgabe
 von bereits gespeicherter Wärme. Existiert nun auf auf einem solchen Körper
 ein Temperaturungleichgewicht zwischen verschiedenen Stellen, kommt es zum Temperaturausgleich.
 Hierzu wird Wärmeenergie über den Körper transportiert.
Mögliche Vorgänge des Wärmetransportes sind Wärmestrahlung, Konvektion oder
  Wärmeleitung. Im Folgenden wird sich auf den letzten Vorgang beschränkt. Im
  allgemeinen basiert die Wärmeleitung in festen Material zu großen Teilen auf Phonomen und
   frei beweglichen Elektronen. Ist das Material des Körpers ein Metall können auch die Effekte der
    bestehenden Gitterstrukturen vernachlässigt werden. Im Folgenden wird ein metallischer Stab der
    Länge $L$, sowie der Querschnittfläche $A$ betrachtet. Er besitzt die
    gleichmäßige Dichte $\rho$ und die spezifische Wärme $c$. Existiert nun ein
    Temperaturungleichgewicht $T$ zwischen den beiden Stabenden, folgt für den Fluss
     der Wärmeenergie $\text{d}Q$ durch $A$ in der Zeit $\text{d}t$ :
     \begin{equation}
       \text{d}Q = -\kappa A \frac{\partial T}{\partial x} \text{d}t \label{eq:form1}
       \end{equation}
       Der Faktor \kappa beschreibt dabei die Wärmeleitfähigkeit und ist eine
        Konstante des Materials. Nach Konvention fließt die Wärmeenergie vom
         energiereichen zum energieärmeren Stabende. Analog folgt für die Wärmestromdichte $j_\text{w}$:
         \begin{equation}
           j_\text{w} = -\kappa \frac{\partial T}{\partial x} \label{eq:form2}
           \end{equation}
           Auf Basis der Gleichungen \ref{eq:form1} und \ref{eq:form2}, sowie die Kontinuitätsgleichung folgt
            die eindimensionale Wärmeleitungsgleichung
            \begin{equation}
              \frac{\partial T}{\partial t} = \frac{\kappa}{\rho c}\frac{\partial² T}{\partial x²}\label{eq:form3}
              \end{equation}
              mit der materialspezifischen Temperaturleitfähigkeit
              \begin{equation}
                \sigma_\text{T} = \frac{\kappa}{\rho c}\text{.}\label{eq:form4}
                \end{equation}
             Sie beschreibt
             die zeitliche und räumliche Abhängigkeit des Temperaturverlaufes im Stab.
             Die Schnelligkeit, mit welcher der Temperaturausgleich abläuft wird
             über die Temperaturleitfähigkeit festgelegt. Die Lösungsfunktion ist abhängig
             von der Geometrie des Stabes und den angenommenen Anfangsbedingungen.

In der folgenden Konfiguration wird der Stab nun periodisch erwärmt und
 wieder abgekühlt.
 Im zuge dessen bildet sich eine räumliche und zeitliche Temperaturwelle
  innerhalb des Stabes aus. Ist der Stab sehr lang im Vergleich zur Breite, lassen
   sich die Wellen durch folgende Formel ausdrücken:
  \begin{equation}
    T(x,t) = T_\text{max}e^{-\sqrt{{\frac{\omega \rho c}{2\kappa}}x}}\cos \left( \omega t - \sqrt{\frac{\omega \rho c}{2\kappa}}x \right)\label{eq:form5}
  \end{equation}
mit der Phasengeschwindigkeit
\begin{equation}
  v = \sqrt{\frac{2 \kappa \omega}{\rho c}}\text{.}\label{eq:form6}
\end{equation}
Sind die zugehörigen Amplituden $A_\text{nah}$ und $A_\text{fern}$ zu zwei Messstellen
$x_\text{nah}$ und $x_\text{fern}$ im Abstand $\Delta x$ und die dazwischenliegende Phasenverschiebung $\Delta t$
 der Temperaturwelle bekannt, lässt sich $\kappa$ mithilfe dieser bestimmen. Für $\kappa$ folgt:
 \begin{equation}
   \kappa = \frac{\rho c (\Delta x)²}{2 \Delta t (\ln(A_\text{nah})-\ln(A_\text{fern}))}\label{eq:form7}
 \end{equation}
