
\section{Diskussion}
\label{sec:Diskussion}
Zunächst muss zwischen den Ergebnissen der statischen und der dynamischen Methode
 unterschieden werden. Anhand der Graphen der statischen Methode lassen sich
 die Wärmeleitkoeffizienten nur in Reihe vom niedrigstem zum höchstem bringen,
 es lassen sich jedoch keine genauen Werte dieser bestimmen. Da Literaturwerte
 von $\kappa$ zur Bestimmung von $\frac{\Delta Q}{\Delta t}$ verwendet werden
 bringen diese keine neuen Ergebnisse. Zusätzlich sind diese auch noch von der
 verwendeten Spannung und Stromstärke abhängig. Beide Graphen, welche die
 Temperaturdifferenz auf einem Stab zeigen einen Peak,
 ca. $\SI{100}{\second}$ nach Beginn der Messung. Dieser ist beim
 Differenzgraphen zwischen $T1$ und T2 jedoch um einiges ausgeprägter als bei
 Differenzgraphen von $T7$ und $T8$. Grund hierfür ist die weitaus bessere
 Wärmeleitfähigkeit von Messing gegenüber Edelstahl, weswegen sich das
 Material am nahen Messpunkt nach kurzer Zeit bereits aufheizen bevor Wärme
 überhaupt zum fernen Thermoelement gelangen konnte. Beim Edelstahlstab geschieht
 dieser Aufwärmvorgang langsamer, weswegen der Differenzpeak kleiner ist. Im
 Gegensatz dazu lassen sich mit der dynamischen Methode konkrete Werte für die
 Wärmeleitkoeffizienten bestimmen. Beim Vergleich der ermittelten Koeffizienten
  fällt jedoch auf, dass diese große Abweichungen zu Literaturwerten aufzeigen.
  So liegt das ermittelte $\kappa$ von Messung mit
  $125 \pm \SI{5}{\frac{\watt}{\kilogram\kelvin}}$ ca. $\SI{20}{\procent}$ über
  dem Literaturwert von $\SI{105}{\frac{\watt}{\kilogramm\kelvin}}$, beim $\kappa$
  Aluminium liegt sie bei ca. $\SI{10}{\procent}$. Der ermittelte Wert für
  Edelstahl ist mit $12,9 \pm \SI{0.5}{\frac{\watt}{\kilogram\kelvin}}$ sogar
   nur ca. halb so groß wie der Literaturwert. Dieser liegt bei $\SI{20}{\frac{\watt}{\kilogram\kelvin}}$.
   Die Hauptursache der Abweichungen liegt wahrscheinlich in der Methode der manuellen
   Auswertung. Die gemessenen Größen liegen oftmals nur im Bereich weniger Millimeter.
   Da bei der Auswertung mithilfe eines Geodreiecks jedoch nur eine
   Genauigkeit von ca. $\SI{0.5}{\mili\meter}$ verfälschen
   Messungenauigkeiten die Ergebnisse deutlich.
  %  Da mithilfe eines Geodreiecks nur eine Genauigkeit von ca.
   %$\SI{0.5}{\mili\meter}$ erreicht wird, die gemessenen Größen , allen voran
   %die Phasendifferenz, jedoch nur im Bereich weniger Millimeter liegen,
   %verfälschen Messungenauigkeiten die Ergebnisse deutlich.
