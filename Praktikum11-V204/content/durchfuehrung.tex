
\section{Durchführung}
\label{sec:Durchführung}

\subsection{Statische Methode}
Zunächst wird eine statische Methode zur Bestimmung der Wärmeleitkoeffizienten
  durchgeführt. Hierzu werden am Explorer GLX zunächst alle Thermoelemente aktiv geschaltet.
  Dies geschieht über den Menüunterpunkt "SENSOREN". Über den Menüpunkt "DIGITAL"
   wird eine Abtastrate von $\SI{5}{\second}$ eingestellt. An der Energiequelle wird
    eine Spannung von $\SI{5}{\volt}$ bei maximaler Stromstärke eingerichtet. Um
     einer Wärmeabgabe der Metalle über die Umgebung vorzubeugen, werden
     Isolierungen über alle Stäbe gelegt. Nach einer Zeit von $\SI{700}{\second}$
     werden die Temperaturen der Thermoelemente T1, T4, T5 und T8 notiert. Die Messreihe
      wird beendet sobald das Thermoelement T7 eine Temperatur von ca.
       $\SI{45}{\degreeCelsius}$ erreicht hat oder eine Messzeit von über
       $\SI{40}{\minute}$ erreicht ist. Mithilfe des Xplorer GLX wird jeweils
       ein Graph mit T1 und T4 sowie einer mit T5 und T8 erstellt und ausgedruckt.
       Zusätzlich werden auch die Differenzgraphen von T2-T1 und T7-T8 ausgedruckt.
       Die Stäbe sollen vor Beginn des nächsten Versuches auf unter $\SI{30}{\degreeCelsius}$ abgekühlt sein.

\subsection{Dynamische Methode}
Nun wird eine dynamische Methode zur Bestimmung der Wärmeleitkoeffizienten verwendet.
       Die Abtastrate wird auf $\SI{2}{\second}$ eingestellt. Die Spannung wird
       auf $\SI{8}{\volt}$ erhöht. Die Stäbe werden nun unter einer Periode von $\SI{80}{\second}$
        periodisch erwärmt und wieder abgekühlt. Hierzu wird der Schalter alle
        $\SI{40}{\second}$ umgelegt. Die Messung wird nach frühestens 10 Perioden
        beendet. Es wird jeweils ein Graph des breiten Messingstabes (T1 und T2)
        und einer des Alluminiumstabes( T5 und T6) erstellt und gedruckt.
        Nachdem die Stäbe wieder abgekühlt sind wird der Versuch unter einer
        Periodendauer von $\SI{200}{\second}$ wiederholt. Diesmal wird die Messung
         beendet, wenn das erste Thermoelement $\SI{80}{\degreeCelsius}$ erreicht.
         Es wird ein Graph des Edelstahls (T7 und T8) erstellt und ausgedruckt.


