
\section{Durchführung}
\label{sec:Durchführung}

\renewcommand{\labelenumi}{\alph{enumi})}
\begin{enumerate}
  \item Zunächst wird eine statische Methode zur Bestimmung der Wärmeleitkoeffizienten
  durchgeführt. Hierzu werden am Explorer GLX zunächst alle Thermometer zugeschaltet.
  Dies geschieht über den Menüunterpunkt "SENSOREN". Über den Menüpunkt "DIGITAl"
   wird eine Abtastrate von $\SI{5}{\second}$ eingestellt. An der Energiequelle wird
    eine Spannung von $\SI{5}{\volt}$ bei maximaler Stromstärke eingerichtet. Um
     einer Wärmeabgabe der Metalle über die Umgebung vorzubeugen, werden
     Isolierungen über alle Stäbe gelegt. Nach einer Zeit von $\SI{700}{\second}$
     werden die Temperaturen der Thermoelemente T1, T4, T5 und T8 notiert. Die Messreihe
      wird beendet sobald entweder das Thermoelement T7 eine Temperatur von ca.
       $\SI{45}{\degreeCelsius}$ erreicht hat oder aber eine Messzeit von über $\SI{40}{\minutes}$ erreicht ist.
\end{enumerate}
