\section{Auswertung}
\label{sec:Auswertung}


Die Graphen wurden sowohl mit Matplotlib \cite{matplotlib} als auch NumPy \cite{numpy} erstellt. Die
Fehlerrechnung wurde mithilfe von Uncertainties \cite{uncertainties} durchgeführt.
Die Konstanten $k$, $\hslash$, $e_0$, $m_0$, $u_0$ und $N_\text{A}$ sind vom NIST \cite{nistgov}. Die Größe des betrachteten Gegenstandes beträgt $\SI{2.8}{\centi\meter}$.



\subsection{Überprüfung der pyhsikalischen Gesetze der geometrischen Optik}

\begin{figure}
 \centering
 \includegraphics[width=\linewidth-70pt,height=\textheight-70pt,keepaspectratio]{build/erstelinse.pdf}
 \caption{Darstellung von Gegenstandsweite und Brennweite der Sammellinse mit $f = \SI{10}{\centi\meter}$}
 \label{fig:erste}
\end{figure}

\begin{table}
	\centering
	\caption{Die gemessenen Daten der Sammellinse mit $f = \SI{10}{\centi\meter}$}
	\input{build/taberste.tex}
	\label{tab:b}
\end{table}

\begin{table}
	\centering
	\caption{Die experimentellen Verhältnisse}
	\input{build/tabverh.tex}
	\label{tab:vv}
\end{table}

Zunächst wird die Wirksamkeit der Formeln \eqref{HJKHJKH} und \eqref{hHK} überprüft.
Mit den Daten aus Tabelle \ref{tab:erste} ergibt sich der Graph in Abb. \ref{fig:erste}. Nach Formel \eqref{HJHJ} ergeben sich zudem die Brennweiten in Tabelle \ref{tab:b}
, welche einen Mittelwert von $\SI{9.55(2)}{\centi\meter}$ liefert. Der siebte Wert wurde aufgrund zu hoher Abweichungen nicht berücksichtigt. Beim Vergleich mit dem Graphen erscheint dieser Wert realistisch.
Für die Formel \eqref{hGHJ} ergeben sich die experimentellen Werte in Tabelle \ref{tab:vv}. Diese liefern für $\frac{b}{g}$ den Mittelwert
$\SI{0.7(2)}{}$, für $\frac{B}{G}$ den Mittelwert $\SI{0.9(2)}{}$.



\subsection{Bestimmung der Brennweite einer Wasserlinse bei festem Wasserdruck}

\begin{table}
	\centering
	\caption{Die gemessenen Daten der Wasserlinse}
	\input{build/tabwasser.tex}
	\label{tab:w}
\end{table}

\begin{figure}
 \centering
 \includegraphics[width=\linewidth-70pt,height=\textheight-70pt,keepaspectratio]{build/wasserlinse.pdf}
 \caption{Darstellung der Brennweite der Sammellinse mit unbekannter Brennweite.}
 \label{fig:water}
\end{figure}

Nun wird die Brennweite einer unbekannten Linse bestimmt. Mit den bestimmten Werten in Tabelle \ref{tab:w} folgt der Graph in Abb. \ref{fig:water}. Nach Formel \eqref{HJHK} ergeben sich zudem die Brennweiten in Tabelle \ref{tab:w}. Damit ergibt sich eine mittlere Brennweite von $\SI{8.45(2)}{\centi\meter}$.



\subsection{Bestimmung der Brennweite nach Bessel}

\begin{table}
	\centering
	\caption{Die gemessenen Daten zur Bestimmung der Brennweite nach Bessel}
	\input{build/tabbessel.tex}
	\label{tab:bessel}
\end{table}



 Als nächstes wird die Brennweite über die Methode nach Bessel bestimmt. Mit den Messungen ergeben sich die Daten in Tabelle \ref{tab:bessel}. Mithilfe von Formel von \eqref{J} folgen zusätzlich die zugehörigen Brennweiten. Diese liefern eine mittlere Brennweite von $\SI{9.67(2)}{\centi\meter}$. Das Ergebnis dieser Methode liegt damit näher am Theoriewert von $f = \SI{10}{\centi\meter}$, als die nach Formel \eqref{jKJ} berechnete.

 \subsection{Betrachtung der Effekte aufgrund von chromatischen Abberation}

 \begin{table}
 	\centering
 	\caption{Die gemessenen Daten zur Bestimmung der chromatischen Abberation unter rotem und blauem Licht.}
 	\input{build/tabbesself.tex}
 	\label{tab:besself}
 \end{table}

Die Methode nach Bessel wird nun nochmals jeweils für blaues und rotes Licht wiederholt. Mit den gemessenen Daten ergeben sich die Brennweiten in Tabelle \ref{tab:besself}. Daraus folgt für das rote Licht eine mittlere Brennweite von $\SI{9.85(1)}{\centi\meter}$, für das blaue hingegen eine von $\SI{9.78(2)}{\centi\meter}$. Daraus lässt sich folgern, dass die Brennweite bei rotem Licht größer ausfällt als bei blauem.


\subsection{Brennweitenbestimmung eines Linsensystems mit der Methode nach Abbe}

\begin{figure}
 \centering
 \includegraphics[width=\linewidth-70pt,height=\textheight-70pt,keepaspectratio]{build/abbeg.pdf}
 \caption{Die Gegenstandsweite aufgetragen gegen $1+\frac{1}{V}$.}
 \label{fig:abbeg}
\end{figure}

\begin{figure}
 \centering
 \includegraphics[width=\linewidth-70pt,height=\textheight-70pt,keepaspectratio]{build/abbeb.pdf}
 \caption{Die Bildweite aufgetragen gegen $1+V$.}
 \label{fig:abbeb}
\end{figure}


\begin{table}
 \centering
 \caption{Die gemessenen Daten zur Bestimmung der Brennweite eines Linsensystems.}
 \input{build/tababbe.tex}
 \label{tab:abbe}
\end{table}


Zur Bestimmung der Brennweite eines Linsensystems wird nun die Methode nach Abbe verwendet. Da die benötigten Hauptebenen nicht bekannt sind, wird zur Bestimmung von Bildweite und Gegenstandsweite ein Punkt $A$ gewählt. Dieser wurde auf die Mitte der Zerstreuungslinse festgesetzt. Damit ergeben sich die Weiten in Tabelle \ref{tab:abbe}. Mit den Formeln \eqref{jJ} und \eqref folgen die Graphen in Abb. \ref{fig:abbeg} und \ref{fig:abbeb}. Mithilfe einer linearen Ausgleichsrechnung ergibt sich aus der Regression der Gegenstandsweite eine Brennweite von $\SI{14.5(4)}{\centi\meter}$, aus der Bildweitenregression eine Brennweite von $\SI{14.7(5)}{\centi\meter}$.
Gemittelt folgt eine Brennweite von $14.6(3)$. Mit Formel \eqref{JJ} lässt sich die theoretische Brennweite des Linsensystems ermitteln.
Bei einer Brennweite der Zerstreuunglinse von $f_\text{Z} = \SI{-10}{\centi\meter}$, einer von $f_\text{S} = \SI{10}{\centi\meter}$ bei der Sammellinse sowie einem Linsenabstand von $\SI{6}{\centi\meter}$ liefert sie eine theoretische Gesamtbrennweite von $\SI{16.7}{\centi\meter}$. Im Vergleich liegt der experimentelle Wert damit um $\SI{2.1(3)}{\centi\meter}$ oder $\SI{12(2)}{\percent}$ neben dem Theoriewert. Die Gründe hierfür müssen in der Diskussion geklärt werden.
 Für die Hauptachse der Zerstreunungslinse ergibt sich ein Abstand zum gewählten Punkt $A$ von $\SI{-8(1)}{\centi\meter}$, für die der Sammellinse einer von $\SI{15.8(9)}{\centi\meter}$. Auch diese Ergebnisse müssen in der Diskussion erörtert werden.
