\section{Durchführung}
\label{sec:Durchführung}

Die Halogenlampe, der Gegenstand, eine Sammellinse mit bekannter Brennweite ($f=100$) und der Schirm werden in dieser Reihenfolge auf der optischen Bank angeordnet. Der Schirm wird nun verschoben, bis das Bild des Gegenstandes möglichst scharf auf dem Schirm abgebildet wird. Es wird der Abstand des Gegenstandes zur Linse und der Abstand zwischen Linse und Schirm notiert. Diese Messung wird für 10 verschiedene Abstände zwischen Gegenstand und Linse durchgeführt, dabei wird 5 Mal auch die Größe des Bildes notiert.

Anschließend wird diese Messreihe noch einmal mit einer mit Wasser gefüllten Linse unbekannter Brennweite durchgeführt (die Größe des Bildes wird hier nicht gemessen).

Nun soll die Brennweite nach der Methode von Bessel bestimmt werden. Hierzu wird der selbe Aufbau wie bei der ersten Messung verwendet, wobei nun jedoch nicht das Bild durch verschieben des Schirmes für verschiedene Abstände zwischen Gegenstand und Linse scharfgestellt wird, sondern der Abstand zwischen Gegenstand und Schirm jeweils gleich bleibt und nur die Linse verschoben wird. Es gibt für jeden Abstand zwischen Gegenstand und Schirm zwei Positionen der Linse, an welchen das Bild scharf wird. Diese Positionen, sowie der Abstand zwischen Gegenstand und Schirm werden jeweils notiert. Diese Messungen werden für 10 verschiedene Abstände zwischen Gegenstand und Schirm durchgeführt. Es sollte darauf geachtet werden, dass der Abstand zwischen Gegenstand und Schirm mindestens vier mal so groß ist, wie die Brennweite der Linse. Diese Messreihe wird mit jeweils 5 verschiedenen Abständen zwischen Gegenstand und Schirm mit blauen bzw. roten Licht wiederholt. Das blaue bzw. rote Licht wird durch einen blauen bzw. roten Filter vor dem Gegenstand erreicht.

Zum Schluss soll die Brennweite und die Lage der Hauptebenen eines Linsensystems nach der Methode von Abbe bestimmt werden. Dafür wird erste Messreihe durchgeführt, diesmal jedoch mit einer Zersreuungslinse mit bekannter Brennweite ($f=-100$) in festem Abstand vor der Sammellinse. Zudem wird jedes mal die Größe des Bildes notiert.