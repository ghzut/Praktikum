
\section{Theorie}
\label{sec:Theorie}

Geometrische Optik bezeichnet ein Teilgebiet der Physik in welchem die Ausbreitung des Lichtes durch Strahlen beschrieben wird. Diese werden beim Übergang zwischen zwei Medien mit verschiedenen Brechungsindizes nach dem Brechungsgesetz gebrochen. In der geometrischen Optik werden Linsen verwendet. Es wird zwischen Sammellinsen und Zerstreuungslinsen unterschieden. Eine Sammellinse bündelt einfallende parallel zur optischen Achse verlaufende Lichtstrahlen in einem Punkt, dem Brennpunkt in der Brennweite $f$. Bei der Zerstreuungslinse werden die parallelen Lichtstrahlen gestreut, sodass die Verlängerungen der gestreuten Strahlen sich vor der Linse in einem Punkt kreuzten. Die Brennweite ist somit negativ. Ist die Linse sehr dünn, sodass die Brechung annähernd in einer Mittelebene stattfindet wird von einer dünnen Linse ansonsten von einer dicken Linse gesprochen. Eine dicke Linse kann angenähert werden, indem die Mittelebene durch zwei Hauptebenen ersetzt wird, an welchen die Strahlen gebrochen werden. Um das Bild eines Gegenstandes zu konstruieren können drei Strahlen betrachtet werden. Der Parallelstrahl $P$ verläuft vom Gegenstand aus parallel zur optischen Achse und wird anschließend an der Linse entsprechend gebrochen. Der Mittelpunktstrahl $M$ geht durch die Mitte der Linse, ohne dass sich die Richtung des Strahles ändert. Der Brennpunktstrahl $B$ geht durch den Brennpunkt vor der Linse und wird an der Linse gebrochen, sodass er anschließend parallel zur optischen Achse verläuft. Dort wo sich die (verlängerten) Strahlen schneiden wird der Punkt, von welchem die drei Strahlen am Gegenstand ausgingen scharf abgebildet. Alle Punkte zusammen ergeben ein Bild, welches reelles Bild genannt wird, falls die Strahlen zur Konstruktion nicht verlängert werden müssen ansonsten virtuelles Bild. Das Verhältnis der Bildgröße $B$ zur Gegenstandsgröße $G$ heißt Abbildungsmaßstab $V$ und es folgt aus den Konstruktionsregeln
\begin{equation}
 V = \frac{B}{G}=\frac{b}{g},
\end{equation}
wobei $b$ der Abstand des Bildes und $g$ der Abstand des Gegenstandes zur nächsten Mittelebene bzw. Hauptebene ist. Für dünne Linsen folgt auch die Linsengleichung
\begin{equation}
	\frac{1}{f}=\frac{1}{b}+\frac{1}{g}
\end{equation}

