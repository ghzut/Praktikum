
\section{Diskussion}
\label{sec:Diskussion}
Die Auswertung hat einige Ergebnisse gebracht, welche nun noch zu diskutieren sind.
Die Bestimmung der Brennweite der ersten Linse zeigt Abweichungen von
 ca. $\SI{5}{\percent}$ bezüglich des Theoriewertes von $f = \SI{10}{\centi\meter}$.
  Die Graphen zeigen eine ähnliche Abweichung. Ursache hierfür sind wahrscheinlich
   Messungenauigkeiten während des Messvorgangs, da sich das Bild insbesondere für
   größere Bildweiten nicht eindeutig scharf stellen lies. Da versucht wurde immer dieselbe Stelle
    scharf zu stellen, wird dies die Werte systematisch verändert haben. Die Auswirkungen dieses Messfehlers
   zeigen sich insbesondere beim Vergleich der Verhältnisse von $B/G$ und $b/g$,
    welche teilweise fast $\SI{50}{\percent}$ voneinander abweichen. Des weiteren können
     die Werte zusätzlich durch eine leichte Neigung der Halogenleuchte verfälscht worden sei,
     da die überprüften Formeln nur für kleine Winkel gelten. Die
      Brennweitenbestimmung der Wasserlinse liefert einen Wert von $\SI{8.45(2)}{\centi\meter}$.
      Im Vergleich mit dem zugehörigen Graphen in Abb. \ref{fig:water} erscheint dieses Ergebnis realistisch.
       Die bestimmte experimentelle Brennweite mit der Besselmethode zeigt eine
        bessere Abweichung bezüglich des Theoriewertes, als die mit der Linsengleichung bestimmte.
        Es ist jedoch nicht erkennbar, ob dies an einer besseren Toleranz der Formel mit
        Messungenauigkeiten liegt oder ob die neuen Messwerte besser bestimmt worden sind.
         Der Effekt der chromatische Abberation zeigt sich in den bestimmten
         Brennweiten in Form eines Unterschiedes von ca. $\SI{7}{\milli\meter}$. Es
          ist jedoch davon auszugehen, dass die Hauptgründe hierfür nicht in der
           chromatischen Abberation liegen, sondern eher in der mangelnden
            Messgenauigkeit und in der spährischen Abberation. Letztere wird durch
            die bereits genannte Neigung der Leuchte noch verstärkt. Auch die
            Brennweitenbestimmung mit der Methode nach Abbe zeigt einige Abweichungen.
            Da der Messvorgang analog zur ersten Messung verläuft, sind die
            gleichen Fehlerquellen für die Messung anzunehmen. Dementsprechend
            fällt Brennweite systematisch zu gering aus, es lässt sich jedoch auf die Messungenauigkeit des Laboranten zurückführen..
            Da die Messpunkte
             nicht in Nähe der Y-Achsen liegen, wirkt sich dies noch
             stärker auf die bestimmten Hauptebenen aus.
