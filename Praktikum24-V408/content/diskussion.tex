
\section{Diskussion}
\label{sec:Diskussion}
Die Auswertung hat einige Ergebnisse gebracht, welche nun noch zu diskutieren sind.
Die Bestimmung der Brennweite der ersten Linse zeigte Abweichungen von
 ca. $\SI{5}{\percent}$ bezüglich des Theoriewertes von $f = \SI{10}{\centi\meter}$.
  Die Graphen zeigten eine ähnliche Genauigkeit. Ursache hierfür sind
   Messungenauigkeiten während des Messvorgangs, da sich das Bild insbesondere für
   größere Bildweiten nicht eindeutig scharf stellen lies. Die Auswirkungen dieses Messfehlers
   zeigen sich insbesondere beim Vergleich der Verhältnisse von $B/G$ und $b/g$,
    welche Abweichungen von teils fast $\SI{50}{\percent}$. Desweiteren könnten
     die Werte durch zusätzlich verfälscht worden sein, da die Halogenleuchte leicht
      nach unten geneigt war und die geprüften Formeln nur für kleine Winkel gelten. Die
      Brennweitenbestimmung der Wasserlinse ergibt einen Wert von $\SI{8.45(2)}{\centi\meter}$.
      Im Vergleich mit dem zugehörigen Graphen in Abb. \ref{fig:water} erscheint dieses Ergebnis realistisch.
       Die bestimmte experimentelle Brennweite mit der Besselmethode zeigte eine
        bessere Abweichung bezüglich des Theoriewertes als die mit der Linsengleichung bestimmte.
        Es ist jedoch nicht erkennbar, ob dies an einem Umgang mit den verursachten
         Messungenauigkeiten liegt oder ob die neuen Messwerte besser bestimmt worden sind.
         Die chromatische Abberation zeigt sich in den bestimmten
         Brennweite in Form eines Unterschiedes von ca. $\SI{7}{\milli\meter}$. Es
          ist jedoch davon auszugehen, dass die Hauptgründe hierfür nicht in der
           chromatischen Abberation liegen, sondern eher in der mangelnden
            Messgenauigkeit und in der spährischen Abberation. Auch die
            Brennweitenbestimmung mit der Methode nach Abbe zeigt einige Abweichungen.
            Da der Messvorgang analog zur ersten Messung verläuft, sind die
            gleichen Fehlerquellen während der Messung anzunehmen. Dementsprechend
            liegt die Abweichung der bestimmten Brennweite im Rahmen der Messungenauigkeit.
            Da die Messpunkte
             nicht in Y-Achsennähe liegen, wirken sich die Messungenauigkeiten
             in der Brennweitenbestimmung stärker aus.
