
\section{Theorie}
\label{sec:Theorie}
Um die Natur der Elektronenhülle zu untersuchen werden Elektronenstöße untersucht.
Hierzu werden Atome mit Elektronen beschossen. Aus den auftretenden Energieverlusten
lässt sich anschließend auf die Natur der Elektronen schließen. Auch der Frank-Hertz
Versuch, bei welchem Hg-Atome verwendet werden,
folgt diesem Konzept. Es treten nun sowohl elastische als auch unelastische Stöße auf.
Für letztere folgt bei nicht relativistischen Betrachtung der Energieunterschied:
\begin{equation}
  m_0 \frac{v_\text{vor}²}{2} - m_0 \frac{v_\text{nach}²}{2} = \Delta E = E_1 - E_0
  \end{equation}
Die auftretende Energiedifferenz $\Delta E$ wird dazu verwendet, das beschossene
Atom von seinem Grundzustand $E_0$ in den angeregten Zustand $E_1$ anzuheben.
In der Theorie folgt der Aufbau des Frank-Hertz Versuches nun nach Abb.
\ref{fig:djkshjkdhjfdhjfk} . Den Kern bildet eine mit Quecksilberdampf versehene und ansonsten evakuierte
Diode. Elektronen werden mithilfe einer Glühspannung aus einer Glühkathode emittiert
und mittels einer Beschleunigungsspannung bis zu einer in der Mitte angebrachten
Elektrode beschleunigt. Die Elektronen haben danach die Energie
\begin{equation}
  m_0 \frac{v_\text{vor}²}{2} = e_0 U_\text{B}\text{.}
  \end{equation}
   Anschließend werden die Elektronen mit einer Auffängerelektrode
gefangen und der daraus resultierende Strom gemessen.
