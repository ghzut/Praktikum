
\section{Diskussion}
\label{sec:Diskussion}
Die obige Auswertung liefert einige Ergebnisse, welche nun zu diskutieren sind.
 Die in $5.3$ bestimmten Kontaktpotentiale in Abb. \ref{tab:erg} basieren auf einer
jeweils anderen Kurve. Aufgrund des größeren Fehlers von $K_\text{b}$ ist das
Ergebnis von $K_\text{a}$ jedoch aussagekräftiger und vorzuziehen. Die Fehler der anderen, bestimmten
Größen sind jedoch ungleich kleiner.
So liegt der ermittelte Wert von $E_1$ mit seiner 1-Sigma Umgebung im
Bereich des Literaturwertes. Der relative Fehler beträgt weniger als 0.5 $\textperthousand$.
Die daraus ermittelte Wellenlänge zeigt dementsprechend dieselbe, geringe Abweichung. Auch die
 bestimmte Ionisierungsenergie liegt im Bereich des Literaturwertes, wenngleich die Unsicherheit geringfügig größer ist.
Ursachen der Unsicherheiten liegen in Temperaturschwankungen während des
Messvorgang sowie in Ablesefehlern bei der manuellen Auswertung. Die relativ
großen Ungenauigkeiten der bestimmten Kontaktpotentiale lassen sich zum Teil auf einen nicht näher bestimmbaren Fehler
in der Messapparatur zurückführen. Dieser verursacht den in Abb. \ref{fig:a2A}
in der Mitte erkennbaren Steigungshügel.
