
\section{Diskussion}
\label{sec:Diskussion}
\begin{table}
	\caption{Die Ergebnisse aus der Auswertung.}
	\centering
	\input{build/Ergebnisse.tex}
	\label{tab:erg}
\end{table}
Die obige Auswertung liefert einige Ergebnisse, welche nun zu diskutieren sind.
 Die in \ref{subsec:Bestimmung_des_Kontaktpotentials} und \ref{subsec:Bestimmung_der_1._Anregungsenergie_von_Quecksilber} bestimmten Kontaktpotentiale in Tabelle \ref{tab:erg} basieren auf einer
jeweils anderen Kurve. Aufgrund des größeren Fehlers von $K_\text{b}$ ist das
Ergebnis von $K_\text{a}$ jedoch aussagekräftiger und vorzuziehen. Die Fehler der anderen bestimmten
Größen sind jedoch ungleich kleiner.
So liegt der ermittelte Wert von $E_1$ mit seiner $1\sigma$-Umgebung im
Bereich des Literaturwertes. Der relative Fehler beträgt weniger als $\SI{0.04}{\percent}$.
Die daraus ermittelte Wellenlänge zeigt dementsprechend dieselbe, geringe Abweichung. Auch die
 bestimmte Ionisierungsspannung liegt im Bereich des Literaturwertes, wenngleich die Unsicherheit geringfügig größer ist.
Ursachen der Unsicherheiten liegen in Temperaturschwankungen während des
Messvorgang sowie in Ablesefehlern bei der manuellen Auswertung. Die relativ
große Ungenauigkeit bei dem Wert von $K_\text{b}$ lässt sich auf die stark fehleranfällige Bestimmungsweise des Wertes zurückführen. Auch liegt vermutlich ein nicht näher bestimmbarer Fehler in der Messapparatur vor. Dieser verursacht den in Abb. \ref{fig:a2A} bei geringen Spannungen ein Peak. Zu Abweichungen von den theoretischen Werten kann es zudem auch durch mangelnde Abschirmung der Kabel kommen.
