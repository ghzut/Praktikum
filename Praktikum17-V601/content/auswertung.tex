\section{Auswertung}
\label{sec:Auswertung}


Die Graphen wurden sowohl mit Matplotlib \cite{matplotlib} als auch NumPy \cite{numpy} erstellt. Die
Fehlerrechnung wurde mithilfe von Uncertainties \cite{uncertainties} durchgeführt.
Die Konstanten $\hbar$, $e_0$ und $c$ sind vom NIST \cite{nistgov}.

\subsection{Berechnung der mittleren freien Weglänge und Vergleich mit dem Abstand zwischen der BESCH und GITTER}
Die mittlere Weglänge $\overline{w}$ lässt sich bei bekannter Temperatur $T$ mit Formel \ref{eq:w} berechnen.
Bei der ersten Messung bei $T=\SI{25}{\degreeCelsius}$ ergibt sich
\begin{displaymath}
	\overline{w}=\SI{ 0.55}{\centi\meter}\text{.}
\end{displaymath}
Also ist der Abstand zwischen der BESCH und GITTER $a=\SI{1}{\centi\meter}$ ca. $1.8$ mal so groß wie die mittlere Weglänge.
Bei der zweiten Messung bei $T=\SI{150(10)}{\degreeCelsius}$ liegt sie bei
\begin{displaymath}
\overline{w}=\SI{6(2)e-4}{\centi\meter}\text{.}
\end{displaymath}
Dies entspricht einem Faktor von $frac{a}{\overline{w}}=\num{1.7(6)e3}$. In der dritten Messung bei $T=\SI{105(5)}{\degreeCelsius}$ ergibt sich
\begin{displaymath}
\overline{w}=\SI{4(1)e-3}{\centi\meter}
\end{displaymath}
mit dem Verhältnis $\frac{a}{\overline{w}}=\num{2(1)e2}$. Eine mittlere Weglänge
\begin{displaymath}
\overline{w}=\SI{2(1)e-4}{\centi\meter}
\end{displaymath}
ergibt sich schließlich bei einer Temperatur $T=\SI{180(20)}{\degreeCelsius}$ mit dem Verhältnis $\frac{a}{\overline{w}}=\num{5(3)e3}$. Da der Faktor $\frac{a}{\overline{w}}$ nur bei den Temperaturen $T=\SI{150(10)}{\degreeCelsius}$ und $T=\SI{180(20)}{\degreeCelsius}$ in dem Bereich von 1000-4000 liegt, sind nur bei diesen die Beobachtung des Franck-Hertz-Effektes zu erwarten \cite{V601}.


\subsection{Bestimmung der Umrechnungsfaktoren}

\begin{center}
	\begin{table}
		\begin{minipage}[t]{0.5\textwidth}
			\caption{Erste}
			\centering
			\input{build/a1Abstaende.tex}
		\end{minipage}
		\begin{minipage}[t]{0.5\textwidth}
			\caption{Zweite}
			\centering
			\input{build/a2Abstaende.tex}
		\end{minipage}
	\end{table}
\end{center}
\begin{center}
	\begin{table}
		\begin{minipage}[t]{0.5\textwidth}
			\caption{Dritte}
			\centering
			\input{build/bneuAbstaende.tex}
		\end{minipage}
		\begin{minipage}[t]{0.5\textwidth}
			\caption{Vierte}
			\centering
			\input{build/cAbstaende.tex}
		\end{minipage}
	\end{table}
\end{center}
Mit den gemessenen Abständen zwischen den Markierungen in den Tabellen \ref{tab:a1Abstaende}, \ref{tab:a2Abstaende}, \ref{tab:bneuAbstaende} und \ref{tab:cAbstaende} lassen sich die Umrechnungsfaktoren $f_i$ bestimmen.


\subsection{a}
\begin{figure}
	\centering
	\caption{Die Filterkurve des Selektiv-Verstärkers.}
	\includegraphics[width=\linewidth-70pt,height=\textheight-70pt,keepaspectratio]{build/a1neu.pdf}
	\label{fig:GraphSelektiv}
\end{figure}
\begin{figure}
	\centering
	\caption{Die Filterkurve des Selektiv-Verstärkers.}
	\includegraphics[width=\linewidth-70pt,height=\textheight-70pt,keepaspectratio]{build/a2neu.pdf}
	\label{fig:GraphSelektiv}
\end{figure}
\begin{table}
	\caption{Fünfte}
	\centering
	\input{build/a1Grad.tex}
\end{table}
\begin{table}
	\caption{Sechste}
	\centering
	\input{build/a2Grad.tex}
\end{table}

\subsection{b}
\begin{table}
	\caption{Siebte}
	\centering
	\input{build/bneuDiff.tex}
\end{table}


\subsection{c}
\begin{figure}
	\centering
	\caption{Die Filterkurve des Selektiv-Verstärkers.}
	\includegraphics[width=\linewidth-70pt,height=\textheight-70pt,keepaspectratio]{build/c.pdf}
	\label{fig:GraphSelektiv}
\end{figure}
\begin{table}
	\caption{Achte}
	\centering
	\input{build/cKoordinaten.tex}
\end{table}

\begin{table}
	\caption{Neunte}
	\centering
	\input{build/Ergebnisse.tex}
\end{table}


