\section{Auswertung}
\label{sec:Auswertung}


Die Graphen wurden sowohl mit Matplotlib \cite{matplotlib} als auch NumPy \cite{numpy} erstellt. Die
Fehlerrechnung wurde mithilfe von Uncertainties \cite{uncertainties} durchgeführt.
Die Konstanten $\hbar$, $e_0$ und $c$ sind vom NIST \cite{nistgov}.

\subsection{Berechnung der mittleren freien Weglänge und Vergleich mit dem Abstand zwischen der BESCH und GITTER}
Die mittlere Weglänge $\overline{w}$ lässt sich bei bekannter Temperatur $T$ mit Formel \ref{eq:w} berechnen.
Bei der ersten Messung bei $T=\SI{25}{\degreeCelsius}$ ergibt sich
\begin{displaymath}
	\overline{w}=\SI{ 0.55}{\centi\meter}\text{.}
\end{displaymath}
Also ist der Abstand zwischen der BESCH und GITTER $a=\SI{1}{\centi\meter}$ ca. $1.8$ mal so groß wie die mittlere Weglänge.
Bei der zweiten Messung bei $T=\SI{150(10)}{\degreeCelsius}$ liegt sie bei
\begin{displaymath}
\overline{w}=\SI{6(2)e-4}{\centi\meter}\text{.}
\end{displaymath}
Dies entspricht einem Faktor von $frac{a}{\overline{w}}=\num{1.7(6)e3}$. In der dritten Messung bei $T=\SI{105(5)}{\degreeCelsius}$ ergibt sich
\begin{displaymath}
\overline{w}=\SI{4(1)e-3}{\centi\meter}
\end{displaymath}
mit dem Verhältnis $\frac{a}{\overline{w}}=\num{2(1)e2}$. Eine mittlere Weglänge
\begin{displaymath}
\overline{w}=\SI{2(1)e-4}{\centi\meter}
\end{displaymath}
ergibt sich schließlich bei einer Temperatur $T=\SI{180(20)}{\degreeCelsius}$ mit dem Verhältnis $\frac{a}{\overline{w}}=\num{5(3)e3}$. Da der Faktor $\frac{a}{\overline{w}}$ nur bei den Temperaturen $T=\SI{150(10)}{\degreeCelsius}$ und $T=\SI{180(20)}{\degreeCelsius}$ in dem Bereich von 1000-4000 liegt, sind nur bei diesen die Beobachtung des Franck-Hertz-Effektes zu erwarten \cite{V601}.


\subsection{Bestimmung der Umrechnungsfaktoren}

\begin{center}
	\begin{table}
		\begin{minipage}[t]{0.5\textwidth}
			\caption{Erste}
			\centering
			\input{build/a1Abstaende.tex}
		\end{minipage}
		\begin{minipage}[t]{0.5\textwidth}
			\caption{Zweite}
			\centering
			\input{build/a2Abstaende.tex}
		\end{minipage}
	\end{table}
\end{center}
\begin{center}
	\begin{table}
		\begin{minipage}[t]{0.5\textwidth}
			\caption{Dritte}
			\centering
			\input{build/bneuAbstaende.tex}
		\end{minipage}
		\begin{minipage}[t]{0.5\textwidth}
			\caption{Vierte}
			\centering
			\input{build/cAbstaende.tex}
		\end{minipage}
	\end{table}
\end{center}
Mit den gemessenen Abständen zwischen den Markierungen in den Tabellen \ref{tab:a1Abstaende}, \ref{tab:a2Abstaende}, \ref{tab:bneuAbstaende} und \ref{tab:cAbstaende} lassen sich die Umrechnungsfaktoren $f$ bestimmen.
Es ergibt sich für die mittleren Spannungen pro Zentimeter entlang der x-Achse der Graphen \ref{label}:
\begin{gather*}
	f_\text{a1}=\SI{0.44(1)}{\volt\per\centi\meter}\\
	f_\text{a2}=\SI{0.46(1)}{\volt\per\centi\meter}\\
	f_\text{b}=\SI{2.25(10)}{\volt\per\centi\meter}\\
	f_\text{c}=\SI{1.18(7)}{\volt\per\centi\meter}\text{.}
\end{gather*}


\subsection{a}
\begin{figure}
	\centering
	\caption{Die Filterkurve des Selektiv-Verstärkers.}
	\includegraphics[width=\linewidth-70pt,height=\textheight-70pt,keepaspectratio]{build/a1neu.pdf}
	\label{fig:GraphSelektiv}
\end{figure}
\begin{figure}
	\centering
	\caption{Die Filterkurve des Selektiv-Verstärkers.}
	\includegraphics[width=\linewidth-70pt,height=\textheight-70pt,keepaspectratio]{build/a2neu.pdf}
	\label{fig:GraphSelektiv}
\end{figure}
\begin{table}
	\caption{Fünfte}
	\centering
	\input{build/a1Grad.tex}
\end{table}
\begin{table}
	\caption{Sechste}
	\centering
	\input{build/a2Grad.tex}
\end{table}
Die in den Abbildungen \ref{zzz} und \ref{label} dargestellten Graphen FEHLENDER TEXT
Das Peak in dem Graphen bei $T=\SI{25}{\degreeCelsius}$ würde ohne Kontaktpotential bei der Beschleunigungsspannung $U_\text{B}=\SI{8.5}{\volt}$ erwartet werden. Jedoch ist das Peak offenbar um das Kontaktpotential verschoben.
Es kann abgelesen werden
\begin{displaymath}
K_\text{a}=\SI{1.77(18)}{\volt}\text{.}
\end{displaymath}
\subsection{b}
\begin{table}
	\caption{Siebte}
	\centering
	\input{build/bneuDiff.tex}
\end{table}
Aus den mittleren Abstand zwischen den Peaks in Tabelle \ref{tab:bneuDiff} und dem zuvor bestimmten Umrechnungsfaktor $f_\text{b}$ berechnet sich die 1. Anregungsenergie zu
\begin{displaymath}
	E_1=\SI{4.82(9)}{\electronvolt}\text{.}
\end{displaymath}
Mit dieser lässt sich die Wellenlänge $\lambda$ der emittierten Strahlung nach ??? berechnen.
Es ergibt sich
\begin{displaymath}
\lambda=\SI{257(5)}{\nano\meter}\text{.}
\end{displaymath}


\subsection{c}
\begin{figure}
	\centering
	\caption{HI.}
	\includegraphics[width=\linewidth-70pt,height=\textheight-70pt,keepaspectratio]{build/c.pdf}
	\label{fig:GraphSelektiv}
\end{figure}
\begin{table}
	\caption{Achte}
	\centering
	\input{build/cKoordinaten.tex}
\end{table}
Die lineare Fit in der Abbildungen \ref{fig:GraphultraV2} besitzt die Form $y=a x + b$. Eine lineare Ausgleichsrechnung der Form $y=a x+b$ liefert mit den Wertepaaren aus Tabelle \ref{tab:cKoordinaten} für die Summe von der Ionisierungsspannung $U_\text{ion}$ und dem Kontaktpotential $K$
\begin{displaymath}
	U_\text{ion}+K=\SI{12.2(5)}{\volt}\text{.}
\end{displaymath}
Daraus lässt sich nun mit dem zuvor entnommenen Kontaktpotential $K_\text{a}=\SI{1.77(18)}{\volt}$ die Ionisierungsspannung $U_\text{ion}$ berechnen. Es folgt
\begin{displaymath}
U_\text{ion}=\SI{10.5(6)}{\volt}\text{.}
\end{displaymath}

\begin{table}
	\caption{Neunte}
	\centering
	\input{build/Ergebnisse.tex}
\end{table}


