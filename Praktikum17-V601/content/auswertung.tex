\section{Auswertung}
\label{sec:Auswertung}


Die Graphen wurden sowohl mit Matplotlib \cite{matplotlib} als auch NumPy \cite{numpy} erstellt. Die
Fehlerrechnung wurde mithilfe von Uncertainties \cite{uncertainties} durchgeführt.
Die Konstanten $k$, $\hbar$, $e_0$, $m_0$, $u_0$ und $N_\text{A}$ sind vom NIST \cite{nistgov}.

\subsection{Berechnung der mittleren freien Weglänge und Vergleich mit dem Abstand zwischen der BESCH und GITTER}

\subsection{Berechnung des Skalierungsfaktor}

\begin{center}
	\begin{table}
		\begin{minipage}[t]{0.5\textwidth}
			\caption{Erste}
			\centering
			\input{build/a1Abstaende.tex}
		\end{minipage}
		\begin{minipage}[t]{0.5\textwidth}
			\caption{Zweite}
			\centering
			\input{build/a2Abstaende.tex}
		\end{minipage}
	\end{table}
\end{center}
\begin{center}
	\begin{table}
		\begin{minipage}[t]{0.5\textwidth}
			\caption{Dritte}
			\centering
			\input{build/bneuAbstaende.tex}
		\end{minipage}
		\begin{minipage}[t]{0.5\textwidth}
			\caption{Vierte}
			\centering
			\input{build/cAbstaende.tex}
		\end{minipage}
	\end{table}
\end{center}

\subsection{a}
\begin{figure}
	\centering
	\caption{Die Filterkurve des Selektiv-Verstärkers.}
	\includegraphics[width=\linewidth-70pt,height=\textheight-70pt,keepaspectratio]{build/a1neu.pdf}
	\label{fig:GraphSelektiv}
\end{figure}
\begin{figure}
	\centering
	\caption{Die Filterkurve des Selektiv-Verstärkers.}
	\includegraphics[width=\linewidth-70pt,height=\textheight-70pt,keepaspectratio]{build/a2neu.pdf}
	\label{fig:GraphSelektiv}
\end{figure}
\begin{table}
	\caption{Fünfte}
	\centering
	\input{build/a1Grad.tex}
\end{table}
\begin{table}
	\caption{Sechste}
	\centering
	\input{build/a2Grad.tex}
\end{table}

\subsection{b}
\begin{table}
	\caption{Siebte}
	\centering
	\input{build/bneuDiff.tex}
\end{table}


\subsection{c}
\begin{table}
	\caption{Siebte}
	\centering
	\input{build/cKoordinaten.tex}
\end{table}

