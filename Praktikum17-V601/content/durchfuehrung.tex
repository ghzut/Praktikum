
\section{Durchführung}
\label{sec:Durchführung}
%teil a)
Zunächst werden Kurven zur Bestimmung der integralen Energieverteilung der
beschleunigten Elektronen erstellt, einmal bei Zimmertemperatur und einmal bei
ca. $140- \SI{160}{\degreeCelsius}$. Hierzu wird die Beschleunigungsspannung auf
$\SI{11}{\volt}$ eingestellt. Die Bremsspannung wird für den Kurvenverlauf
zwischen $0$ und $\SI{11}{\volt}$ variiert. Für die Hochtemperatur muss zusätzlich
darauf geachtet, dass die Temperatur möglichst konstant bleibt. Dazu muss die
Heizleistung gegebenenfalls nachjustiert werden, da die Temperatur nicht direkt
eingestellt werden kann. Die ist bei allen nachfolgenden Kurven zu beachten. Der
XY-Schreiber ist vorher ohne Anschluss einer Spannung auf seinen Nullpunkt zu justieren.
%teil b)
Anschließend folgt eine Messung der Ionisierungsspannung von Hg. Hierzu wird der
Quecksilberdampf auf eine Temparatur von ca. $110- \SI{120}{\degreeCelsius}$ gebracht.
Die Bremsspannung wird auf $\SI{-30}{\volt}$ eingestellt, sodass möglichst alle
verfügbaren Elektronen auf die Auffängeranode treffen. Die Kurve des Auffängerstroms
$I_\text{A}$ folgt über die Variation von $U_\text{B}$. Der interessante Bereich endet
nach dem ersten Maximum. Da das Graphenbild im negativen Bereich verlaufen würde sind die
Kontakte am Y-Eingang zu vertauschen.

% teil c)
Zuletzt werden Frank-Hertz Kurven im Temperaturbereich zwischen $160-\SI{200}{\degreeCelsius}$
angefertigt. Dazu wird $U_\text{A}$ auf $\SI{1}{\volt}$ eingestellt und $U_\text{B}$
im Bereich von $0-\SI{60}{\volt}$ variiert. Aufgrund der in Theorie dargestellten
Fehlerquellen werden mehrere Kurven unter verschiedenen Temperaturen erstellt und
diejenige mit den ausgeprägtesten Extrema für die Auswertung verwendet.
