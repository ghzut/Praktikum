
\section{Auswertung}
\label{sec:Auswertung}


\subsection{Bestimmung des Elastizitätsmoduls über die Durchbiegung eines runden, einseitig eingespannten Stabes}
\begin{figure}
	\centering
	\caption{Die gemessene Durchbiegung $D(x)$ des runden, einseitig eingespannten Stabes
	 in Abhängigkeit des horizontalen Abstandes zum fixierten Ende.}
	\includegraphics[width=\linewidth-70pt,height=\textheight-70pt,keepaspectratio]{build/rundstab.pdf}
	\label{fig:rund}
\end{figure}
\begin{figure}
	\centering
	\caption{Die gemessene Durchbiegung $D(x)$ des runden, einseitig eingespannten Stabes
	 in Abhängigkeit des horizontalen Abstandes zum fixierten Ende in linearisierter Darstellung.}
	\includegraphics[width=\linewidth-70pt,height=\textheight-70pt,keepaspectratio]{build/rundstab2.pdf}
	\label{fig:rundlinear}
\end{figure}
\begin{table}
	\caption{Die gemessene Durchbiegung $D(x)$ des einseitig eingespannten, runden Stabes an den jeweiligen horizontalen Abständen $x$ zum fixierten Ende.}
	\begin{minipage}{0.5\textwidth}
		\centering
		\input{build/tabeinseitigrund1.tex}
	\end{minipage}
	\begin{minipage}{0.5\textwidth}
		\centering
		\input{build/tabeinseitigrund2.tex}
	\end{minipage}
\end{table}
Der Fit in Abbildung \ref{fig:rund} besitzt die Form $y = a ( L x^2 - \frac{x^3}{3} ) $ und in Abbildung \ref{fig:rundlinear} die Form $y=a x$. Aus Formel \eqref{eq:Flächenträgheitsmoment} ergibt sich das Flächenträgheitsmoment $I$ zu $\SI{4.91e-2}{\centi\meter\tothe{4}}$. Eine nicht lineare Ausgleichsrechnung der Form $y = a ( L x^2 - \frac{x^3}{3})$ liefert mit der gemessenen Länge $L$ von $\SI{52}{\centi\meter}$, dem Flächenträgheitsmoment $I$, der Kraft $F$ und den Wertepaaren aus Tabelle \ref{tab:tabeinseitigrund1} mit Formel \eqref{eq:EinseitigEingespannt}:
\begin{displaymath}
	E = \frac{F}{2 a I}= \SI{91.0(4)}{\giga\pascal}\text{.}
\end{displaymath}
Es ist zu erkennen, dass die Messwerte durch die Fits angenähert werden können.



\subsection{Bestimmung des Elastizitätsmoduls über die Durchbiegung eines quadratischen, einseitig eingespannten Stabes}

\begin{figure}
	\centering
	\caption{Die gemessene Durchbiegung $D(x)$ des quadratischen, einseitig eingespannten Stabes
	 in Abhängigkeit des horizontalen Abstandes zum fixierten Ende.}
	\includegraphics[width=\linewidth-70pt,height=\textheight-70pt,keepaspectratio]{build/quadratstabeinseitig.pdf}
	\label{fig:quadratisch}
\end{figure}
\begin{figure}
	\centering
	\caption{Die gemessene Durchbiegung $D(x)$ des quadratischen, einseitig eingespannten Stabes
	 in Abhängigkeit des horizontalen Abstandes zum fixierten Ende in linearisierter Darstellung.}
	\includegraphics[width=\linewidth-70pt,height=\textheight-70pt,keepaspectratio]{build/quadratstabeinseitig2.pdf}
	\label{fig:quadratischlinear}
\end{figure}
\begin{table}
	\caption{Die gemessene Durchbiegung $D(x)$ des einseitig eingespannten, quadratischen Stabes an den jeweiligen horizontalen Abständen $x$ zum fixierten Ende.}
	\begin{minipage}{0.5\textwidth}
		\centering
		\input{build/tabeinseitigeckig1.tex}
	\end{minipage}
	\begin{minipage}{0.5\textwidth}
		\centering
		\input{build/tabeinseitigeckig2.tex}
	\end{minipage}
\end{table}
Der Fit in Abbildung \ref{fig:quadratisch} besitzt die Form $y = a ( L x^2 - \frac{x^3}{3} ) $ und in Abbildung \ref{fig:quadratischlinear} die Form $y=a x$. Aus Formel \eqref{eq:Flächenträgheitsmoment} ergibt sich das Flächenträgheitsmoment $I$ zu $\SI{8.33e-2}{\centi\meter\tothe{4}}$. Eine nicht lineare Ausgleichsrechnung der Form $y = a ( L x^2 - \frac{x^3}{3})$ liefert mit der gemessenen Länge $L$ von $\SI{50}{\centi\meter}$, dem Flächenträgheitsmoment $I$, der Kraft $F$ und den Wertepaaren aus Tabelle \ref{tab:tabeinseitigeckig1} mit Formel \eqref{eq:EinseitigEingespannt}:
\begin{displaymath}
E = \frac{F}{2 a I}= \SI{87.38(26)}{\giga\pascal}\text{.}
\end{displaymath}
Es ist zu erkennen, dass die Messwerte durch die Fits angenähert werden können.



\subsection{Bestimmung des Elastizitätsmoduls über die Durchbiegung eines quadratischen, beidseitig aufgelegten Stabes}

\begin{figure}
	\centering
	\caption{Die gemessene Durchbiegung $D(x)$ des quadratischen, beidseitig aufliegenden Stabes
	 in Abhängigkeit des horizontalen Abstandes zum rechten Auflagepunkt.}
	\includegraphics[width=\linewidth-70pt,height=\textheight-70pt,keepaspectratio]{build/quadratstabbeidseitig.pdf}
	\label{fig:beidseitig}
\end{figure}
\begin{figure}
	\centering
	\caption{Die gemessene Durchbiegung $D(x)$ des quadratischen, beidseitig aufliegenden Stabes von der rechten Seite bis zur Stabmitte
	 in Abhängigkeit des horizontalen Abstandes zum rechten Auflagepunkt in linearisierter Form.}
	\includegraphics[width=\linewidth-70pt,height=\textheight-70pt,keepaspectratio]{build/quadratstabbeidseitig2.pdf}
	\label{fig:beidseitiglinear1}
\end{figure}
\begin{figure}
	\centering
	\caption{Die gemessene Durchbiegung $D(x)$ des quadratischen, beidseitig aufliegenden Stabes von der linken Seite bis zur Stabmitte
	 in Abhängigkeit des horizontalen Abstandes zum rechten Auflagepunkt in linearisierter Form.}
	\includegraphics[width=\linewidth-70pt,height=\textheight-70pt,keepaspectratio]{build/quadratstabbeidseitig3.pdf}
	\label{fig:beidseitiglinear2}
\end{figure}
\begin{table}
	\caption{Die gemessene Durchbiegung $D(x)$ des beidseitig aufliegenden, quadratischen Stabes an den jeweiligen horizontalen Abständen $x$ zum rechten Auflagepunkt.}
	\begin{minipage}[c]{0.5\textwidth}
		\centering
		\input{build/tabbeidseitig1.tex}
	\end{minipage}
	\begin{minipage}[c]{0.5\textwidth}
		\centering
		\input{build/tabbeidseitig2.tex}
	\end{minipage}
\end{table}
Der Fit in Abbildung \ref{fig:beidseitig} besitzt die Form:
\begin{equation}
	y = 
	\begin{cases}
	3L^2 x-4x^3& 0\leq x \leq \frac{L}{2} \\ 
	4 x^3 -12 L x^2 + 9 L^2 x -L 3  & \frac{L}{2} < x < L 
	\end{cases}\text{.} \label{FunktionBeidseitig}
\end{equation}
In den Abbildungen \ref{fig:beidseitiglinear1} und \ref{fig:beidseitiglinear2} die Form $y=a x$. Aus Formel \eqref{eq:Flächenträgheitsmoment} ergibt sich das Flächenträgheitsmoment $I$ zu $\SI{8.33e-2}{\centi\meter\tothe{4}}$. Eine nicht lineare Ausgleichsrechnung der Form \eqref{FunktionBeidseitig} liefert mit der gemessenen Länge $L$ von $\SI{55.5}{\centi\meter}$, dem Flächenträgheitsmoment $I$, der Kraft $F$ und den Wertepaaren aus Tabelle \ref{tab:tabbeidseitig1} mit Formel \eqref{eq:BeidseitigAufgelegtLinks} und \eqref{eq:BeidseitigAufgelegtRechts}:
\begin{displaymath}
E = \frac{F}{48 a I}= \SI{91.63(25)}{\giga\pascal}\text{.}
\end{displaymath}
Es ist zu erkennen, dass die Messwerte durch die Fits angenähert werden können.


