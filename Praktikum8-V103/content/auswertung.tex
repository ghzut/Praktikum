
\section{Auswertung}
\label{sec:Auswertung}


\subsection{Bestimmung des Elastizitätsmoduls über die Durchbiegung eines runden, einseitig eingespannten Stabes}
\begin{figure}
	\centering
	\caption{Die gemessene Durchbiegung $D(x)$ des runden, einseitig eingespannten Stabes
	 in Abhängigkeit des horizontalen Abstandes zum fixierten Ende.}
	\includegraphics[width=\linewidth-70pt,height=\textheight-70pt,keepaspectratio]{build/rundstab.pdf}
	\label{fig:grad3}
\end{figure}
\begin{figure}
	\centering
	\caption{Die gemessene Durchbiegung $D(x)$ des runden, einseitig eingespannten Stabes
	 in Abhängigkeit des horizontalen Abstandes zum fixierten Ende in linearisierter Darstellung.}
	\includegraphics[width=\linewidth-70pt,height=\textheight-70pt,keepaspectratio]{build/rundstab2.pdf}
	\label{fig:grad3}
\end{figure}
\input{build/tabeinseitigrund.tex}


\subsection{Bestimmung des Elastizitätsmoduls über die Durchbiegung eines quadratischen, einseitig eingespannten Stabes}
\begin{figure}
	\centering
	\caption{Die gemessene Durchbiegung $D(x)$ des quadratischen, einseitig eingespannten Stabes
	 in Abhängigkeit des horizontalen Abstandes zum fixierten Ende.}
	\includegraphics[width=\linewidth-70pt,height=\textheight-70pt,keepaspectratio]{build/quadratstabeinseitig.pdf}
	\label{fig:grad3}
\end{figure}
\begin{figure}
	\centering
	\caption{Die gemessene Durchbiegung $D(x)$ des quadratischen, einseitig eingespannten Stabes
	 in Abhängigkeit des horizontalen Abstandes zum fixierten Ende in linearisierter Darstellung.
	\includegraphics[width=\linewidth-70pt,height=\textheight-70pt,keepaspectratio]{build/quadratstabeinseitig2.pdf}
	\label{fig:grad3}
\end{figure}
\input{build/tabeinseitigeckig.tex}


\subsection{Bestimmung des Elastizitätsmoduls über die Durchbiegung eines quadratischen, beidseitig aufgelegten Stabes}
\begin{figure}
	\centering
	\caption{Die gemessene Durchbiegung $D(x)$ des quadratischen, beidseitig aufliegenden Stabes
	 in Abhängigkeit des horizontalen Abstandes zum rechten Auflagepunkt.}
	\includegraphics[width=\linewidth-70pt,height=\textheight-70pt,keepaspectratio]{build/quadratstabbeidseitig.pdf}
	\label{fig:grad3}
\end{figure}
\begin{figure}
	\centering
	\caption{Die gemessene Durchbiegung $D(x)$ des quadratischen, beidseitig aufliegenden Stabes bis zur Stabmitte
	 in Abhängigkeit des horizontalen Abstandes zum rechten Auflagepunkt in linearisierter Form.}
	\includegraphics[width=\linewidth-70pt,height=\textheight-70pt,keepaspectratio]{build/quadratstabbeidseitig2.pdf}
	\label{fig:grad3}
\end{figure}
\begin{figure}
	\centering
	\caption{Die gemessene Durchbiegung $D(x)$ des quadratischen, beidseitig aufliegenden Stabes bis zur Stabmitte
	 in Abhängigkeit des horizontalen Abstandes zum linken Auflagepunkt in linearisierter Form}
	\includegraphics[width=\linewidth-70pt,height=\textheight-70pt,keepaspectratio]{build/quadratstabbeidseitig3.pdf}
	\label{fig:grad3}
\end{figure}
Hallo
\input{build/tabbeidseitig.tex}
Hallo