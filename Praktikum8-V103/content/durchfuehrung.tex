
\section{Durchführung}
\label{sec:Durchführung}

\renewcommand{\labelenumi}{\alph{enumi})}
\begin{enumerate}
  \item Es wird die Durchbiegung eines einseitig eingespannten, runden Stabes in
   Abhängigkeit der Länge gemessen. Hierzu wird der Stab an einer Seite eingespannt.
   Da nicht auzuschließen ist, dass der Stab bereits gebogen ist, wird zunächst
    eine Auslenkungsmessung ohne angehängte Masse durchgeführt. Es ist bei der
     Messung darauf zu achten, dass das Rädchen der Messuhr auf der Mitte des Stabes aufliegt.
     Der Abstand zwischen der einzelnen Messstellen beträgt
     $\SI{1}{\centi\meter}$. Anschließend wird eine Masse angehängt, sodass
      die maximale Durchbiegung ca. $5-\SI{7}{\milli\meter}$ beträgt. Die Auslenkungsmessung
       wird nun mit denselben messtellen wiederholt. Die Durchbiegung $D(x)$
       berechnet sich über die Differenz beider Werte.

       \item Das oben beschriebene Verfahren wird nochmals mit einem
        rechteckigen Stab durchgeführt.

        \item Nun wird die Durchbiegung eines Stabes untersucht, welcher an beiden Enden frei aufliegt.
         Die Normalspannung soll nun in der Stabmitte wirken und wird durch eine
          Masse realisiert, welche in dieser befestigt wird.
          Da sich die Messuhr bei einer in der Mitte angehängten Masse nicht
           über die gesamte Stablänge fahren lässt, kommen zwei Messuhren zum Einsatz.
           Auch wird zunächst die Auslenkung ohne Last gemessen.
            Um systematische Messfehler zu vermeiden werden bereits hier
             beide Uhren für die Linke bzw. Rechte Hälfte verwendet. Im Anschluss
              wird wieder eine Masse angehängt und nochmals gemessen. Auch hier
               berechnet sich $D(x)$ über die Differenz.

\end{enumerate}
