
\section{Durchführung}
\label{sec:Durchführung}

\renewcommand{\labelenumi}{\alph{enumi})}
\begin{enumerate}
  \item Es wird die Durchbiegung eines einseitig eingespannten, runden Stabes in
   Abhängigkeit der Länge gemessen. Hierzu wird der Stab an einer Seite eingespannt.
   Da nicht auzuschließen ist, dass der Stab bereits gebogen ist, wird zunächst
    eine Messung ohne angehängtes Gewicht ausgeführt. Die Abstände der Messpunkte betragen
     $\SI{1}{\centi\meter}$. Anschließend wird ein Gewicht angehängt, sodass
      die Durchbiegung am Ende ca. $5-\SI{7}{\mili\meter}$ beträgt. Anschließend
       wird an den voherigen Stellen nochmals gemessen. Die Durchbiegung $D(x)$
       berechnet sich über die Differenz beider Werte.

       \item Das oben beschriebene Verfahren wird nochmals mit einem
        rechteckigen Stab durchgeführt.

        \item Nun wird die Durchbiegung eines Stabes untersucht, welcher an beiden Enden aufliegt.
        
\end{enumerate}
