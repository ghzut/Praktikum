
\section{Theorie}
\label{sec:Theorie}

\subsection{Das Verhalten von deformierbaren Körpern unter einer Spannung}
Kräfte, die auf die Oberfläche eines Körpers wirken, können dessen Gestalt oder
 Volumen verändern. Eine solche Kraft wird als Spannung bezeichnet und normalerweise
  in der Form $\frac{\text{Kraft}}{\text{Fläche}}$ angegeben. Ihre oberflächenparallele
   Komponente wird als Tangentialspannung $\sigma$ bezeichnet, ihre senkrechte als Normalspannung $\tau$.
Sind die aufgrund der Spannung auftretenden relativen Längenänderungen hinreichend
 klein, liegt ein linearer Zusammenhang zwischen wirkender Spannung und Längenänderung vor.
    %Sind der Gestaltsänderung klein, so l ein
     %linearer Zusammenhang zwischen Längenänderung und wirkender Spannung vor.
      Dieser wird Hooksches Gesetz genannt.
       Allgemein lässt sich das Verhalten eines Stoffes unter einer Spannung durch eine
        $6\times6$ Matrix ausdrücken.
      Handelt es sich beim Material des Körpers um einen isotropen Stoff,
       reichen zwei Materialkonstante aus um dieses vollständig
        zu beschreiben, aus praktischen Gründen werden zwei weitere eingeführt.
         Eine davon ist der Elastizitätsmodul $E$, welcher die
         relative Längenänderung eines Körpers in Spannungsrichtung unter
          einer Normalspannung beschreibt.

\subsection{Der einseitig eingespannte homogene Stab unter Einfluss einer konstanten Normalspannung}
Wird ein homogener Stab einseitig eingespannt und anschließend am freien Ende unter den Einfluss
 einer konstanten Normalspannung gestellt, kommt es zu einer Biegung des Stabes. Diese
  reicht soweit bis ein Gleichgewicht zwischen der angreifenden Kraft und einer Stabinneren Gegenkraft besteht. Die
   innere Kraft entsteht aufgrund von Streckungen und Stauchungen der einzelnen
    Materialschichten, welche durch die Biegung verursacht werden. Da es sich bei
     der Kraftwirkung um eine Drehung handelt, kann die verursachte Durchbiegung
      $D(x)$ an der Stelle $x$ mithilfe eines Drehmomentansatz bestimmt werden.
       Ist die Durchbiegung klein im Verhältnis zur Stablänge folgt:


 % Diese führt zu einer
  %Durchbiegung $D(x)$ im Vergleich zum unbelasteten Stab. Während die oberen
   %Schichten des Stabes bei der Biegung gedehnt werden, werden die Unteren gestaucht.
   %In ihrer Mitte existiert eine Schicht welche keine Längenänderung erfährt.
   %Diese heißt neutrale Faser. Ist die auftretende Krümmung des Stabes
    %hinreichend klein folgt über einen Drehmomentansatz:
   \begin{equation}
     D(x) = \frac{F}{2EI}\left(Lx² - \frac{x³}{3}\right) \text{ mit }0 \leq x \leq L \label{eq:EinseitigEingespannt}
     \end{equation}
     mit der Spannung $F$ und der Stablänge $L$. Die Variable $I$ bezeichnet das
     Flächenträgheitsmoment des Stabes:
\begin{equation}
  I = \int_{Q}^{}y^2 \, \symup{d}q \label{eq:Flächenträgheitsmoment}
\end{equation}
Hierzu wird über die Querschnittsfläche $Q$ des Stabes integriert.
In der Versuchsdurchführung wird jeweils ein Stab mit qudratischer
 und einer mit kreisförmiger Querschnittsfläche verwendet.
 Diese lassen sich explizit durch folgende Formeln berechnen.
 \begin{equation}
   I_\text{Kreis} = \int_{0}^{2\pi}\text{d}\varphi \int_{0}^{R} ry^2\text{d}r \stackrel{y = rsin(\varphi)}{=}  \int_{0}^{2\pi}\text{d}\varphi \int_{0}^{R} r^3 \sin²(\varphi)\text{d}r = \frac{\pi R⁴}{4} \label{eq:Ir}
   \end{equation}
\begin{equation}
  I_\text{Quadrat} = a \int_{-\frac{a}{2}}^{\frac{a}{2}}y²\text{d}y = a \left[\frac{1}{3}y^3\right]_{-\frac{a}{2}}^{\frac{a}{2}} = \frac{a⁴}{12} \label{eq:Iq}
\end{equation}
\subsection{Der beidseitig aufgelegte Stab unter Einfluss einer konstanten Normalspannung}
Der Stab liegt nun an beiden Enden auf einem Träger auf. Wirkt nun eine wieder
 konstante Kraft auf den Mittelpunkt des Stabes, kommt es auch in diesem Fall
  zu einer Durchbiegung.
 Auch hier gelangt man über das Gleichgewicht der auftretenden Drehmomente an die Durchbiegung des Stabes.
 Das rechte Ende wird mit $ x = 0$, das linke Ende mit $x= L$ ausgedrückt.
    Dann folgt für die Durchbiegung der linken Hälfte $D_\text{L}$:
  \begin{equation}
    D_\text{R}(x) = \frac{F}{48 EI}\left(3L²x-4x³\right) \text{ mit } 0 \leq x \leq \frac{L}{2}\text{.}\label{eq:BeidseitigAufgelegtRechts}
  \end{equation}
Für die rechte Hälfte folgt:
\begin{equation}
  D_\text{L}(x) = \frac{F}{48EI}\left(4x³ -12Lx² +9L²x-L³\right) \text{ mit } \frac{L}{2} \leq x \leq L\text{.}\label{eq:BeidseitigAufgelegtLinks}
\end{equation}
