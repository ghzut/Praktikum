
\section{Theorie}
\label{sec:Theorie}

\subsection{}
Kräfte, die auf die Oberfläche eines Körpers wirken, können dessen Gestalt oder
 Volumen verändern. Solche Kräfte werden als Spannung bezeichnet und normalerweise
  in form von $\frac{\text{Kraft}}{\text{Fläche}}$ angegeben. Ihre oberflächenparallele
   Komponente wird als Tangentialspannung $\sigma$ bezeichnet, ihre senkrechte Normalspannung $\tau$.
    Sind die relativen Längenänderung der Gestaltsänderung klein, so liegt ein
     linearer Zusammenhang zwischen Längenänderung und wirkender Spannung vor.
      Dieser wird Hooksches Gesetz genannt.

      Handelt es sich beim Material des Körpers um einen isotropen Stoff,
       reichen zwei Materialkonstanten aus um sein Verhalten unter einer Spannung
        zu beschreiben, aus praktischen Gründen werden zwei weitere eingeführt.
         Eine davon ist der Elastizitätsmodul $E$, welcher die
         relative Längenänderung eines Körpers in Spannungsrichtung unter
          einer Normalspannung beschreibt.

\subsection{Der einseitig eingespannte homogene Stab unter Einfluss einer konstanten Kraft}
Wird ein homogener Stab einseitig eingespannt und anschließend unter den Einfluss
 einer Kraft gestellt, kommt es zu einer Biegung des Stabes. Diese führt zu einer
  Auslenkung $D(x)$ im Vergleich zum unbelasteten Stab. Während die oberen
   Schichten des Stabes bei der Biegung gedehnt werden, werden die Unteren gestaucht.
   In ihrer Mitte existiert eine Schicht welche keine Längenänderung erfährt.
   Diese heißt neutrale Faser. Ist die auftretende Krümmung des Stabes
    hinreichend klein folgt über einen Drehmomentansatz:
   \begin{equation}
     D(x) = \frac{F}{2EI}\left(Lx² - \frac{x³}{3}\right) \text{ mit }0 \leq x \leq L
     \end{equation}
     mit der Kraft $F$ und der Stablänge $L$. Die Variable $I$ bezeichnet das
     Flächenträgheitsmoment:
\begin{equation}
  I = \int_{Q}^{}y^2 \, \symup{d}q
\end{equation}
Hierzu wird über die Querschnittsfläche $Q$ des Stabes integriert.

\subsection{Der beidseitig auflegte Stab unter Einfluss einer konstanten Kraft}
Auch bei beidseitiger Auflage des Stabes kommt es zu einer Durchbiegung, wenn
 der Stab einer konstanten Kraft unterliegt. Da beide Enden frei sind, kommt
 Auch hier gelangt man über einen Ansatz der wirkenden Drehmomente an die
  auftretende Auslenkung. Wirkt die Kraft auf die Mitte des Stabes folgt für die
   Auslenkung der linken Hälfte $D_\text{L}$:
  \begin{equation}
    D_\text{L}(x) = \frac{F}{48 EI}\left(3L²x-4x³)\right) \text{ mit } 0 \leq x \leq \frac{L}{2}\text{.}
  \end{equation}
Für die rechte Hälfte folgt:
\begin{equation}
  D_\text{R}(x) = \frac{F}{48EI}\left(4x³ -12Lx² +9L²x-L³\right) \text{ mit } \frac{L}{2} \leq x \leq L\text{.}
\end{equation}
