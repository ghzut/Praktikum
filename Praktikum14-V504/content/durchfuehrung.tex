
\section{Durchführung}
\label{sec:Durchführung}
Zuerst werden Durchmesser und Masse von zwei unterschiedlich großen Glaskugeln
gemessen um daraus die jeweiligen Dichten zu ermitteln. Im Anschluss wird das
Viskosimeter mithilfe der Libelle ausgerichtet, sodass es gerade steht und die Fallröhre anschließend
mit destilliertem Wasser aufgefüllt. Um eine Verfälschung der Messzeiten durch
Luftblasen zu vermeiden, werden diese vorher mit einem Glasstab entfernt.
Anschließend wird zunächst die kleine Kugel unter Zimmertemperatur in die
Fallröhre eingesetzt und letztere wasserdicht verschlossen. Danach wird die Fallzeit
gemessen, welche die Kugel von der oberen bis zur unteren Markierung benötigt.
Dieser Vorgang wird zehn mal wiederholt, zunächst bei der kleineren, danach noch einmal
mit der größeren. Im Folgenden wird das Wasser mithilfe des Wasserbades erwärmt. Es
werden für die große Glaskugel zusätzlich jeweils zwei Fallzeiten zu zehn
verschiedenen Temperaturen unter $\SI{70}{\degreeCelsius}$ notiert.
