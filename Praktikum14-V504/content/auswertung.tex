\section{Auswertung}
\label{sec:Auswertung}

Die Graphen wurden sowohl mit Matplotlib \cite{matplotlib} als auch NumPy \cite{numpy} erstellt. Die
 Fehlerrechnung wurde mithilfe von Uncertainties \cite{uncertainties} durchgeführt.


 \subsection{Bestimmung des Sättigungsstroms einer Hochvakuum-Diode mithilfe von Kennlinien}
Zunächst werden die Sättigungsströme als Grenzwerte der Kennlinien bestimmt. Den Graphen nach werden die
Sättigungsströme zu den drei untersten Graphen durch ihre letzten Messwerte hinreichend
gut beschrieben. Die Sättigungsströme zu den Heizströmen $I_\text{f} = \SI{2.3}{\ampere}$ und $I_\text{f} = \SI{2.4}{\ampere}$
sind Schätzungen, basierend auf den bekannten Graphenverläufen. Aus der Kennlinie
zu $I_f = \SI{2.5}{\ampere}$ lässt sich kein Sättigungsstrom zu folgern.

 \begin{table}
 	\centering
 	\caption{Die gemessenen Stromstärken in Abhängigkeit der Saugspannung unter Heizströmen zwischen $\SI{2,0}{\ampere}$ und $\SI{2,4}{\ampere}$ .}
 	\input{build/tabV1wo.tex}
 \end{table}

 \begin{table}
  \centering
  \caption{Die gemessenen Stromstärken in Abhängigkeit der Saugspannung bei einem Heizstrom von $\SI{2,5}{\ampere}$.}
  \input{build/tabV1ol1.tex}
  \input{build/tabV1ol2.tex}
 \end{table}

 \begin{figure}
 	\centering
 	\caption{Die Kennlinien der Hochvakuumdiode unter verschiedenen Heizleistungen.}
 	\includegraphics[width=\linewidth-70pt,height=\textheight-70pt,keepaspectratio]{build/Kennlinien.pdf}
 	\label{fig:Graph}
 \end{figure}

 \begin{table}
  \centering
  \caption{Die geschätzten Sättigungsströme unter Variation der Heizleistung.}
  \input{build/IS.tex}
 \end{table}




\subsection{Bestimmung des Exponenten der Strom-Spannungs-Beziehung im Geltungsbereich des Langmuir-Schottkyschen Gesetzes}
\begin{figure}
 \centering
 \caption{Die doppellogarithmische Darstellung Kennlinie des mit $I_f = \SI{2.5}{\ampere}$.}
 \includegraphics[width=\linewidth-70pt,height=\textheight-70pt,keepaspectratio]{build/Kennlinielog.pdf}
 \label{fig:Graphlog2}
\end{figure}
Mithilfe einer doppellogarithmischen Darstellung der Kennlinie bei einem Heizstrom von $\SI{2.5}{\ampere}$
 wird die Strom-Spannungsbeziehung im Geltungsbereich des Langmuir-Schottkyschen Gesetzes untersucht.
 Es wird nach Formel \ref{eq:Langmuir-Schottkysche_Raumladungsgleichung} ein linearer Zusammenhang vermutet.
Daher werden die Messwerte mit einer Saugspannung zwischen $\SI{70}{\volt}$
und $\SI{180}{\volt}$ verwendet. Ein linearer Fit der Form $y = a x+b$ liefert
eine Steigung von $a = \num{1.576(8)}$.
%wie macht man das mit dem plusminus richtig























\subsection{Bestimmung der Kathodentemperatur über das Anlaufstromgebiet}
Es folgt eine Untersuchung des Anlaufstromgebietes mithilfe der Messwerte aus \ref{tab:tabV2}.
Da die Temperatur nach Formel \ref{eq:Anlaufstromstärke} logarithmisch bezüglich
der Stromstärke skaliert, wird eine halblogarithmische Darstellung in
\ref{fig:Graphlog3} verwendet. Mithilfe einer linearen Ausgleichsrechnung der Form
$y = a x+b$ und der Formel
\begin{equation}
  T = \frac{-e_0 V}{k a}
  \end{equation}
  folgt eine Temperatur von $\SI{3960(170)}{\kelvin}$. Da in \ref{fig:Graphlog3}
  keine eindeutige Gerade zu erkennen ist,
werden alle Messwerte gewichtet. Dieses Ergebnis ist im Hinblick auf den
 Schmelzpunkt von Wolfram zu hoch und unrealistisch. Der Schmelzpunkt liegt bei
 \SI{3422}{\degreeCelsius} \cite{wolfschmelz} und somit unter dem ermittelten Betriebswert.


 \begin{table}
  \centering
  \caption{Die gemessenen Stromstärken in Abhängigkeit der Saugspannung bei einem Heizstrom von $\SI{2,5}{\ampere}$.}
  \input{build/tabV2.tex}
  \label{tab:tabV2}
 \end{table}

 \begin{figure}
  \centering
  \caption{Die halblogarithmische Darstellung der maximalen Kennlinie in Abhängigkeit der Gegenspannung.}
  \includegraphics[width=\linewidth-70pt,height=\textheight-70pt,keepaspectratio]{build/KennlinielogV2.pdf}
  \label{fig:Graphlog3}
 \end{figure}

\subsection{Bestimmung der Kathodentemperatur mithilfe der Heizleistung}
Zum Vergleich werden die Kathodentemperaturen zu den anderen Heizströmen über die Heizleistung bestimmt.
Mithilfe von \ref{eq:TausLeistung} folgen die Temperaturen in \ref{tab:tabheit2}. Es wird ein Leistungsverlust durch Wärmeleitung $N_\text{WL}$ von $\SI{1}{\watt}$ angenommen.
Die Fläche $f$ beträgt $\SI{0.32}{\centi\meter\squared}$.
Auffällig ist, dass die hier bestimmten Temperaturen signifikant kleiner sind,
als die über die Anlaufstrommethode. Nach \ref{eq:TausLeistung} ist ein solcher Anstieg auch nicht zu erwarten.

\begin{table}
 \centering
 \caption{Die Kathodentemperatur in Abhängigkeit der Heizleistung.}
 \input{build/tabheiz.tex}
 \label{tab:tabheit2}
\end{table}

\subsection{Bestimmung der Austrittsarbeit des verwendeten Kathodenmaterials}
Durch Umstellung der Richardson-Gleichung in \ref{eq:Richardson-Gleichung} nach
$\phi$ folgt:
\begin{equation}
  \phi = -\ln\left(\frac{I_\text{S} h^3}{4\pi e_0 m_0 f k^2 T^2}\right) \frac{k T}{e_0}
\end{equation}
Diese liefert die Austrittsarbeit in $\si{\electronvolt}$. Die aus den Heizleistungen
bestimmten Temperaturen ergeben eine Austrittsarbeit von $\SI{2.19(7)}{\electronvolt}$
