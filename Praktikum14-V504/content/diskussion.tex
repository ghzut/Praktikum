
\section{Diskussion}
\label{sec:Diskussion}

 \begin{table}
 	\centering
 	\caption{Ergebnisse.}
 	\input{build/tabs.tex}
 \end{table}

Die erstellten Kennlinien folgen dem in der Theorie \ref{fig:Kennlinie} dargestellten Verlauf.
 Auch die Kennlinie unter einem Heizstrom von $I_\text{f} = \SI{2.5}{\ampere}$
 besitzt einen Sättigungsstrom. Dieser konnte jedoch nicht ermittelt werden.
Die logarithmische Darstellung dieser Kennlinie zeigt leichte Schwankungen im
unteren und oberen Bereich der X-Achse. Dies folgt, da die Ausgleichsgerade auf
den Daten in der Mitte basiert, da dort das Raumladungsgebiet vermutet wird.
Die Strommessung im Anlaufstromgebiet zeigt, in der halblogarithmischen
Darstellung \ref{fig:Graphlog3}, Schwankungen. Möglichen Fehlerquellen sind im
Versuchsaufbau zu finden. Da sich die gemessenen Ströme im $\si{\nano\ampere}$
Bereich befinden, wird ein sehr empfindliches Messgerät mit Verstärker benötigt.
Dieses ist störanfällig und benötigt eine sehr kurze Leitung zwischen Anode und Eingang.
Zudem kommt es beim Amperemeter zu Schwankungen, wenn sich Objekte in der Nähe der
Leitung befinden. Zudem verfälscht der Übergangswiderstand zwischen Stecker und
Buchse das Ergebnis aufgrund seiner exponentiellen Spannungsabhängigkeit das Ergebnis.
 Er kann auch nicht komplett behoben werden. Eine zusätzliche Fehlerquelle sind
 Folgen der direkten Heizung. Abweichungen aufgrund des internen Widerstandes
 sind irrelevant klein. Aufgrund dessen liegt der hiermit bestimmte Temperaturwert
 deutlich über den per Heizleistung bestimmten Temperaturen. Die ermittelte
 Austrittsarbeit liegt $\SI{3}{\percent}$ über dem Literaturwert von $\SI{4.5}{\electronvolt}$ \cite{wolfaus}. Dies einspricht drei Standartabweichungen. Dies kann durch einen der zuvor genannten Gründe zustande kommen, jedoch auch durch nicht berücksichtigte Effekte und durch Missachtung der Ablesefehler.
