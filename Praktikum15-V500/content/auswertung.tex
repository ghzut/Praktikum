\section{Auswertung}
\label{sec:Auswertung}

Die Graphen wurden sowohl mit Matplotlib \cite{matplotlib} als auch NumPy \cite{numpy} erstellt. Die
 Fehlerrechnung wurde mithilfe von Uncertainties \cite{uncertainties} durchgeführt. 

\subsection{a)}
\begin{center}
	\begin{table}
		\caption{Die gemessenen Stromstärken $I$ in Abhängigkeit der Saug- bzw. Gegenspannung $U$ bei den Wellenlängen von $\SI{577}{\nano\meter}$ und $ \SI{579}{\nano\meter}$.}
		\begin{minipage}[t]{0.5\textwidth}
			\centering
			\input{build/tabgelb1.tex}
		\end{minipage}
		\begin{minipage}[t]{0.5\textwidth}
			\centering
			\input{build/tabgelb2.tex}
		\end{minipage}
	\end{table}
\end{center}
\begin{figure}
	\centering
	\caption{Die Wurzel der Stromstärke $I$ gegen die Gegenspannung $U$ bei den Wellenlängen von $\SI{577}{\nano\meter}$ und $ \SI{579}{\nano\meter}$ aufgetragen.}
	\includegraphics[width=\linewidth-70pt,height=\textheight-70pt,keepaspectratio]{build/sqrtIgegenUgelb.pdf}
	\label{fig:Graphgelb1}
\end{figure}
\begin{table}
	\centering
	\caption{Die gemessenen Stromstärken $I$ in Abhängigkeit der Saug- bzw. Gegenspannung $U$ bei einer Wellenlänge von $\SI{546}{\nano\meter}$.}
	\input{build/tabgruen.tex}
\end{table}
\begin{figure}
	\centering
	\caption{Die Wurzel der Stromstärke $I$ gegen die Gegenspannung $U$ bei einer Wellenlänge von $\SI{546}{\nano\meter}$ aufgetragen.}
	\includegraphics[width=\linewidth-70pt,height=\textheight-70pt,keepaspectratio]{build/sqrtIgegenUgruen.pdf}
	\label{fig:Graphgruen}
\end{figure}
\begin{table}
	\centering
	\caption{Die gemessenen Stromstärken $I$ in Abhängigkeit der Gegenspannung $U$ bei einer Wellenlänge von $\SI{492}{\nano\meter}$.}
	\input{build/tabblaugruen.tex}
\end{table}
\begin{figure}
	\centering
	\caption{Die Wurzel der Stromstärke $I$ gegen die Gegenspannung $U$ bei einer Wellenlänge von $\SI{492}{\nano\meter}$ aufgetragen.}
	\includegraphics[width=\linewidth-70pt,height=\textheight-70pt,keepaspectratio]{build/sqrtIgegenUblaugruen.pdf}
	\label{fig:Graphblaugruen}
\end{figure}
\begin{table}
	\centering
	\caption{Die gemessenen Stromstärken $I$ in Abhängigkeit der Gegenspannung $U$ bei den Wellenlängen von  $\SI{434}{\nano\meter}$, $\SI{435}{\nano\meter}$ und $ \SI{436}{\nano\meter}$.}
	\input{build/tabviolett.tex}
\end{table}
\begin{figure}
	\centering
	\caption{Die Wurzel der Stromstärke $I$ gegen die Gegenspannung $U$ bei den Wellenlängen von  $\SI{434}{\nano\meter}$, $\SI{435}{\nano\meter}$ und $ \SI{436}{\nano\meter}$ aufgetragen.}
	\includegraphics[width=\linewidth-70pt,height=\textheight-70pt,keepaspectratio]{build/sqrtIgegenUviolett.pdf}
	\label{fig:Graphviolett}
\end{figure}
\begin{table}
	\centering
	\caption{Die gemessenen Stromstärken $I$ in Abhängigkeit der Gegenspannung $U$ bei den Wellenlängen von $\SI{405}{\nano\meter}$ und $ \SI{408}{\nano\meter}$.}
	\input{build/tabultraV1.tex}
\end{table}
\begin{figure}
	\centering
	\caption{Die Wurzel der Stromstärke $I$ gegen die Gegenspannung $U$ bei den Wellenlängen von $\SI{405}{\nano\meter}$ und $ \SI{408}{\nano\meter}$ aufgetragen.}
	\includegraphics[width=\linewidth-70pt,height=\textheight-70pt,keepaspectratio]{build/sqrtIgegenUultraV1.pdf}
	\label{fig:GraphultraV1}
\end{figure}
\begin{table}
	\centering
	\caption{Die gemessenen Stromstärken $I$ in Abhängigkeit der Gegenspannung $U$ bei den Wellenlängen von $\SI{365}{\nano\meter}$ und $ \SI{366}{\nano\meter}$.}
	\input{build/tabultraV2.tex}
\end{table}
\begin{figure}
	\centering
	\caption{Die Wurzel der Stromstärke $I$ gegen die Gegenspannung $U$ bei den Wellenlängen von $\SI{365}{\nano\meter}$ und $ \SI{366}{\nano\meter}$ aufgetragen.}
	\includegraphics[width=\linewidth-70pt,height=\textheight-70pt,keepaspectratio]{build/sqrtIgegenUultraV2.pdf}
	\label{fig:GraphultraV2}
\end{figure}

\subsection{b)}
\begin{table}
	\centering
	\caption{???.}
  	\input{build/tabZwischenErgebnisse.tex}
\end{table}
\begin{figure}
	\centering
	\caption{Die berechnete Grenzspannung $U_\text{g}$ bei den verschiedenen Wellenlängen gegen die Frequenz $f$ aufgetragen.}
	\includegraphics[width=\linewidth-70pt,height=\textheight-70pt,keepaspectratio]{build/Ugegennu.pdf}
	\label{fig:GraphUgegennu}
\end{figure}
 

\subsection{c)}
\begin{figure}
	\centering
	\caption{Die Wurzel der Stromstärke $I$ gegen die Saug- bzw. Gegenspannung $U$ bei den Wellenlängen von $\SI{577}{\nano\meter}$ und $ \SI{579}{\nano\meter}$ aufgetragen.}
	\includegraphics[width=\linewidth-70pt,height=\textheight-70pt,keepaspectratio]{build/sqrtIgegenUgelb2.pdf}
	\label{fig:Graphgelb2}
\end{figure}





 

 