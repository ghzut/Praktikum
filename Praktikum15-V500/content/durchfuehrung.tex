
\section{Durchführung}
\label{sec:Durchführung}
Zunächst wird der unter verschiedenen Wellenlängen auftretende Photostrom in
Abhängigkeit der angelegten Gegenspannung gemessen. Hierzu wird die Öffnung der
Photodiode in den ersten der aufgespaltenen Lichtstrahlen gesetzt. Um den
Lichstrahl scharf zu stellen und eine möglichst hohe Lichtintensität zu erhalten,
 können die optischen Instrumente verschoben werden.
 Anschließend wird die angelegte Gegenspannung varriert und der zugehörige Strom
 über das Pikoamperemeter notiert. Die Gegenspannung wird solange erhöht, bis sich
 kein Strom mehr messen lässt. falls die anfänglichen Ströme bereits zu gering sind,
 werden zusätzlich einige Werte unter einer Beschleunigungsspannung entnommen.
 Da die messbaren Ströme im Pikoamperebereich liegen,
 wird ein sehr empfindliches benötigt. Daher muss ein Koaxialkabel verwendet werden
 und das Pikoamperemeter geerdet sein. Dieser Vorgang wird mit allen erkennbaren
 Linien des Quecksilberlampe wiederholt.
 Anschließend wird der Versuch nochmals mit der gelben Kennlinie des Spektrums
 wiederholt. Dieses mal wird die Gegenspannung jedoch im Bereich von
 $\SI{-20}{\volt}$ bis $\SI{20}{\volt}$ varriert. Da das zu Verfügung stehende
 Pikoampere jedoch keine negativen Ströme messen kann, kann bei die Variation der
 Gegenspannung bei einem abfallen des Stromes auf Null beendet werden.
