
\section{Durchführung}
\label{sec:Durchführung}
Zum Erstellen einer Kennlinienschar der Hochvakuum-Diode 1 wird die Schaltung in Abbildung \ref{fig:Schaltung1} aufgebaut. Der Heizstrom wird von $\SI{2.0}{\ampere}$ bis $\SI{2.5}{\ampere}$ in $\SI{0.1}{\ampere}$-Schritten variiert und zusammen mit der zugehörigen Spannung notiert. Für jede Heizstromstärke wird nun de Saugspannung von $\SI{0}{\volt}$ bis $\SI{250}{\volt}$ variiert und zusammen mit der zugehörigen Stromstärke notiert.
Zum Erstellen einer Kennlinie im Anlaufstromgebiet wird die Schaltung aus Abbildung \ref{fig:Schaltung2} aufgebaut. Die Gegenspannung wird von $\SI{0}{\volt}$ bis $\SI{-1}{\volt}$ variiert und zusammen mit der zugehörigen Anlaufstromstärke notiert.
