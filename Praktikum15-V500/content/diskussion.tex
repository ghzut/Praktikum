
\section{Diskussion}
\label{sec:Diskussion}

 \begin{table}
  \centering
  \caption{Die Ergebnisse der Auswertung.}
  \input{build/ergebnisse.tex}
 \end{table}

Schon bei der Durchführung ist aufgefallen, dass die Intensität des Lichtes sich
 sprunghaft änderte. Dadurch kamen Schwankungen in der Größenordnung der gemessenen
  Werte für den Photostrom zustande. Diese wurden so gut wie möglich herausgefiltert
   indem gewartet wurde, bis sich wieder eine annähernd gleiche Intensität eingestellt
   hatte. Dennoch sind besonders in dem Graphen \ref{fig:GraphultraV2} dadurch
   entstandene Schwankungen erkennbar. Zusätzlich können die Ergebnisse durch
   ungewollte Lichteinstrahlung verfälscht worden sein. Eine weitere mögliche Fehlerquelle
   ist das Verbindungskabel zwischen Diode und Pikoamperemeter, welches aufgrund der
   geringen Ströme möglicherweise nicht ausreichend abgeschirmt ist.
   Die in Abb. \ref{fig:GraphUgegennu} bestimmten Grenzspannungen
 bilden den linearen Zusammenhang zwischen $U_\text{g}$ und $f$ hingegen hinreichend ab. Das aus
   der Steigung der Geraden ermittelte Verhältnis $h/e_0$ weicht um
   $\SI{3}{\percent}$ nach unten bezüglich des Literaturwertes
   $h/e_0=\SI{4.13e-15}{\volt\second}$ \cite{nistgov} ab. Aufgrund der beiden, letztgenannten Punkte ist zu vermuten
, dass das berechnete Verhältnis aussagekräftig ist. Für $A_\text{k}$ ist diese Aussage nicht erfüllt. Da das Kathodenmaterial
nicht bekannt ist, kann ein systematischer Fehler bei $A_\text{k}$ nicht ausgeschlossen werden. 
