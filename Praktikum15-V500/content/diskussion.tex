
\section{Diskussion}
\label{sec:Diskussion}

 \begin{table}
  \centering
  \caption{Die Ergebnisse der Auswertung.}
  \input{build/ergebnisse.tex}
 \end{table}

Schon bei der Durchführung ist aufgefallen, dass die Intensität des Lichtes sich
 sprunghaft änderte. Dadurch kamen Schwankungen in der Größenordnung der gemessenen
  Werte für den Photostrom zustande. Diese wurden so gut wie möglich herausgefiltert
   indem gewartet wurde, bis sich wieder eine annähernd gleiche Intensität eingestellt
   hatte. Dennoch sind besonders in dem Graphen \ref{fig:GraphultraV2} dadurch
   entstandene Schwankungen erkennbar. Zusätzlich können die Ergebnisse durch
   ungewollte Lichteinstrahlung verfälscht werden. Die Wertepaare in Abbildung
   \ref{fig:GraphUgegennu} liegen hingegen auf einer Geraden. Auch das aus
   der Steigung der Geraden ermittelte Verhältnis $h/e_0$ weicht nur um
   $\SI{3}{\percent}$ nach unten bezüglich des Literaturwertes
   $h/e_0=\SI{4.13e-14}{\volt\second}$ \cite{nistgov} ab.
