
\section{Theorie}
\label{sec:Theorie}
\subsection{Die Natur von Schallwellen in Medien}
Schallwellen sind longitudinale Wellen des Druckes der Form:
\begin{equation}
  p \left( x , t \right) = p_0 + v_0 Z \cdot \cos \left(\omega t -kx \right)\text{,}\label{eq:Schallwelle}
  \end{equation}
mit der akustischen Impedanz $Z = c \rho$ mit der Dichte $\rho$. Innerhalb von Festkörpern kommt es jedoch
aufgrund von Schubspannungen auch zu transversalen Komponenten.
Schallwellen oberhalb des hörbaren Bereiches von $\SI{22}{\kilo\hertz}$ werden als
Ultraschall, oberhalb von $\SI{1}{\giga\hertz}$ als Hyperschall bezeichnet. Die
Schallgeschwindigkeit in Flüssigkeiten und Gasen lassen sich durch:
\begin{equation}
  c_\text{Fl} = \sqrt{\frac{1}{\kappa \cdot \rho}}\label{Fl}
\end{equation}
mit der Kompressibilität $\kappa$. In Festkörpern folgt die Schallgeschwindigkeit:
\begin{equation}
  c_\text{Fe} = \sqrt{\frac{E}{\rho}}\label{Fe}
\end{equation}
, mit dem Elastizitätsmodul $E$.
Aufgrund der Struktur des Festkörpers sind die $c_\text{Fe}$ im allgemeinen Richtungsabhängig.
In der Regel kommt es zu einem Energieverlust der Schallwelle. Die Intensität nimmt daher mit
\begin{equation}
  I\left(x\right) = I_0 \exp\left(\alpha x\right)\label{I}
\end{equation}
ab. Der Absorptionskoeffizient $\alpha$ beschreibt dabei, wie schnell die Wellenintensität
abnimmt. Da $\alpha_\text{Luft}$ nur sehr gering ist wird in der Praxis ein Kontaktmittel
zwischen Schallgeber und Beobachtungsmaterial eingesetzt.
Trifft die Schallwelle auf eine Grenzfläche kommt es zur teilweisen Reflexion dieser.
Für das Verhältnis der Intensitäten zwischen reflektiertem und einfallendem Anteil gilt:
\begin{equation}
  R = \left(\frac{Z_1-Z_2}{Z_1+Z_2}\right)^2
\end{equation}
Für den transmittierten Anteil folgt analog:$ T = 1-R $.
\subsection{Die Erzeugung von hochfrequenten Ultraschallwellen}
Eine mögliche Methode zur Erzeugung der benötigten, hochfrequenten Schallwellen
liegt im piezo--elektrischen Effekt. Hierzu wird ein Piezo-Kristall z.B ein Quarz
in elektrisches Wechselfeld gebracht. beim Kristall kommt es infolge dessen zur
Ausdehnung und Zusammenziehen mit der angelegten Frequenz des Wechselfelds. Um
ausreichend große Ausdehnungsamplituden zu erzielen muss als Anregungsfrequenz
die Eigenfrequenz des Kristalls angelegt sein. Diese variiert je nach Kristall
und Einbau. Umgekehrt können auch Ultraschallwellen mithilfe des Kristalls
ausgewertet werden, da sie diesen zu Schwingungen im Wechselfeld anregen.
\subsection{Das Durchschallungsverfahren}
beim Durchschallungsverfahren werden eine Sender und ein Empfängersonde an je
eine Seite der Probe angebracht. Es wird nun ein kurzzeitiger Schallimpuls ausgesendet, welcher
anschließend vom Empfänger gemessen wird und in Form eines Peaks dargestellt wird.
Befindet sich jedoch eine Fehlstelle im gemessenen Material misst die Empfängersonde einen geringeren Peak als den ausgesendeten.
Es lässt sich jedoch keine Aussage machen, an welcher Stelle sich die Fehlstelle befindet.

\subsection{Das Impuls-Echo-Verfahren}
Beim Impuls-Echo-Verfahren wird die verwendete Sonde sowohl als Sender als auch
als Empfänger verwendet. Es wird wieder ein kurzzeitiger Schallimpuls ausgesendet,
welcher anschließend an der Grenzfläche reflektiert wird und von der Sonde wieder aufgenommen wird.
Liegt eine Fehlstelle vor, wird diese als weiterer Peak im Graphen deutlich. Ist
die Schallgeschwindigkeit im verwendeten Medium bekannt, kann aus dem Graphen
auch die Tiefe der Fehlstelle ermittelt werden. 
