
\section{Theorie}
\label{sec:Theorie}
\subsection{Die Natur von Schallwellen in Medien}
Schallwellen sind longitudinale Wellen des Druckes.
%wellen der Form:
%\begin{equation}
%  p \left( x , t \right) = p_0 + v_0 Z \cdot \cos \left(\omega t -kx %\right)\text{,}\label{eq:Schallwelle}
%  \end{equation}
%unter der akustischen Impedanz $Z = c \rho$ mit der Dichte $\rho$.
 Innerhalb von Festkörpern besitzen Schallwellen
aufgrund von Schubspannungen jedoch auch transversalen Schwingungskomponenten.
Schallwellen oberhalb des hörbaren Bereiches von $\SI{22}{\kilo\hertz}$ werden als
Ultraschall, oberhalb von $\SI{1}{\giga\hertz}$ als Hyperschall bezeichnet.
% Die
%Schallgeschwindigkeit in Flüssigkeiten und Gasen lassen sich durch:
%\begin{equation}
%  c_\text{Fl} = \sqrt{\frac{1}{\kappa \cdot \rho}}\text{, }\label{Fl}
%\end{equation}
%mit der Kompressibilität $\kappa$ beschreiben. In Festkörpern folgt die %Schallgeschwindigkeit:
%\begin{equation}
%  c_\text{Fe} = \sqrt{\frac{E}{\rho}}\text{, }\label{Fe}
%\end{equation}
%mit dem Elastizitätsmodul $E$.
Aufgrund der Struktur eines Festkörpers ist die Ausbreitungsgeschwindigkeit in  diesen $c_\text{Fe}$ im allgemeinen Richtungsabhängig.
In der Regel kommt es zu einem Energieverlust der Schallwelle. Die Intensität nimmt daher mit
\begin{equation}
  I\left(x\right) = I_0 \exp\left(\alpha x\right)\label{I}
\end{equation}
ab. Der Absorptionskoeffizient $\alpha$ beschreibt, wie schnell die Wellenintensität
abnimmt. Da $\alpha$ für Luft nur sehr gering ist, wird in der Praxis ein Kontaktmittel
zwischen Schallgeber und Beobachtungsmaterial eingesetzt. Ein weiterer Aspekt ist die Wellenreflexion.
Trifft die Schallwelle auf eine Grenzfläche, kommt es zur teilweisen Reflexion der Welle.
%Für das Verhältnis der Intensitäten zwischen reflektiertem und einfallendem Anteil gilt:
%\begin{equation}
%  R = \left(\frac{Z_1-Z_2}{Z_1+Z_2}\right)^2
%\end{equation}
%Für den transmittierten Anteil folgt analog:$ T = 1-R $.

\subsection{Die Erzeugung von hochfrequenten Ultraschallwellen}
Eine mögliche Methode zur Erzeugung der benötigten, hochfrequenten Schallwellen
ist das Ausnutzen des piezo--elektrischen Effektes. Hierzu wird ein Piezo-Kristall z.B ein Quarz
in ein elektrisches Wechselfeld gebracht. Der Kristall dehnt sich infolge dessen
mit der angelegten Frequenz des Wechselfeldes aus und zieht sich später wieder zusammen.
Ausdehnung und Zusammenziehen mit der angelegten Frequenz des Wechselfelds. Um
ausreichend große Ausdehnungsamplituden zu erzielen muss die Anregungsfrequenz
mit der Eigenfrequenz des Kristalls übereinstimmen. Letztere variiert je nach Kristall
und Form des Einbaus. Umgekehrt können auch Ultraschallwellen mithilfe des Kristalls
ausgewertet werden, da sie diesen zu Schwingungen im Wechselfeld anregen.
\subsection{Das Durchschallungsverfahren}
Beim Durchschallungsverfahren werden eine Sender und eine Empfängersonde an je
einer Seite der Probe angebracht. Nun wird ein kurzzeitiger Schallimpuls ausgesendet, welcher
anschließend vom Empfänger gemessen wird und im Graphen einen Peak abzeichnet. Aus
der messbaren Zeitdifferenz zwischen Start und Endpeak lässt sich die Schallgeschwindigkeit im Medium ermitteln. Aufgrund dieser fällt der Endpeak auch bei einer fehlerfreien Probe geringfügig kleiner aus. Fällt der Endpeak jedoch weitaus kleiner aus, lässt dies auf eine Fehlstelle im Material schließen.
Es lässt sich jedoch keine Aussage machen, an welcher Stelle sich die Fehlstelle befindet.

\subsection{Das Impuls-Echo-Verfahren}
Beim Impuls-Echo-Verfahren findet die benutzte Sonde sowohl als Sender als auch
als Empfänger Verwendung. Es wird wieder ein kurzzeitiger Schallimpuls ausgesendet,
welcher anschließend an der Grenzfläche reflektiert wird und von der Sonde wieder aufgenommen wird.
Liegt eine Fehlstelle vor, ist diese als weiterer Peak im Graphen erkennbar. Ist
die Schallgeschwindigkeit im verwendeten Medium bekannt, kann aus dem Graphen
auch die Tiefe der Fehlstelle ermittelt werden.
