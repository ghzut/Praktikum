
\section{Durchführung}
\label{sec:Durchführung}
Als erstes werden die Längen aller Acrylzylinder mit einer Schieblehre vermessen.
Der allgemeine Gain wird so eingestellt, dass die Peaks des Impuls Echo-Verfahrens
noch komplett sichtbar sind.
Im Anschluss wird zunächst das Impuls-Echo-Verfahren an den Zylindern erprobt.
Hierzu wird der Zylinder auf ein Papiertaschentuch gestellt. Anschließend wird eine Sonde
 mithilfe von bidestiliertem Wasser von oben angekoppelt. Das Ultraschallechoskop
 ist so eingestellt, dass die verwendete Sonde sowohl Sender als auch Empfänger ist.
 Nun wird ein AScan durchgeführt und die Daten der auftretenden Peaks werden notiert.
 Dies wird für die anderen fünf Zylinderlängen wiederholt. Falls die auftretenden
 Peaks zu gering oder zu hoch ausfallen muss das TGC gegebenenfalls nachjustiert werden.
  Auch diese Parameter werden notiert
 Als nächstes wird der Versuch nochmals mit dem Durschallungsverfahren wiederholt.
 Hierzu wird der jeweilige Zylinder in eine Halterung gelegt und von beiden
 Seiten mit Koppelgel an jeweils eine Sonde angekoppelt. Das Ultraschallechoskop
 wird entsprechend eingestellt. auch hier wird ein AScan durchgeführt und die
 entsprechenden Daten notiert.
 Nun wird eine spektrale Analyse und eine Ceptrumbestimmung durchgeführt. Hierzu
 werden zwei Acrylscheiben unterschiedlicher Dicke mit bidestliertem wasser gekoppelt.
 Die dünnere Scheibe liegt ganz unten. Um die auftretenden Peaks besser vom
 anfänglichen Peak an der Grenzschicht Sonde, Acryl zu trennen wird ein Zylinder
 dazwischen gekoppelt. Das TGC wird so eingestellt, dass 3 Peaks zu erkennen sind.
 Die Marker werden so gesetzt das bei der nachfolgenden FFT Analyse alle Peaks
 verwendet werden.
 Zuletzt wird ein Augenmodell im Maßstab 3:1 untersucht. Hierzu wird wieder eine
 Sonde im Impuls-Echo-Verfahren mit Koppelgel an die künstliche Hornhaut angekoppelt
 und leicht bewegt bis ein Echo der Augenrückwand zu sehen ist. Auch hier werden
 die Positionsdaten der Peaks notiert. Es ist darauf zu achten, dass kein übermäßiger
 Druck auf die Hornhaut angewendet wird, da das Augenmodell sonst beschädigt werden könnte.   
