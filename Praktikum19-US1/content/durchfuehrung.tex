
\section{Durchführung}
\label{sec:Durchführung}
Als erstes werden die Längen aller Acrylzylinder mit einer Schieblehre vermessen.
Der Gain wird so eingestellt, dass die resultierenden Peaks eines AScans im Impuls Echo-Verfahren
noch komplett sichtbar sind.
Im Anschluss wird zunächst das Impuls-Echo-Verfahren an den Zylindern erprobt.
Hierzu wird der Zylinder auf ein Papiertaschentuch gestellt. Anschließend wird eine Sonde
 mithilfe von bidestiliertem Wasser von oben angekoppelt. Das Ultraschallechoskop
 wird so eingestellt, dass die verwendete Sonde sowohl als Sender als auch als Empfänger fungiert.
 Nun wird ein AScan durchgeführt und die Koordinaten der auftretenden Peaks, sowie der zugehörige TCG-Wert, werden notiert.
 Dies wird für die anderen fünf Zylinderlängen wiederholt. Falls die auftretenden
 Peaks zu gering oder zu hoch ausfallen muss die TCG-Einstellung gegebenenfalls nachjustiert werden.
 Aus diesen lässt sich sowohl die Dämpfungskosntante als auch die
 Schallgeschwindigkeit in Acryl bestimmen.
 Als nächstes wird der Versuch nochmals mit dem Durchschallungsverfahren wiederholt.
 Hierzu wird der jeweilige Zylinder in eine Halterung gelegt und von beiden
 Seiten mit Koppelgel an jeweils eine Sonde angekoppelt. Die Eingänge des Ultraschallechoskops
 werden entsprechend eingestellt. Auch hier wird ein AScan durchgeführt und die
 entsprechenden Daten werden notiert.
 Nun wird eine spektrale Analyse sowie eine Cepstrumbestimmung durchgeführt. Hierzu
 werden zwei Acrylscheiben unterschiedlicher Dicke mit bidestiliertem Wasser gekoppelt.
 Die dünnere Scheibe liegt ganz unten. Um die auftretenden Peaks besser vom
 Anfangspeak an der Grenzschicht von Sonde und Acryl zu trennen wird ein Zylinder
 zwischen Sonde und Platten gekoppelt. Der TCG wird so eingestellt, dass 3 Peaks zu erkennen sind.
 Die Curser werden so gesetzt, dass alle Peaks im Bereich zwischen ihnen liegen. Daraufhin wird eine
 FFT-Analyse über den entsprechenden Menüpunkt gestartet. Diese liefert Spektrum und Cepstrum.
 Auch hier werden die Koordinaten der Peaks notiert.
 Zuletzt wird ein Augenmodell untersucht. Hierzu wird wieder eine
 Sonde im Impuls-Echo-Verfahren mit Koppelgel an die künstliche Hornhaut des Modells angekoppelt
 und leicht bewegt bis ein Echo der Augenrückwand zu sehen ist. es werden wieder
 die Positionsdaten der Peaks notiert. Es ist darauf zu achten, dass kein übermäßiger
 Druck auf die Hornhaut ausgeübt wird, da das Augenmodell sonst beschädigt werden könnte.
