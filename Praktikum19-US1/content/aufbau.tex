\section{Aufbau}
\label{sec:Aufbau}
Der Versuchsaufbau besteht aus 2 Messsonden, welche mit $\SI{2}{\mega\hertz}$
betrieben werden, einem Ultraschallechoskop und einem Computer zur Aufnahme der Daten.
Am Ultraschallechoskop lässt sich einstellen ob die jeweilige Sonde als Sender,
Empfänger oder beides fungieren soll. Zusätzlich kann eine Verstärkung der
auftretenden Signale sowohl über einen allgemeinen GAIN als auch über eine zeitlich
veränderliche Tgc verstärkt werden. Bei letzterer kann eine linear ansteigende Verstärkung
über die Parameter $Threshold$, $Wide$, $Slope$ und $Start$
modifiziert werden. Die gemessenen Signale werden mithilfe der
Software AScan ausgewertet. Ein Ascan lässt sich über den Menüpunkt Ascan starten. Im oberen
Graphen werden die auftretenden Peaks, im unteren der zugehörige Tgc Wert
dargestellt. Mithilfe von einblendbaren Cursern können die Peaks präziser vermessen
werden. Über die Funktion FFT wird zusätzlich das Spektrum sowie ihr zugehöriges
Cepstrum angezeigt. Für die dafür benötigten Berechnungen wird nur der
Graphenabschnitt verwendet der sich zwischen zwei Cursern befindet.
