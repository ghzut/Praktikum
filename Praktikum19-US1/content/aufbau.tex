\section{Aufbau}
\label{sec:Aufbau}
Der Versuchsaufbau besteht aus 2 Messsonden, welche mit $\SI{2}{\mega\hertz}$
betrieben werden, einem Ultraschallechoskop und einem Computer zur Aufnahme der Daten.
Am Ultraschallechoskop lässt sich einstellen ob die jeweilige Sonde als Sender,
Empfänger oder beides fungieren soll. Zusätzlich kann eine Verstärkung der
auftretenden Signale sowohl über einen allgemeinen GAIN als auch über eine zeitlich
veränderliche Time gain control verstärkt werden. Bei letzterer kann ein linearer
Anstieg über die Parameter $Threshold$, $Wide$, $Slope$ und $Start$ beliebig
modifiziert werden. Die gemessenen Signale werden mithilfe der
Software AScan ausgewertet. Ein Ascan wird den Punkt Ascan gestartet. Im oberen
Graphen werden die auftretenden Peaks, im unteren die zeitabhängige Verstärkung
dargestellt.Mithilfe von einblendbaren Cursern können die Peaks geneauer vermessen
werden. Über die Funktion FFT wird zusätzlich das Spektrum und ihr zugehöriges
Ceptrum angezeigt. Für die zugehörigen Berechnungen werden nur die Daten vewendet
die sich zwischen den zwei Cursern befinden. 
