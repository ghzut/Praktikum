
\section{Diskussion}
\label{sec:Diskussion}
\begin{table}
	\centering
	\caption{Die in der Auswertung bestimmten Werte mit zugehörigen Vergleichswerten und die relative Abweichung von diesen.}
	\input{build/Ergebnisse.tex}
\end{table}
Die in Tabelle \ref{tab:Ergebnisse} dargestellten Ergebnisse aus der Auswertung sollen nun diskutiert werden.
Die bestimmte Schallgeschwindigkeit in Acryl $c_2$, welche mit dem Impuls-Echo-Verfahren bestimmt wurde weicht um ABWEICHUng vom Literaturwert \cite{cAcryl} ab. Die mit dem Durchschallungs-Verfahren bestimmte Schallgeschwindigkeit in Acryl $c_1$ weicht stärker (ABWEICHUNG) von dem Literaturwert \cite{cAcryl} ab.Eine Ursache hierfür ist unter anderem, dass die im Acrylzylinder zurückgelegte Strecke beim Impuls-Echo-Verfahren doppelt so groß ist wie beim Durchschallungs-Verfahren und dadurch Schwankungen in der Laufzeit des Schalls außerhalb des Acrylzylinders nur den halben Einfluss haben. Auch die Messfehler der Laufzeit im Zylinder halbieren sich dadurch im Impuls-Echo-Verfahren im Vergleich zum Durchschallungs-Verfahren. Die bestimmte mittlere Laufzeit des Schalls vor Eintritt in die Zylinder beim Durchschallungs-Verfahren unterscheidet sich geringfügig von der beim Impuls-Echo-Verfahren. Ein Grund hierfür ist, dass bei der Messung mit Durchschallungs-Verfahren Ultraschallgel und bei der Messung mit Impuls-Echo-Verfahren bidestilliertes Wasser verwendet wurde. Der mit dem Impuls-Echo-Verfahren bestimmte DÄMPFUNGSKOEFFIZENT $\alpha$ weicht vom Literaturwert \cite{alphaAcryl} um ABWEICHUNG ab. Der Messwert liegt somit im Bereich akzeptablen Bereich von einer $\sigma$-Umgebung. Bei der Messung der Positionen der Peaks im Cepstrum wurde vermutlich ein Peak bei ca. $\SI{7.26}{\micro\second}$ übersehen, sonst würde sich mit der Laufzeit der Reflexion $T'_2=\SI{11.74}{\micro\second}$ eine deutlich zu dicke Dicke für die zweite Acrylplatte ergeben. Da die Peaks bei vielfachen der Laufzeiten der Reflexe und Kombinationen dieser liegen, ist die Differenz zwischen der Position des zweiten und ersten Peaks die Laufzeit der Reflexion in der zweiten Acrylplatte. Die unter dieser Annahme berechneten Werte für die Dicken der Acrylplatten weichen unter $\SI{1.6}{\percent}$ von den manuell gemessenen Werten ab. Für die Positionen der Peaks im Spektrum wurde nur der Wert des absoluten Maximums bei $\SI{1.78}{\mega\hertz}$ aufgenommen, woraus keine weiteren Werte ermittelt werden konnten. Die bestimmten Abstände im Augenmodell scheinen in der richtigen Größenordnung zu liegen. Die große Ungenauigkeit bei dem ersten Abstand $A_1$ lässt sich auf die Ungenauigkeit der Laufzeit des Schalls vor Eintritt in des Augenmodell erklären.
