\section{Auswertung}
\label{sec:Auswertung}


Die Graphen wurden sowohl mit Matplotlib \cite{matplotlib} als auch NumPy \cite{numpy} erstellt. Die
Fehlerrechnung wurde mithilfe von Uncertainties \cite{uncertainties} durchgeführt.

\subsection{Bestimmung der Schallgeschwindigkeit mit Durchschallungs-Verfahren}
\begin{table}
	\centering
	\caption{Die gemessenen Laufzeiten $T$ für die Acryl-Zylinder der Länge $l$ bei dem Durchschallungs-Verfahren.}
	\input{build/b.tex}
\end{table}
\begin{figure}
	\centering
	\caption{Die bei einer Laufzeit $t$ zurückgelegte Strecke $x$ des Schalls im Zylinder bei dem Durchschallungs-Verfahren gegen die Laufzeit $t$ aufgetragen.}
	\includegraphics[width=\linewidth-70pt,height=\textheight-70pt,keepaspectratio]{build/XgegenT2.pdf}
	\label{fig:XgegenT2}
\end{figure}
In der Abbildung \ref{fig:XgegenT2} sind die Messwertepaare aus Tabelle \ref{tab:b} aufgetragen und durch einen linearen Fit der Form $x=a t + b$ genähert worden.
Es folgt für die Schallgeschwindigkeit im Zylinder
\begin{equation}
	c_1=a=\SI{2714(21)}{\meter\per\second}\text{.}
\end{equation}
Nun lässt sich in dem Graphen in Abbildung \ref{fig:XgegenT2} erkennen, dass der Schall sich erst ab einer bestimmten Laufzeit überhaupt durch einen möglichen Zylinder bewegen kann. Dies ist durch die Strecke, die der Schalls vor Eintritt in und nach Austritt aus dem Zylinder zurücklegen muss zu erklären. Für die mittlere benötigte Zeit $T_{\text{A}1}$ um eine dieser Strecken zurückzulegen gilt 
\begin{equation}
	T_{\text{A}1}=-\frac{1}{2} \cdot \frac{b}{a}=\SI{0.5(1)}{\micro\second}\text{.}
\end{equation}



\subsection{Bestimmung der Schallgeschwindigkeit mit Impuls-Echo-Verfahren}
\begin{table}
	\centering
	\caption{Die gemessenen Laufzeiten $T$, Spannungen $U$, TGC-Werte und die daraus berechneten ursprünglichen Spannungen $U_\text{n}$ für die Acryl-Zylinder der Länge $l$ bei einem Gain von $\SI{10}{\decibel}$ bei dem Impuls-Echo-Verfahren.}
	\input{build/a.tex}
\end{table}
\begin{figure}
	\centering
	\caption{Die bei einer Laufzeit $t$ zurückgelegte Strecke $x$ des Schalls im Zylinder bei dem Impuls-Echo-Verfahren gegen die Laufzeit $t$ aufgetragen.}
	\includegraphics[width=\linewidth-70pt,height=\textheight-70pt,keepaspectratio]{build/XgegenT1.pdf}
	\label{fig:XgegenT1}
\end{figure}
In der Abbildung \ref{fig:XgegenT1} sind die Messwertepaare aus Tabelle \ref{tab:a} aufgetragen und durch einen linearen Fit der Form $x=a t + b$ genähert worden.
Es folgt für die Schallgeschwindigkeit im Zylinder
\begin{equation}
	c_2=a=\SI{2722(19)}{\meter\per\second}\text{.}
\end{equation}
Nun lässt sich in dem Graphen in Abbildung \ref{fig:XgegenT1} erkennen, dass der Schall sich erst ab einer bestimmten Laufzeit überhaupt durch einen möglichen Zylinder bewegen kann. Dies ist durch die Strecke, die der Schalls vor Eintritt in und nach Austritt aus dem Zylinder zurücklegen muss zu erklären. Für die mittlere benötigte Zeit $T_{\text{A}2}$ um eine dieser Strecken zurückzulegen gilt 
\begin{equation}
	T_{\text{A}2}=-\frac{1}{2} \cdot \frac{b}{a}=\SI{0.4(2)}{\micro\second}\text{.}
\end{equation}

\subsection{Bestimmung der Dämpfungskonstanten mit Impuls-Echo-Verfahren}
\begin{figure}
	\centering
	\caption{Der Logarithmus von der ursprünglichen Spannungen $U_\text{n}$ bei dem Impuls-Echo-Verfahren gegen die nach der Zeit $t$ zurückgelegten Strecke $x$ des Schalls im Zylinder aufgetragen.}
	\includegraphics[width=\linewidth-70pt,height=\textheight-70pt,keepaspectratio]{build/UgegenX.pdf}
	\label{fig:UgegenX}
\end{figure}
In der Abbildung \ref{fig:UgegenX} ist der Logarithmus von der ursprünglichen Spannung $U_\text{n}$ gegen die zurückgelegte Strecke $x$ des Schalls im Zylinder mit den Werten aus Tabelle \ref{tab:a} aufgetragen und durch einen linearen Fit der Form $x=a t + b$ genähert worden. Für den Dämpfungsfaktor $\alpha$ gilt, da die Intensität $I$ proportional zu $U_\text{n}^2$ ist, nach Formel \eqref{I}
\begin{equation}
	\alpha=-2 a = \SI{58(3)}{\per\meter}\text{.}
\end{equation}



\subsection{Untersuchung des Cepstrums/Spektrums}
Im Spektrum wurde ein Peak bei $\SI{1.78}{\mega\hertz}$ aufgenommen.
\begin{table}
	\centering
	\caption{Die gemessenen zeitliche Differenz $T$ zu dem ersten Peak im Cepstrum der gemessenen Peaks und die daraus berechneten zeitlichen Abstände $\Delta T$ zwischen diesen und dem jeweils vorherigem.}
	\input{build/c.tex}
\end{table}
In Tabelle \ref{tab:c} fällt auf, das sich die Werte für die gemessenen und bestimmten zeitlichen Differenzen durch Kombination der zeitlichen Differenzen von $T'_1=\SI{4.48}{\micro\second}$ und von $T'_2=\SI{7.26}{\micro\second}$ darstellen lassen. Da es sich hierbei um die zusätzlichen Laufzeiten der Reflexe in den Acrylplatten handelt lässt dies vermuten, dass diese die Laufzeiten des Schalls in den jeweiligen Acrylplatten sind.
Mit der bekannten Schallgeschwindigkeit in Acryl $c_2=a=\SI{2722(19)}{\meter\per\second}$ ergeben sich somit für die Dicken der Platten
\begin{equation}
D_1=\frac{T'_1}{2} c_2=\SI{6.10(4)}{\milli\meter}
\end{equation}
und
\begin{equation}
D_2=\frac{T'_2}{2} c_2=\SI{9.88(7)}{\milli\meter}\text{.}
\end{equation}


\subsection{Untersuchung der Abstände in einem Augenmodell}
\begin{table}
	\centering
	\caption{Die gemessenen Werte für die Laufzeiten $T$ des Schalls und die unter Einbeziehung der Laufzeiten außerhalb des Augenmodells $T_{\text{A}1}$ berechneten Werte für die Laufzeit des Schalls $T_\text{I}$ im Augenmodell.}
	\input{build/d.tex}
\end{table}
Mit den Werten für die Laufzeiten des Schalls im Auge aus Tabelle \ref{tab:d} lässt sich mit den bekannten Schallgeschwindigkeiten von $c_\text{GK}=\SI{1410}{\meter\per\second}$ außerhalb der Linse und von $c_\text{L}=\SI{2500}{\meter\per\second}$ in der Linse die Abstände im Augenmodell berechnen. Für den Abstand zwischen der Hornhaut und der Iris ergibt sich
\begin{equation}
A_1=\frac{T_{\text{I}1}}{2 } c_\text{GK}=\SI{7(2)}{\milli\meter}\text{,}
\end{equation}
für den Abstand zwischen Iris und Linse
\begin{equation}
A_2=\frac{T_{\text{I}2}-T_{\text{I}1}}{2 }c_\text{GK}=\SI{4.413}{\milli\meter}\text{,}
\end{equation}
für die Dicke der Linse
\begin{equation}
A_3=\frac{T_{\text{I}3}-T_{\text{I}2}}{2 }c_\text{L}=\SI{9.125}{\milli\meter}
\end{equation}
und für den Abstand zwischen Linse und Retina 
\begin{equation}
A_4=\frac{T_{\text{I}4}-T_{\text{I}3}}{2} c_\text{GK}=\SI{32.96}{\milli\meter}\text{.}
\end{equation}



