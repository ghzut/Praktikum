\section{Auswertung}
\label{sec:Auswertung}


Die Graphen wurden sowohl mit Matplotlib \cite{matplotlib} als auch NumPy \cite{numpy} erstellt. Die
Fehlerrechnung wurde mithilfe von Uncertainties \cite{uncertainties} durchgeführt.
Die Konstanten $k$, $\hslash$, $e_0$, $m_0$, $u_0$ und $N_\text{A}$ sind vom NIST \cite{nistgov}.

\subsection{Bestimmung der Schallgeschwindigkeit mit Durchschallungs-Verfahren}
\begin{table}
	\centering
	\caption{Die gemessenen Laufzeiten $T$ für die Acryl-Zylinder der Länge $l$.}
	\input{build/b.tex}
\end{table}
\begin{figure}
	\centering
	\caption{Die nach der Zeit $t$ zurückgelegte Strecke $x$ des Schalls im Zylinder gegen die Zeit $t$ aufgetragen.}
	\includegraphics[width=\linewidth-70pt,height=\textheight-70pt,keepaspectratio]{build/XgegenT2.pdf}
	\label{fig:XgegenT2}
\end{figure}
In der Abbildung \ref{fig:XgegenT2} sind die Messwertepaare aus Tabelle \ref{tab:b} aufgetragen und durch einen linearen Fit der Form $x=a t + b$ genähert worden. 


\subsection{Bestimmung der Schallgeschwindigkeit mit Impuls-Echo-Verfahren}
\begin{table}
	\centering
	\caption{Die gemessenen Laufzeiten $T$, Spannungen $U$ und TGC-Werte für die Acryl-Zylinder der Länge $l$.}
	\input{build/a.tex}
\end{table}

\subsection{Bestimmung der Dämpfungskonstanten mit Impuls-Echo-Verfahren}


\subsection{Untersuchung des Cepstrums/Spektrums}
\begin{table}
	\centering
	\caption{Die gemessenen Daten am dünnen Rohr und die zugehörigen Geschwindigkeiten, berechnet aus der Leistung.}
	\input{build/c.tex}
\end{table}
