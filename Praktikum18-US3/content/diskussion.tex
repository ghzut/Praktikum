
\section{Diskussion}
\label{sec:Diskussion}
Die Auswertung liefert einige Ergebnisse, welche nun noch zu diskutieren sind.
Die Konstante, welche aus den Graphen \ref{fig:k15},\ref{fig:k60},\ref{fig:k30},\ref{fig:m15},\ref{fig:m60},
\ref{fig:m30},\ref{fig:g15},\ref{fig:g60},\ref{fig:g30} entnommen wird ist zwar
bei allen Graphen hinreichend gleich, die zugehörigen Y Achsenabschnitte sind es
hingegen nicht. In der Theorie sollte es keinen zusätzlichen Y-Achsenabschnitt geben.
 Eine mögliche Ursache der Abschnitte sind die linearen Ausgleichsrechnungen,
welche auf wenigen Werten basieren. Es ist jedoch zu erkennen, dass die Y-Abschnitte
bei gleichen Rohrdurchmessern unter größeren Winkeln stärker abweichen. Ein weiterer
Grund ist die Nichtbeachtung der Schallgeschwindigkeiten in
den Rohrwänden. Nun folgt der zweite Versuchsteil. Die Tiefenmessungen der Geschwindigkeit in den Graphen \ref{fig:P45} und
\ref{fig:P70} zeigen Verschiebungen ihrer Peaks. Nach den Rohrdaten in \cite{US3}
 liegt die Mitte des Rohres in einer Tiefe von $\SI{38}{\milli\meter}$. Die Auswertung ergibt bei beiden
  Leistungen hingegen einen Peak bei ca. $\SI{35}{\milli\meter}$. Ein möglicher Grund
  hierfür könnte eine weitaus größere Geschwindigkeit der Welle in der Rohrwand sein.
  Alternativ sind die gegebenen Angaben vll. auch veraltet.
Auch können die abgelesenen Werte für die Frequenzverschiebung $\Delta f$ durch
die teilweise nicht ausgeprägten Peaks der Fourierzerlegung von den tatsächlichen abweichen.
Ein zusätzlicher Fehler bei beiden Messreihen während der Messung könnten nicht bemerkte Luftbläschen sein,
welche sich noch in der Apparatur befunden haben.





 %Ursache hierfür können Ungenauigkeiten in den verwendeten Daten
%der Rohrdurchmesser und der Schallgeschwindigkeiten. So wird für die Konstante
%$\frac{\Delta f}{\cos(\alpha)}$ nur die Schallgeschwindigkeit in der
%Dopplerflüssigkeit verwendet. In der Realität durchdringt die Welle jedoch auch
%das Material des Rohres, dessen $c$ nicht bekannt ist.
