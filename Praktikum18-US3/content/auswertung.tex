\section{Auswertung}
\label{sec:Auswertung}


Die Graphen wurden sowohl mit Matplotlib \cite{matplotlib} als auch NumPy \cite{numpy} erstellt. Die
Fehlerrechnung wurde mithilfe von Uncertainties \cite{uncertainties} durchgeführt.

\subsection{Die gemessenen Daten}

\begin{table}
 \centering
 \caption{Die gemessenen Daten am dünnen Rohr und die zugehörigen Geschwindigkeiten, berechnet aus der Leistung.}
 \input{build/tabkleinrohr.tex}
 \label{tab:k}
\end{table}

\begin{table}
 \centering
 \caption{Die gemessenen Daten am mittleren Rohr und die zugehörigen Geschwindigkeiten, berechnet aus der Leistung.}
 \input{build/tabmittelrohr.tex}
 \label{tab:m}
\end{table}

\begin{table}
 \centering
 \caption{Die gemessenen Daten am breiten Rohr und die zugehörigen Geschwindigkeiten, berechnet aus der Leistung.}
 \input{build/tabgrossrohr.tex}
 \label{tab:b}
\end{table}

\begin{table}
 \centering
 \caption{Die bestimmten Dopplerwinkel}
 \input{build/tabdopplerwinkel.tex}
 \label{tab:dopplerwinkel}
\end{table}

\subsection{Betrachtung der winkelabhängigen Frequenzverschiebung in Abhängigkeit der Strömungsgeschwindigkeit}

\begin{table}
 \centering
 \caption{Die bestimmten Steigungen der Graphen}
 \input{build/tabsteigungen.tex}
 \label{tab:steigungen}
\end{table}

\begin{figure}
 \centering
 \caption{Die Dopplerverschiebung bei einem Winkel von $\SI{15}{\degree}$ in einem dünnen Rohr.}
 \includegraphics[width=\linewidth-70pt,height=\textheight-70pt,keepaspectratio]{build/deltacosperV15.pdf}
 \label{fig:k15}
\end{figure}

\begin{figure}
 \centering
 \caption{Die Dopplerverschiebung bei einem Winkel von $\SI{60}{\degree}$ in einem dünnen Rohr.}
 \includegraphics[width=\linewidth-70pt,height=\textheight-70pt,keepaspectratio]{build/deltacosperV60.pdf}
 \label{fig:k60}
\end{figure}

\begin{figure}
 \centering
 \caption{Die Dopplerverschiebung bei einem Winkel von $\SI{30}{\degree}$ in einem dünnen Rohr.}
 \includegraphics[width=\linewidth-70pt,height=\textheight-70pt,keepaspectratio]{build/deltacosperV-30.pdf}
 \label{fig:k30}
\end{figure}

\begin{figure}
 \centering
 \caption{Die Dopplerverschiebung bei einem Winkel von $\SI{15}{\degree}$ in einem mittleren Rohr.}
 \includegraphics[width=\linewidth-70pt,height=\textheight-70pt,keepaspectratio]{build/deltacosperVmittel15.pdf}
 \label{fig:m15}
\end{figure}

\begin{figure}
 \centering
 \caption{Die Dopplerverschiebung bei einem Winkel von $\SI{60}{\degree}$ in einem mittleren Rohr.}
 \includegraphics[width=\linewidth-70pt,height=\textheight-70pt,keepaspectratio]{build/deltacosperVmittel60.pdf}
 \label{fig:m60}
\end{figure}

\begin{figure}
 \centering
 \caption{Die Dopplerverschiebung bei einem Winkel von $\SI{30}{\degree}$ in einem mittleren Rohr.}
 \includegraphics[width=\linewidth-70pt,height=\textheight-70pt,keepaspectratio]{build/deltacosperVmittel-30.pdf}
 \label{fig:m30}
\end{figure}

\begin{figure}
 \centering
 \caption{Die Dopplerverschiebung bei einem Winkel von $\SI{15}{\degree}$ in einem breiten Rohr.}
 \includegraphics[width=\linewidth-70pt,height=\textheight-70pt,keepaspectratio]{build/deltacosperVgross15.pdf}
 \label{fig:g15}
\end{figure}

\begin{figure}
 \centering
 \caption{Die Dopplerverschiebung bei einem Winkel von $\SI{60}{\degree}$ in einem breiten Rohr.}
 \includegraphics[width=\linewidth-70pt,height=\textheight-70pt,keepaspectratio]{build/deltacosperVgross60.pdf}
 \label{fig:g60}
\end{figure}

\begin{figure}
 \centering
 \caption{Die Dopplerverschiebung bei einem Winkel von $\SI{30}{\degree}$ in einem breiten Rohr.}
 \includegraphics[width=\linewidth-70pt,height=\textheight-70pt,keepaspectratio]{build/deltacosperVgross-30.pdf}
 \label{fig:g30}
\end{figure}

Zunächst wird das Verhältnis $\frac{\Delta f}{\cos(\alpha)}$ als Funktion der
Strömungsgeschwindigkeit des Rohres betrachtet. Für letztere wird aufgrund der Geräteangaben eine eine maximale
Pumpleistung von $\SI{10}{\litre\per\minute}$ angenommen. Die Rohrdurchmesser betragen
$\SI{7}{\milli\meter}$ für das dünne, $\SI{10}{\milli\meter}$ für das mittlere und
$\SI{16}{\milli\meter}$ für das breite Rohr. Mithilfe von
Formel \ref{eq:vmittel} folgen die auf den X-Achsen aufgetragenen Geschwindigkeiten der Tabellen \ref{tab:k},\ref{tab:m},\ref{tab:b}.
Die zu den Winkeln zugehörigen Dopplerwinkel ergeben sich aus Formel \ref{eq:dopplerwinkel}.
Zusammen mit den anderen Angaben der Y-Achse aus den Tabellen \ref{tab:k},\ref{tab:m},\ref{tab:b} und \ref{tab:dopplerwinkel} folgen die Graphen
\ref{fig:k15},\ref{fig:k60},\ref{fig:k30},\ref{fig:m15},\ref{fig:m60},\ref{fig:m30},\ref{fig:g15},\ref{fig:g60},\ref{fig:g30}.
Es ist zu erkennen, dass die Strömungsgeschwindigkeit vom Rohrdurchmesser abhängig
ist und steigt, wenn der Durchmesser kleiner wird. Mithilfe einer Ausgleichsrechnung
der Form $y=ax+b$ folgen die Parameter in Tabelle \ref{tab:steigungen}. Diese
sind trotz unterschiedlicher Winkel und Rohrdurchmesser alle ca. gleich groß und
liegen im Mittel bei $\SI{21.6(8)}{}$. Ein Vergleich mit der theoretischen
Konstante $\frac{2*f_0}{c}$, welche sich aus der Formel der theoretischen
Frequenzverschiebung \ref{eq:deltaf} ergibt, zeigt deutliche Übereinstimmungen, wenn für $c$
die Schallgeschwindigkeit $c = \SI{1800}{\meter\per\second}$ der Dopplerphantomflüssigkeit
eingesetzt wird.



\subsection{Betrachtung der Geschwindigkeit und der Streuintensität in Abhängigkeit der Messtiefe}

\begin{figure}
 \centering
 \caption{Das Strömungsprofil bei $P = \SI{45}{\percent}$.}
 \includegraphics[width=\linewidth-70pt,height=\textheight-70pt,keepaspectratio]{build/messtiefe45.pdf}
 \label{fig:P45}
\end{figure}

\begin{figure}
 \centering
 \caption{Das Strömungsprofil bei $P = \SI{70}{\percent}$.}
 \includegraphics[width=\linewidth-70pt,height=\textheight-70pt,keepaspectratio]{build/messtiefe70.pdf}
 \label{fig:P70}
\end{figure}

\begin{table}
 \centering
 \caption{Die gemessen Daten zum Messprofil bei $P = \SI{45}{\percent}$.}
 \input{build/tabmtief45.tex}
 \label{tab:mtief45}
\end{table}

\begin{table}
 \centering
 \caption{Die gemessen Daten zum Messprofil bei $P = \SI{70}{\percent}$.}
 \input{build/tabmtief70.tex}
 \label{tab:mtief70}
\end{table}

Nun wird nun das Strömungsprofil der Dopplerflüssigkeit am Schlauch mit mittlerem
Durchmesser unter einem Winkel von $\SI{15}{\degree}$ untersucht. Da die Streuintensitäten ungleich größer als die Geschwindigkeiten
sind,wird nur $\SI{0.2}{\percent}$ der Wertehöhe in den Graphen \ref{fig:P45} und \ref{fig:70} dargestellt. Die
dargestellten Geschwindigkeiten in den Tabellen \ref{tab:mtief45} und \ref{tab:mtief70} folgen aus Umstellung von Formel \ref{eq:deltaf}.
 In beiden Graphen lässt sich das Strömungsprofil in Form des Hochpunktes erkennen. Dieses
 entsteht, da die Geschwindigkeit in Richtung Rohrzentrum ansteigt. Den Graphen nach liegt
 der breiteste Querschnitt des Rohres bei einer Messtiefe von ca. $\SI{35}{\milli\meter}$.
 Anschließend zeigen die Graphen die darauf folgende Verengung des Rohrquerschnitts
 und den somit resultierenden Abfall der Kurve. Da die Viskosität der Flüssigkeit so
 gewählt ist, dass sich laminare Strömungen bilden, sind die Muster der Streuintensität
 bei beiden Leistungen sehr gleichförmig.
