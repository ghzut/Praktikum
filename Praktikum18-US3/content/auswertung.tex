\section{Auswertung}
\label{sec:Auswertung}


Die Graphen wurden sowohl mit Matplotlib \cite{matplotlib} als auch NumPy \cite{numpy} erstellt. Die
Fehlerrechnung wurde mithilfe von Uncertainties \cite{uncertainties} durchgeführt.

\subsection{Die gemessenen Daten}

\begin{table}
 \centering
 \caption{Die gemessenen Daten am dünnen Rohr und die zugehörigen Geschwindigkeiten, berechnet aus der Leistung.}
 \input{build/tabkleinrohr.tex}
 \label{tab:k}
\end{table}

\begin{table}
 \centering
 \caption{Die gemessenen Daten am mittleren Rohr und die zugehörigen Geschwindigkeiten, berechnet aus der Leistung.}
 \input{build/tabmittelrohr.tex}
 \label{tab:m}
\end{table}

\begin{table}
 \centering
 \caption{Die gemessenen Daten am breiten Rohr und die zugehörigen Geschwindigkeiten, berechnet aus der Leistung.}
 \input{build/tabgrossrohr.tex}
 \label{tab:b}
\end{table}

\subsection{Betrachtung der winkelabhängigen Frequenzverschiebung in Abhängigkeit der Strömungsgeschwindigkeit}


\begin{figure}
 \centering
 \caption{Die Dopplerverschiebung bei einem Winkel von $\SI{15}{\degree}$ in einem dünnen Rohr.}
 \includegraphics[width=\linewidth-70pt,height=\textheight-70pt,keepaspectratio]{build/deltacosperV15.pdf}
 \label{fig:k15}
\end{figure}

\begin{figure}
 \centering
 \caption{Die Dopplerverschiebung bei einem Winkel von $\SI{60}{\degree}$ in einem dünnen Rohr.}
 \includegraphics[width=\linewidth-70pt,height=\textheight-70pt,keepaspectratio]{build/deltacosperV60.pdf}
 \label{fig:k60}
\end{figure}

\begin{figure}
 \centering
 \caption{Die Dopplerverschiebung bei einem Winkel von $\SI{30}{\degree}$ in einem dünnen Rohr.}
 \includegraphics[width=\linewidth-70pt,height=\textheight-70pt,keepaspectratio]{build/deltacosperV-30.pdf}
 \label{fig:k30}
\end{figure}

\begin{figure}
 \centering
 \caption{Die Dopplerverschiebung bei einem Winkel von $\SI{15}{\degree}$ in einem mittleren Rohr.}
 \includegraphics[width=\linewidth-70pt,height=\textheight-70pt,keepaspectratio]{build/deltacosperVmittel15.pdf}
 \label{fig:m15}
\end{figure}

\begin{figure}
 \centering
 \caption{Die Dopplerverschiebung bei einem Winkel von $\SI{60}{\degree}$ in einem mittleren Rohr.}
 \includegraphics[width=\linewidth-70pt,height=\textheight-70pt,keepaspectratio]{build/deltacosperVmittel60.pdf}
 \label{fig:m60}
\end{figure}

\begin{figure}
 \centering
 \caption{Die Dopplerverschiebung bei einem Winkel von $\SI{30}{\degree}$ in einem mittleren Rohr.}
 \includegraphics[width=\linewidth-70pt,height=\textheight-70pt,keepaspectratio]{build/deltacosperVmittel-30.pdf}
 \label{fig:m30}
\end{figure}

\begin{figure}
 \centering
 \caption{Die Dopplerverschiebung bei einem Winkel von $\SI{15}{\degree}$ in einem breiten Rohr.}
 \includegraphics[width=\linewidth-70pt,height=\textheight-70pt,keepaspectratio]{build/deltacosperVgross15.pdf}
 \label{fig:g15}
\end{figure}

\begin{figure}
 \centering
 \caption{Die Dopplerverschiebung bei einem Winkel von $\SI{60}{\degree}$ in einem breiten Rohr.}
 \includegraphics[width=\linewidth-70pt,height=\textheight-70pt,keepaspectratio]{build/deltacosperVgross60.pdf}
 \label{fig:g60}
\end{figure}

\begin{figure}
 \centering
 \caption{Die Dopplerverschiebung bei einem Winkel von $\SI{30}{\degree}$ in einem breiten Rohr.}
 \includegraphics[width=\linewidth-70pt,height=\textheight-70pt,keepaspectratio]{build/deltacosperVgross-30.pdf}
 \label{fig:g30}
\end{figure}

Zunächst wird das Verhältnis $\frac{\Delta f}{\cos(\alpha)}$ als Funktion der
Strömungsgeschwindigkeit des Rohres betrachtet. Für letztere wird aufgrund der Geräteangaben eine eine maximale
Pumpleistung von $\SI{10}{\litre\per\minute}$ angenommen. Mithilfe der Rohrinnendurchmesser aus Tab. \ref{tab:HJKH} und
Formel \ref{eq:HHH} folgen die auf den X-Achsen aufgetragenen Geschwindigkeiten der Tabellen \ref{tab:k},\ref{tab:m},\ref{tab:b}.
Zusammen mit den anderen Angaben der Y-Achse aus den Tabellen \ref{tab:k},\ref{tab:m},\ref{tab:b} folgen die Graphen
\ref{fig:k15},\ref{fig:k60},\ref{fig:k30},\ref{fig:m15},\ref{fig:m60},\ref{fig:m30},\ref{fig:g15},\ref{fig:g60},\ref{fig:g30}.
Es ist zu erkennen, dass die Strömungsgeschwindigkeit vom Rohrdurchmesser abhängig
ist und steigt, wenn der Durchmesser kleiner wird. Mithilfe einer Ausgleichsrechnung
der Form $y=ax+b$ folgen die Parameter in Tabelle \ref{tab:jhhjkh}. Diese
sind trotz unterschiedlicher Winkel und Rohrdurchmesser alle ca. gleich groß und
liegen im Mittel bei $\SI{21.6(8)}{}$. Ein Vergleich mit der theoretischen
Konstante $\frac{2*f_0}{c}$, welche sich aus der Formel der theoretischen
Frequenzverschiebung \ref{eq:deltaf} ergibt, zeigt deutliche Übereinstimmungen, wenn für $c$
die Schallgeschwindigkeit $c = \SI{1800}{\meter\per\second}$ der Dopplerphantomflüssigkeit
eingesetzt wird.
