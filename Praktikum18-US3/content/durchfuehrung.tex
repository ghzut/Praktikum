
\section{Durchführung}
\label{sec:Durchführung}
Es werden die drei Doppler-Prismen an die jeweiligen dafür vorgesehenen Rohren angebracht. Damit eine gute Übertragung der Schallwellen gewährleistet ist muss zwischen den Rohren und Prismen Ultraschallgel aufgetragen werden.  Nun wird die Zentrifugalpumpe auf eine Leistung von maximal $\SI{70}{\percent}$ eingestellt. Jetzt wird jeweils an den Prismen mit den drei verschieden Winkeln mit der Ultraschallsonde gemessen. Hierbei muss auch das Ultraschallgel zwischen der Sonde und dem Prisma aufgetragen werden. Es ist auch darauf zu achten, dass an dem Ultraschallgenerator die Frequenz auf $\SI{2}{\mega\hertz}$ und das SAMPLE VOLUME auf LARGE gestellt ist. Es wird jeweils die Leistung der Pumpe sowie die gemessene Frequenzverschiebung notiert. Diese Messung wird für fünf verschiedene Pumpleistungen durchgeführt.

Nach der ersten Messreihe wird nun das SAMPLE VOLUME auf SMALL und der Regler DEPTH auf $\SI{12}{\micro\second}$ gestellt. Nun wird in Abständen von $\SI{0.5}{\micro\second}$ bis $\SI{19.5}{\micro\second}$ am Rohr mit dem mittleren Durchmesser bei einem Winkel von $\SI{15}{\degree}$ gemessen. Dabei wird sowohl eine Pumpleistung von $\SI{45}{\percent}$ als auch eine von $\SI{70}{\percent}$ verwendet. Es werden die jeweiligen Frequenzverschiebungen als auch die Streuintensitäten notiert.