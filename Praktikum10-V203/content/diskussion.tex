
\section{Diskussion}
\label{sec:Diskussion}
Zunächst wird die Messreihe im Druckbereich von unter $\SI{1}{\bar}$
 behandelt. Bereits in Abbildung \ref{fig:Graph1} ist zu erkennen, dass deren Messwerte im
  rechten Bereich, im Niedrigdruckbereich, starke Abweichungen zu einem einer Geraden zeigen. Eine mögliche Ursache hierfür können
  undichte Stellen in der Apparatur sein sein, aufgrund dessen es zu einem
   Druckausgleich kommt. Dieser Effekt macht sich am stärksten im
   Niedrigdruckbereich bemerkbar. In der nachfolgende linearen Ausgleichsrechnung
    werden die ersten 14 Werte daher ausgeschlossen. Anhand von Abb.
     \ref{fig:Graph2} zeigt sich, dass sich die Verdampfungswärme im betrachteten
      Temperaturbereich gut durch einen linearen Zusammenhang nähern lässt. Es zeigt sich jedoch, dass der
      ermittelte Mittelwert der Verdampfungswärme von $\SI{34.3}{\kilo\joule\per\mol}$ unterhalb der Verdampfungswärme bei einer
       Temperatur von $\SI{102.32}{\degreeCelsius}$ von $\SI{40.51}{\kilo\joule\per\mol}$ \cite{WasserTab} liegt. Da die benötigte
       Verdampfungswärme den Literaturwerten in Abb. \ref{tab:tab3} nach  unter geringeren Temperaturen zunimmt, liegt der Wert
        daher unterhalb des zu Erwartenden. Die Messkurven beider
          Messreihen sollten im Idealfall in Abbildung \ref{fig:Graph1} ohne Auffälligkeiten ineinander übergehen.
           Es ist jedoch ein Knick zu erkennen und die weiteren Messwertepaare lassen vermuten, dass die zweite Messreihe um ca.
           $\SI{0.4}{\per\milli\kelvin}$ zur ersten verschoben ist. Dies lässt auf einen systematischen Fehler schließen. Es ist zu erwarten, dass eine mögliche
              Fehlerquelle im Aufbau des zweiten Messung existiert. Aufgrund der
               uneinsichtigen Bauweise kann diese jedoch nicht konkretisiert werden.
               Zusätzlich ist bei der Datenübertragung ein Wert verlorengegangen.
               Eine mögliche Stelle  von diesem ist zwar lokalisiert worden, ob diese jedoch die richtige ist, ist unsicher.
                Die Abweichungen sind jedoch nicht allein hierdurch zu erklären.
                 Die erstellte Näherung der Funktion der Verdampfungswärme in Abb. \ref{fig:Graph4} zeigt
                  deutliche Abweichungen bezüglich der Literaturwerte nach \ref{tab:tab3} und stellt deren
                   Verlauf zu keinem Temperaturpunkt hinreichend gut dar. Eine
                   Ursache dieses Verhaltens ist der Näherung durch Polynome geschuldet.
                    Ein weiterer sind die Abweichungen der Messkurve, deren
                     Gewichtung durch das verwendete Verfahren noch verstärkt wird. 
