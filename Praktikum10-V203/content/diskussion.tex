
\section{Diskussion}
\label{sec:Diskussion}
Zunächst wird die Messkurven der $P$,$T$ Aufnahme im druckbereich von unter $\SI{1}{\bar}$
 behandelt. Bereits in \ref{fig:Graph1} ist zu erkennen, dass deren Messwerte im
  rechten Bereich, im Niedrigdruckbereich, starke Abweichungen zu einem einer
  idealisierten Gerade zeigen. Eine mögliche Hauptursache hierfür können
  undichte Stellen in der Apperatur sein sein, aufgrunddessen es zu einem
   Druckausgleich kommt. Dieser Effekt macht sich am stärksten im
   Niedrigdruckbereich bemerkbar. In der nachfolgende linearen Ausgleichsrechnung
    werden die ersten 14 Werte daher ausgeschlossen. Anhand von Abb.
     \ref{fig:Graph2} zeigt sich, dass sich die Verdampfungwärme im betrachteten
      Temperaturbereich gut durch einen linearen Zusammenhang nähern lässt. Im
      Vergleich mit Literaturwerten zu Verdampfungwärmen zeigt sich jedoch, das der
      ermittelte Mittelwert unterhalb der Verdampfungswärme bei einer
       Temperatur von $\SI{100}{\degree\Celsius}$ liegt. Da die nötige
       Verdampfungswärme unter niedrigeren temperaturen zunimmt, liegt der Wert
        daher unterhalb des Erwartbaren.
