
\section{Durchführung}
\label{sec:Durchführung}

\renewcommand{\labelenumi}{\alph{enumi})}
\begin{enumerate}
  \item Zunächst wird die Druckkurve im Bereich von unter $\SI{1}{\bar}$ aufgenommen.
   Es wird der Aufbau nach Abb. \ref{fig:Aufbau1} verwendet.
  Hierzu wird der Mehrhalskolben als erstes solange evakuiert bis ein
  Druck von ca. $\SI{30}{\milli\bar}$ anzeigt wird. Zur Evakuierung werden der
   Absperrhahn und das Drosselventil geöffnet, das Belüftungsventil geschlossen.
    Die Wasserstrahlpumpe wird geöffnet.
    Ist der gewünschte Druck erreicht, werden zunächst wieder der Absperrhahn,
     danach auch Drosselventil und Wasserstrahlpumpe geschlossen. Gleichzeitig
     werden die Wasserkühlung und die Heizhaube eingeschaltet. Die Energieversorgung
     letzterer ist auf $\SI{60}{\percent}$ Leistung eingestellt. Instantan wird mit
      der Messung begonnen. Es wird der vorherrschende Druck $P$ in Abhängigkeit der Temperatur
       $T$ in Temperaturabständen von ca. $0,5- \SI{1}{\degreeCelsius}$ aufgenommen.
       Im Verlaufe des Temperaturanstiegs muss die Rückflusskühlung immer weiter
        verringert um die nötigen Temperaturen zu erreichen. Es werden $P$ und $T$
         Werte im Druckbereich von $30- \SI{1000}{\milli\bar}$ aufgenommen.

         \item Anschließend wird die Druckkurve im Bereich von über $\SI{1}{\bar}$
          aufgenommen. Hierzu wird der Aufbau nach Abb. \ref{fig:Aufbau2} verwendet. Die
           elektrische Energieversorgung wird eingeschaltet und auf ihr Leistungsmaximum
            eingestellt. Auch diesmal wird $P$ in Abhängigkeit von $T$ aufgenommen.
             Die Messung wird gestoppt, sobald entweder eine Temperatur von $\SI{200}{\degreeCelsius}$
              oder ein Druck von $\SI{15}{\bar}$ erreicht sind.
\end{enumerate}
