\section{Auswertung}
\label{sec:Auswertung}
Die Graphen in Abbildung \ref{fig:Graph1}, \ref{fig:Graph2} und \ref{fig:Graph3} wurden sowohl mit Matplotlib \cite{matplotlib} als auch NumPy \cite{numpy} erstellt. Die Fehlerrechnung wurde mit Unterstützung von Uncertainties \cite{uncertainties} durchgeführt.
\subsection{Bestimmung des durchschnittlichen Wertes für die Verdampfungswärme für einen Druck von unter einen Bar mithilfe einer Ausgleichsrechnung}
\begin{figure}
	\centering
	\caption{Der Logarithmus des Dampfdruckes $p$ gegen die reziproke absolute Temperatur $T^{-1}$ aufgetragen.}
	\includegraphics[width=\linewidth-70pt,height=\textheight-70pt,keepaspectratio]{build/logpgegen1durchT.pdf}
	\label{fig:Graph1}
\end{figure}
\begin{center}
	\begin{table}
		\caption{Die gemessene Temperatur $T$ und der zugehörige Dampfdruck $p$ aus der Messung im Bereich von $30-\SI{1000}{\milli\bar}$.}
		\begin{minipage}[t]{0.24\textwidth}
			\centering
			\input{build/tab11.tex}
		\end{minipage}
		\begin{minipage}[t]{0.24\textwidth}
			\centering
			\input{build/tab12.tex}
		\end{minipage}
		\begin{minipage}[t]{0.24\textwidth}
			\centering
			\input{build/tab13.tex}
		\end{minipage}
		\begin{minipage}[t]{0.24\textwidth}
			\centering
			\input{build/tab14.tex}
		\end{minipage}
	\end{table}
	\begin{table}
		\caption{Die gemessene Temperatur $T$ und der zugehörige Dampfdruck $p$ aus der Messung im Bereich von $1-\SI{8}{\bar}$.}
		\begin{minipage}[t]{0.5\textwidth}
			\centering
			\input{build/tab21.tex}
		\end{minipage}
		\begin{minipage}[t]{0.5\textwidth}
			\centering
			\input{build/tab22.tex}
		\end{minipage}
	\end{table}
\end{center}

\begin{figure}
	\centering
	\caption{Der Logarithmus des Dampfdruckes $p$ gegen die reziproke absolute Temperatur $T^{-1}$ aufgetragen mit den Messwerten aus der Messreihe für $p\leq \SI{1}{\bar}$ ohne die ersten 14 Messwertepaare und deren linearer Fit.}
	\includegraphics[width=\linewidth-70pt,height=\textheight-70pt,keepaspectratio]{build/logpgegen1durchTausgleich.pdf}
	\label{fig:Graph2}
\end{figure}
Der Fit in Abbildung \ref{fig:Graph2} besitzt die Form $y=a x + b$. Eine lineare Ausgleichsrechnung der Form $y=a x + b$ mittels SciPy \cite{scipy} liefert mit den Wertepaaren aus Tabelle \ref{tab:tab11} und mit $R=\SI{8.3144598(48)}{\joule\per\mol\kelvin}$ \cite{R} nach Formel \eqref{eq:DGLLs}:
\begin{displaymath}
L = -a R = \SI{3.43(2)e4}{\joule\per\mol}\text{.}
\end{displaymath}

\subsection{Berechnung der benötigten Energie zur Überwindung der molekularen Anziehungskräfte für ein Wassermolekül bei 373 Kelvin}
Nach der allgemeinen Gasgleichung gilt:
\begin{equation}
p V = R T = W \text{,}\label{eq:Gas}
\end{equation}
wobei $W$ eine Energie ist.
Für die Energie um ein Mol von $V_1$ auf $V_2$ isobar auszudehnen folgt:
\begin{equation}
W_\text{1,2} = p (V_2-V_1)\text{.}
\end{equation}
Da bei einer Temperatur von $T=\SI{373}{\kelvin}$ bei Wasser $V_\text{F}\ll V_\text{D}$ gilt, folgt:
\begin{equation}
	L_\text{a} = W_\text{F,D} \approx p V_D = R T = \SI{3101.294(2)}{\joule\per\mol}
\end{equation} 
um ein Mol von $V_\text{F}$ auf $V_\text{D}$ auszudehnen.
Für die benötigte Energie um die molekularen Anziehungskräfte zu überwinden gilt:
\begin{equation}
	L_\text{i}= L - L_\text{a} = \SI{3.12(2)e4}{\joule\per\mol}\text{.}
\end{equation}
Für ein Molekül ergibt sich mit $n=\SI{6.022140857(74)e23}{\per\mol}$ \cite{n} und $\SI{1}{\electronvolt}=\SI{1.6021766208(98)e-19}{\joule}$ \cite{eV}:
\begin{equation}
	\frac{L_\text{i}}{n} = \SI{0.323(3)}{\electronvolt}\text{.}
\end{equation}

\subsection{Näherung der Funktion von der Verdampfungswärme}

\begin{figure}
\centering
\caption{Der Dampfdruck $p$ gegen die absolute Temperatur $T$ aufgetragen mit den Messwerten aus der Messreihe für $p>\SI{1}{\bar}$ und deren Ausgleichspolynom vom 6. Grad.}
\includegraphics[width=\linewidth-70pt,height=\textheight-70pt,keepaspectratio]{build/pgegenTausgleich.pdf}
\label{fig:Graph3}
\end{figure}
Durch Umformen der Formel \eqref{eq:DGL} ergibt sich:
\begin{equation}
	L=(V_\text{D}-V_\text{F})\cdot T\cdot \frac{\text{d} p}{\text{d} T}\text{.}\label{eq:L}
\end{equation}
Da weiterhin $V_\text{F}\ll V_\text{D}$ gilt, kann $V_\text{F}$ vernachlässigt werden. Nach den Hinweisen zur Auswertung des Versuches V203 \cite{V203} kann $V_\text{D}$ durch folgende Formel besser genähert werden:
\begin{equation}
	\left( p + \frac{a}{V^2}\right) V = R T \text{   mit }a=\SI{0.9}{\joule\meter\tothe{3}\per\mol\squared}\text{.}
\end{equation}
Umgestellt nimmt diese folgende Gestalt an:
\begin{equation}
	V_\text{D} \approx \frac{R T}{2p} \pm \sqrt{\left(\frac{R T}{2p}\right)^2-\frac{a}{p}}\text{   mit }a=\SI{0.9}{\joule\meter\tothe{3}\per\mol\squared}\label{eq:VD}\text{,}
\end{equation}
wobei die Lösung mit der positiven Wurzel die physikalisch sinnvolle ist, da diese sich ähnlich zu der Beziehung für $V_\text{D}$ aus der allgemeinen Gasgleichung verhält und die andere Lösung nicht.
Das Ausgleichspolynom 6. Grades in Abbildung \ref{fig:Graph3} wurde mir SciPy \cite{scipy} ermittelt und besitzt die Form:
\begin{equation}
\begin{aligned}
p \approx &- \SI{9e-7}{\pascal\per\kelvin\tothe{6}} \cdot T^6 + \SI{2.2e-3}{\pascal\per\kelvin\tothe{5}} \cdot T^5 - \SI{2.1}{\pascal\per\kelvin\tothe{4}}\cdot T^4 + \SI{1.1e3}{\pascal\per\kelvin\tothe{3}} \cdot T^3 \\
&- \SI{3.1e5}{\pascal\per\kelvin\tothe{2}} \cdot T^2 + \SI{4.8e7}{\pascal\per\kelvin} \cdot T - \SI{3.0e9}{\pascal}\text{.}
\end{aligned}\label{eq:p}
\end{equation}
Damit folgt für die Ableitung nach $T$:
\begin{equation}
\begin{aligned}
\frac{\text{d} p}{\text{d} T} \approx &- \SI{5.6e-6}{\pascal\per\kelvin\tothe{5}} \cdot T^5 + \SI{1.1e-2}{\pascal\per\kelvin\tothe{4}}\cdot T^4 - \SI{9}{\pascal\per\kelvin\tothe{3}} \cdot T^3 \\
&+ \SI{3.3e3}{\pascal\per\kelvin\tothe{2}} \cdot T^2 - \SI{6.3e5}{\pascal\per\kelvin} \cdot T + \SI{4.8e7}{\pascal}\text{.}
\end{aligned}\label{eq:dpdT}
\end{equation}
Somit kann nach \eqref{eq:L} und \eqref{eq:VD} die Verdampfungswärme folgendermaßen genähert werden:
\begin{displaymath}
	\begin{aligned}
		L \approx \left(\frac{R T}{2 p} + \sqrt{\left(\frac{R T}{2p}\right)^2-\frac{a}{p}}\right) T \frac{\text{d} p}{\text{d} T}\text{   mit }a=\SI{0.9}{\joule\meter\tothe{3}\per\mol\squared}
	\end{aligned}\text{,}
\end{displaymath}
wobei für $p$ die Näherung \eqref{eq:p} und für $\frac{\text{d} p}{\text{d} T}$ die Näherung \eqref{eq:dpdT} verwendet wird.
\begin{figure}
	\centering
	\caption{Die genäherte Funktion $L(T)$ und die Literaturwerte aus Tabelle \ref{tab:tab3} in ein $TL$-Diagramm eingetragen.}
	\includegraphics[width=\linewidth-70pt,height=\textheight-70pt,keepaspectratio]{build/LgegenT.pdf}
	\label{fig:Graph4}
\end{figure}
\begin{table}
	\caption{Die Literaturwerte der molaren Verdampfungswärme $L$ mit den zugehörigen Temperaturen $T$ \cite{LWasser}.}
	\centering
	\input{build/tab3.tex}
\end{table}



