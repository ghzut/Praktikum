\section{Aufbau}
Zunächst folgt ein Versuchsaufbau zur Aufnahme einer Dampfdruckkurve im Druckbereich
 von unter $\SI{1}{\bar}$. Hierzu wird eine Apperatur der Form aus Abb. JJJJHJH verwendet.
   Im Zentrum befindet sich ein Mehrhalskolben, welcher mit etwas Wasser gefüllt ist.
    In den äußeren Hälsen befinden sich zwei Flüssigkeitsthermometer, eines um die
     Temperatur des Wassers zu messen und eines für den Raum oberhalb der
      Wasseroberfläche. Das Wasser kann über eine Heizhaube, welche über einen
       Trenntrafo mit Strom versorgt wird, erhitzt werden. Um die Glasteile
        zu kühlen wird ein Rückflusskühler oberhalb des Kolbens angebracht, welcher
         ständig von etwas Wasser durchflossen wird. Um den Kolben zu
          evakuieren wird eine oberhalb der Kolben mit einer Wollfschen Flasche
           verbunden und über ein Drosselventil getrennt. Die flasche besitzt
           zusätzlich ein belüftungsventil sowie eine verschließbare Verbindung
            mit einer Wasserstrahlpumpe.

            Um im Anschluss auch eine Dampfdruckkurve im Druckbereich von über
             $\SI{1}{\bar}$ aufzunehmen wird ein Aufbau nach Abb. JKJKJJ verwendet.
              Dieser besteht aus einem bereits evakuierten, mit etwas Wasser
               gefüllten Gefäß, welches von einer Heizwicklung umschlossen ist.
                Diese wird über eine elektrische Energieversorung erwärmt. Aus
                 Sicherheitsgründen ist das Gefäß mit einem Metallschutz ummantelt
