\section{Theorie}
\label{sec:Theorie}
Die Theorie kann verglichen werden mit der Anleitung zum Versuch Nr. 353 \cite{V353}.

\subsection{Der LC-Kreis}
Der LC-Kreis gehört zu den ungedämpften Schwingkreisen.Unter diesen versteht man
 ein System mit zwei Energiespeichern zwischen denen eine vorher eingeführte Energiemenge
  oszilliert. Bei einem LC-Kreis werden die Speicher durch eine Kapazität C sowie
  eine Induktivität L realisiert. Da im theoretischen LC-Kreis kein Bauteil mit Energieverbrauch
   existiert, bleibt die Oszillation zeitlich erhalten.
   %Bild von LC Kreis einfügen
   \subsection{Der RCL-Kreis}
Eine Erweiterung der Konstruktion besteht im RCL-Kreis, in dem eine gewisse Energierate
kontinuirlich über einen ohmschen Widerstand R in Wärme umgesetzt wird. Aufgrunddessen bilden
U(t) und I(t)  monoton fallende Funktionen der Zeit. Es wird von einer gedämpften
 Schwingung gesprochen.
 %Abbildung 2 einfügen
 Für diese gilt nach den kirchhoffschen Regeln die DGL:
 \begin{equation}
   \ddot{I} + \frac{R}{L} \dot{I} + \frac{1}{LC}I = 0
 \end{equation}
 Mit einem komplesen E-Ansatz erhält man die Winkelgeschwindigkeit $\omega$
 \begin{equation}
   \omega = sqrt{\frac{1}{LC}-\frac{R²}{4L²}}
 \end{equation}
 Sowie der Parameter der Amplitudenabnahme $\mu$:
 \mu = \frac{R}{2L}
 Hiermit gelangt man zur allgemeinen LÖsung:
 \begin{equation}
   F(t) = e^{-\mu t} \cdot \left A_1e^{i\omega_1t} + A_2e^{-i\omega_2t}\right A_1,A_2 \in \C
 \end{equation}
 Wie die Lösungsfunktion nun genau aussieht, hängt davon ab ob $\omega$ reell, imaginär
  oder gleich 0 ist.
  Ist $\omega$ reell liegt eine gedämpfte Schwingung vor und für I(t) gitl:
  \begin{equation}
    I(t) = A_0 e^{-\mu t} \cdot cos(\omega t + \varphi)
  \end{equation}
  der Periodendauer T:
  \begin{equation}
    T = \frac{2 \pi}{\omega}
  \end{equation}
  sowie der Ablinkdauer der Amplituden \tau:
  \begin{equation}
    \tau = \frac{2L}{R}
  \end{equation}

Ist $\omega$ hingegen imaginär liegt keine Schwingung mehr vor und es kommt zu einer
aperiodischen Dämpfung, welche nach hinreichender Zeit ein einfaches Relaxationsverhalten
 aufzeigt. Je nachdem wie $A_1$ und $A_2$ bestimmt sind kann I(t) jedoch vorher noch einen
 Maximalwert durchlaufen.
 %abb 4 einfügen
 Beträgt $\omega$ jedoch 0 liegt der aperiodische Grenzfall vor, bei welchem I(t)
   sofort und am schnellsten gegen 0 verläuft.

\subsection{ Der RCL-Kreis unter Einfluss einer erzwungenen Schwingung}

%Wird der gedämpfte RCL-Schwingkreis einer periodischen Kraft unterworfen


Für die DGl des Schwingkreises folgt dann:
\begin{equation}
  LC \ddot{U_C} + RC \dot{U_C} + U_C = U_0 \cdot e^{i\omega t}
\end{equation}
Mit einem komplexen e-Ansatz für $U_C$ erhält man für den Betrag der zugehörigen $A_C(\omega)$
\begin{equation}
  A_C{\omega} = \frac{U_0}{sqrt{(1-LC\omega^2)^2 + \omega²R²C²}}
\end{equation}
\begin{equation}
  %|A(\omega)| = U_0 \vdot sqrt{\frac{(1-LC\omega^2)² + \omega²R²C²}{\left \left 1-LC\omega²\right ^2 + \omega²R²C²\right^2}}
\end{equation}
Zusätzlich zeichnet sich auch eine Phasenverschiebung zwischen $U_C$ und $U_0$ ab, ähnlich dem in V 353 aufgetretenen Effekt.
Für diese gilt:
\begin{equation}
  \varphi(\omega) = arctan\left \frac{-\omega RC}{1-LC \omega^2}\right
\end{equation}



\subsection{Resonanz und Güte eines RCL-Kreises}
Die Amplitude der Kondensatorspannung ist daher abhängig von der Frequenz der Generatorspannung.
Für $\omega \to \infty$ läuft $\U_C$ gegen 0, für $\omega \to 0$ gegen $U_0$.
Für eine Frequenz $f_{res}$ erreicht $U_C$ ein Maximum, welches größer als $U_0$ selbst ist.
In diesem Fall wird von der Resonanzfrequenz des Systems gesprochen. Für $f_{res}$ folgt:
\begin{equation}
  f_{res} = \frac{sqrt{\frac{1}{LC}-{R²}{2L²}}{2 \pi}}
\end{equation}
Für den Fall einer schwachen Dämpfung verläuft $A_{C,max}$ mit:
\begin{equation}
  A_{C,max} = \frac{U_0}{\omega_0 RC}
  %A_{C,max} = \frac{U_0}{R} \cdot \sqrt{\frac{L}{C}}
\end{equation}
Verschwindet R nun völlig, läuft das Maximum gegen Unendlich und es kommt zu einer
Resonanzkatastrophe. Der Faktor $\frac{1}{\omega_0 RC}$ wird als Güte bezeichnet.
Letzterer lässt sich auch mithilfe von
\begin{equation}
  q = \frac{\omega_0}{\omega_+ - \omega_-}
  \end{equation}
bestimmen. $\omega_+$ und $\omega_-$ bilden die Stellen links und rechts neben dem Maximum,
 an denen $A_C$ $1/sqrt{2}$ von $A_{C,max}$ erreicht.

 Im Falle einer starken Dämpfung existiert keine Resonanzfrequenz, $A_C$ fällt bei steigender Frequenz
 , von $U_0$ ausgehend, monoton gegen 0. Ist die Frequenz ausreichend hoch, zeichnet
  sich eine $\frac{1}{\omega²}$ Abhängigkeit ab.
