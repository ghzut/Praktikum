\section{Theorie}
\label{sec:Theorie}


\subsection{Der LC-Kreis}
Der LC-Kreis gehört zu den ungedämpften Schwingkreisen.Unter diesen versteht man
 ein System mit zwei Energiespeichern zwischen denen eine vorher eingeführte Energiemenge
  oszilliert. Bei einem LC-Kreis sind die Speicher durch eine Kapazität C sowie
  eine Induktivität L realisiert. Da im theoretischen LC-Kreis kein Bauteil mit Energieverbrauch
   existiert, bleiben Oszillation und Anfangsamplitude zeitlich erhalten.

   \begin{figure}[H]
     \centering
     \includegraphics[width=\linewidth-300pt,height=\textheight-300pt,keepaspectratio]{content/CL.png}
     \caption{LC-Kreis}
     \label{fig:CL_Kreis}
   \end{figure}

   \subsection{Der RCL-Kreis}
Eine Erweiterung des Systems ist der RCL-Kreis, in dem eine gewisse Energierate
kontinuierlich über einen ohmschen Widerstand R in Wärme umgesetzt wird. Als Folge sind
U(t) und I(t)  monoton fallende Funktionen der Zeit. Es handelt sich daher um eine gedämpfte
 Schwingung.

 \begin{figure}[H]
   \centering
   \includegraphics[width=\linewidth-200pt,height=\textheight-200pt,keepaspectratio]{content/RCL.png}
   \caption{RCL-Kreis}
   \label{fig:RCL_Kreis}
 \end{figure}

Nach den kirchhoffschen Regeln folgt die DGL:
 \begin{equation}
   \ddot{I} + \frac{R}{L} \dot{I} + \frac{1}{LC}I = 0
 \end{equation}
 Mit einem komplesen E-Ansatz erhält man die Winkelgeschwindigkeit $\omega$
 \begin{equation}
   \omega = \sqrt{\frac{1}{LC}-\frac{R²}{4L²}}
 \end{equation}
 Sowie den Parameter der Amplitudenabnahme $\mu$:
 \begin{equation}
 \mu = \frac{R}{2L}
 \end{equation}
 Hiermit gelangt man zur allgemeinen Lösung:
 \begin{equation}
   F(t) = e^{-\mu t} \cdot  \left( A_1e^{i\omega_1t} + A_2e^{-i\omega_2t} \right) \text{$A_1,A_2$ komplex}
 \end{equation}
 Wie die Lösungsfunktion nun genau aussieht, hängt davon ab ob $\omega$ reell, imaginär
  oder gleich 0 ist.
  bei einem reellen $\omega$ liegt eine gedämpfte Schwingung vor und für I(t) gilt:
  \begin{equation}
    I(t) = A_0 e^{-\mu t} \cdot cos(\omega t + \varphi)
  \end{equation}
 mit der Periodendauer T:
  \begin{equation}
    T = \frac{2 \pi}{\omega}
  \end{equation}

  Für die Ablinkdauer der Amplituden \tau folgt zudem:
  \begin{equation}
    \tau = \frac{2L}{R}
  \end{equation}

Ist $\omega$ hingegen imaginär kommt es zu einer
aperiodischen Dämpfung, welche nach hinreichender Zeit  ein einfaches Relaxationsverhalten
 aufzeigt. Je nachdem wie $A_1$ und $A_2$ bestimmt sind, kann I(t) vorher jedoch noch einen
 Maximalwert durchlaufen.

 \begin{figure}[H]
   \centering
   \includegraphics[width=\linewidth-200pt,height=\textheight-200pt,keepaspectratio]{content/dämpfungen.png}
   \caption{Relaxationsverhalten bei verschiedenen Dämpfungen}
   \label{fig:Dämpfungen}
 \end{figure}

 Beträgt $\omega$ jedoch 0 liegt der aperiodische Grenzfall vor. Bei diesem, in der Darstellung schwarz gestrichelt,
  durchläuft I(t) kein Maximum und flacht am schnellsten ab.

\subsection{ Der RCL-Kreis unter Einfluss einer erzwungenen Schwingung}

Wenn ein gedämpfte Schwingkreis einer äußeren periodischen Kraft F(t), im Fall des RCL-Kreises
 eine Wechselspannung, unterworfen wird zeigen sich einige Eigenschaften.

Für die DGl des Schwingkreises folgt dann:
\begin{equation}
  LC \ddot{U}_C + RC \dot{U}_C + U_C = U_0 \cdot e^{i\omega t}
\end{equation}
Mithilfe eines komplexen e-Ansatzes für $U_C$ erhält man für den Betrag der zugehörigen $A_C(\omega)$
\begin{equation}
  A_C{\omega} = \frac{U_0}{\sqrt{(1-LC\omega^2)^2 + \omega^2R^2C^2}}
\end{equation}
\begin{equation}
  %|A(\omega)| = U_0 \vdot sqrt{\frac{(1-LC\omega^2)² + \omega²R²C²}{\left \left 1-LC\omega²\right ^2 + \omega²R²C²\right^2}}
\end{equation}
Wie beim Rc-Kreis aus V353 zeigt sich auch beim angetriebenen RCL-Kreis eine Phasenverschiebung zwischen $U_C$ und $U_0$.
Für diese gilt:
\begin{equation}
  \varphi(\omega) = arctan\left( \frac{-\omega RC}{1-LC \omega^2}\right)
\end{equation}
Es zeigt sich, dass $U_C$ und U für hinreichend kleine Frequenzen weiterhin in Phase sind,
während $U_C$ für sehr hohe Frequenzen ca $\pi$ hinter U zurückbleibt.



\subsection{Resonanzen und Güte eines angetriebenen RCL-Kreises}
Die Amplitude der Kondensatorspannung ist daher abhängig von der Frequenz der Generatorspannung.
Für $\omega \to \infty$ läuft $U_C$ gegen 0, für $\omega \to 0$ gegen $U_0$.
Für eine Frequenz $f_{res}$ erreicht $U_C$  jedoch ein Maximum, welches größer als $U_0$ selbst ist.
Diese Frequenz wird als Resonanzfrequenz des Systems bezeichnet. Für $f_{res}$ gilt:
\begin{equation}
  f_{res} = \frac{\sqrt{\frac{1}{LC}-\frac{R^2}{2L^2}}}{2 \pi}
  %f_{res} = \frac{\sqrt{\frac{1}{LC} - \frac{R^2}{2L^2}}{2 \pi}}
\end{equation}
Bei auftretender Resonanz ist zwischen zwei Fällen zu unterscheiden:
Für den Fall einer schwachen Dämpfung, also $\frac{R^2}{2L^2} \ll \frac{1}{LC}$, lässt sich $A_{C,max}$ mit:
\begin{equation}
  A_{C,max} = \frac{U_0}{\omega_0 RC}
  %A_{C,max} = \frac{U_0}{R} \cdot \sqrt{\frac{L}{C}}
\end{equation}
beschreiben.
Läuft R dann auch noch gegen 0, wird $A_{C,max}$ unendlich groß und es kommt zu einer
Resonanzkatastrophe.
 Der Faktor $\frac{1}{\omega_0 RC}$ wird auch als Güte bezeichnet.
Letzterer lässt sich auch mithilfe von
\begin{equation}
  q = \frac{\omega_0}{\omega_+ - \omega_-}
  \end{equation}
bestimmen. $\omega_+$ und $\omega_-$ werden  hierzu durch die Stellen links und rechts neben dem Maximum,
 an denen $A_C$ $1/sqrt{2}$ von $A_{C,max}$ erreicht, bestimmt.Die Breite der Resonanzkurve wird durch
 \begin{equation}
   \omega_+ - \omega_- \approx \frac{R}{L}
 \end{equation}
 angegeben. Es zeigt sich, dass diese mit mit den Winkelgesschwindigkeiten $\omega_1,\omega_2$ zusammenfallen.
  Bei diesen beträgt die Phasendifferenz $\varphi$ $\pm \frac{\pi}{4}$.


 Im Falle einer starken Dämpfung existiert keine Resonanzfrequenz, $A_C$ fällt bei steigender Frequenz
 , von $U_0$ ausgehend, monoton gegen 0. Ist die Frequenz ausreichend hoch, zeichnet
  sich eine $\frac{1}{\omega²}$ Abhängigkeit ab.In diesem Fall findet der RCL-Kreis auch als Tiefpassfilter Verwendung.
