\section{Durchführung}
\label{sec:Durchführung}
\renewcommand{\labelenumi}{\alph{enumi})}
\begin{enumerate}
\item Es wird die Zeitabhängigkeit der Amplitude einer gedämpften Schwingung untersucht
 und der effektive Dämpfungswiderstand daraus ermittelt.Hierzu wird ein RCL-Kreis gemäß
 Abb. 99999999 angeschlossen. Der Generator wird auf Rechteckimpulse gestellt
  und die Frequenz so gewählt, dass die Amplitude der Schwingung etwa auf $\frac{1}{6}$
   der ursprünglichen gesunken ist, bevor sie einen neuen Impuls erfährt. Das Ossziloskop
    wird dementsprechend eingestellt, dass der Schwingungsverlauf vollständig auf dem Bildschirm dargestellt wird.
    Anschließend wird ein Abbild des Bildschirm für die Auswertung angefertigt.
    Für diese wird die Einhüllende der Schwingungskurve in das Abbild eingezeichnet.
     Zusätzlich werden einige Wertepaare entnommen und eine Ausgleichsrechnung hiermit vollzogen.
     Aus dem bestimmten Exponenten des exponentiellen Verlaufs werden schließlich nach (9999999)
     $R_{eff}$ und $T_{ex}$ ermittelt. Zudem soll $R_{eff}$ mit dem originalen R der Schaltung verglichen werden.

     \item Es soll der Dämpfungswiderstand $R_{ap}$ bestimmt werden, bei der aperiodische Grenzfall vorliegt.
     Dazu wird der feste Widerstand der Schaltung von a) durch einen variablen Widerstand $R_v$
     ausgetauscht. Dieser wird auf seinen Maximalwert eingestellt. Anschließend verringert man $R_v$
     solange, bis sich das auftretende Relaxationsverhalten zu einer Schwingung ummformt.
      Hat man dies erreicht, dreht man $R_v$ wieder hoch, bis kein Überschwinger mehr zu sehen ist.
      Für die Auswertung wird der gemessene $R_{ap}$-Wert mit dem aus L und C berechneten Theoriewert verglichen.

      \item Es wird ein Serien RCL-Kreis auf die Frequenzabhängigkeit seiner Kondensatorspannung hin untersucht.
      Die Schaltung wird zunächst gemäß Abb. 9999999 aufgebaut, der Generator auf eine Sinusspannung eingestellt.
      Anschließend wird eine Messreihe von $U_C$ für verschiedene Frequenzen ausgeführt
       Da auch die Ausgangsspannung U des Tastkopfes frequenzabhängig ist, ist es notwendig, diese mitzunotieren.
       Zur Auswertung wird $\frac{U_C}{U}$ gegen f in ein halblogarithmisches Diagramm eingetragen und diesem
        anschließend q entnommen.Der Bereich um die Resonanzfrequenz soll zudem nochmals linear
         abgebildet werden. Hieraus wird zusätzlich die Breite der Resonanzkurve ermittelt.
         Anschließend werden die Ergebnisse mit den errechneten Theoriewerten verglichen.

         \item Zuletzt wird die Frequenzabhängigkeit der Phase zwischen $U_C$ und $U_{Antrieb}$ des
          SerienRCL_Kreises untersucht werden. Hierzu wird der RCL-Kreis gemäß Abb. 9999999
          angeschlossen.Das Ossziloskop wird so eingestellt, sodass $U_C$ und U übereinander liegen
             Anschließend werden jeweils
      $\Delta t$ zwischen den Nullstellen und die zugehörige Frequenz notiert. Die jeweilige
       Phasendifferenz berechnet sich mit:
       \begin{equation}
         \varphi = 2 \pi \cdot \Delta t \cdot f_{Antrieb}
       \end{equation}
       Für die Auswertung wird $\varphi$ gegenüber f in einem halblogarithmischen
        Diagramm aufgetragen. Der Bereich um $\varphi = \frac{pi}{2}$ wird zusätzlich
         linear dargestellt um ihm $f_res$, $f_1$ und $f_2$ zu entnehmen. Auch diese werden wieder
         mit den berechneten verglichen. 
\end{enumerate}
