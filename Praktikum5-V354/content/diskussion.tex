\section{Diskussion}
\label{sec:Diskussion}

In allen Graphen lassen sich keine größen Abweichungen zwischen der theoretschen Kurve und den eingetragenen Werten erkennen. Dies lässt sich auch in den Unsicherheiten auf die Parameter der Theoriekurve, die aus den Werten mittels nichtlinearer Ausgleichsrechnung bestimmt wurden erkennen. Diese liegt immer unter $\SI{1.5}{\percent}$ und meistens sogar unter $\SI{1}{\percent}$. Jedoch lässt sich in der Auswertung bei der Messung des Widerstandes, bei dem der aperiodische Grenzfall eintritt eine Abweichung von $\SI{5428(17)}{\ohm}$ festellen. Diese ist zu hoch um mit dem Einfluss des Innenwiderstandes der Spannungsquelle, obwohl dieser auch einen Beitrag liefert erklärt zu werden. Da der gemessene Wert eine so große Abweichung besitzt handeltes sich vermutlich um einen groben Ablesefehler. Auch bei der Berechnung der Güte $q$ und der Breite der Resonanzkurve $f_+ - f_-$ lässt sich eine Abweichung zu den aus den gegebenen Werten $L$, $C$ und $R_2$ berechnenten Werten feststellen. Diese kann nicht durch die Eigenschaften der Spannungsquelle begründet werden, da $A$ durch $A_{Antrieb}$ geteilt wurde. Somit kann die Abweichung nur noch durch die verwendeten Kabel, dem $RCL$-Glied und dem Oszilloskop zustande kommen. Bei der Berechnung von der Resonanzfrequenz $f_{res}$ und $f_{1,2}$ 







	
