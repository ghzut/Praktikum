\section{Diskussion}
\label{sec:Diskussion}

In allen Graphen lassen sich keine größen Abweichungen zwischen der theoretschen Kurve und den eingetragenen Werten erkennen. Dies lässt sich auch in den Unsicherheiten auf die Parameter der Theoriekurve, die aus den Werten mittels nichtlinearer Ausgleichsrechnung bestimmt wurden erkennen. Diese liegt immer unter $\SI{1.5}{\percent}$ und meistens sogar unter $\SI{1}{\percent}$. Jedoch lässt sich in der Auswertung bei der Messung des Widerstandes, bei dem der aperiodische Grenzfall eintritt eine Abweichung von $\SI{5428(17)}{\ohm}$ festellen. Diese ist zu hoch um mit dem Einfluss des Innenwiderstandes der Spannungsquelle, obwohl dieser auch einen Beitrag liefert erklärt zu werden. Da der gemessene Wert eine so große Abweichung besitzt handeltes sich vermutlich um einen groben Ablesefehler. Auch bei der Berechnung der Güte $q$ und der Breite der Resonanzkurve $f_+ - f_-$ lässt sich eine Abweichung zu den aus den gegebenen Werten $L$, $C$ und $R_2$ berechnenten Werten feststellen. Diese kann nicht durch die Eigenschaften der Spannungsquelle begründet werden, da $A$ durch $A_{Antrieb}$ geteilt wurde. Somit kann die Abweichung nur noch durch die verwendeten Kabel, dem $RCL$-Glied und dem Oszilloskop zustande kommen. Bei der Berechnung von der Resonanzfrequenz $f_{res}$ und $f_{1,2}$ ist auch hier eine Abweichung zwischen den mit den gegebenen Werten $L$, $C$ und $R_2$ berechnenten Werten und den mit den experimentell bestimmten Werten für $LC$ und $RC$ berechneten Werten festzustellen. Auch hier ist die Messung unabhängig von den Eigenschaften der Spannungsquelle, da die Phasendifferenz $\varphi$ zwischen $U_{Antrieb}$ und $U_C$ gemessen wird. Bei dem Vergleich der experimentell ermittelten Werte für $LC$ und $RC$ in c) und d) fällt auf, dass diese gleich sind bis auf die Standartabweichung. Dies schließt das Oszilloskop als Quelle der systematischen Abweichung aus, da einmal am Oszilloskop nur die Spanung und das andere Mal die zeitliche Differenz gemessen wurde. Es bleiben also die Kabel und das $RCL$-Glied als mögliche Quelle des systematischen Fehlers. Mit den gegeben Werten für $R2$, $L$, $C$ ergibt sich:
\begin{displaymath}
LC = \SI{3.467(21)e-11}{\second\squared}
\end{displaymath}
\begin{displaymath}
RC = \SI{1.409(5)e-6}{\ohm\squared\second}
\end{displaymath} 
Vergleicht man nun diese Werte mit den experimentell ermittelten $LC$ und $RC$ aus c) und d) 







	
