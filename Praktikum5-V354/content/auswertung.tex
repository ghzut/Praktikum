\section{Auswertung}
\label{sec:Auswertung}

\subsection{Bestimmung von Reff}
\begin{figure}[H]
	\centering
	\caption{Die Kondensatorspannung $U_C$ aus Tabelle 1.}
	\includegraphics[width=\linewidth-70pt,height=\textheight-70pt,keepaspectratio]{graa.pdf}
	\label{fig:graa}
\end{figure}
\input{../build/taba.tex}

\subsection{Bestimmung von Rap}

\subsection{Bestimmung von q und Resonanzbreite}
\begin{figure}[H]
	\centering
	\caption{Die Kondensatorspannung $U_C$ aus Tabelle 2.}
	\includegraphics[width=\linewidth-70pt,height=\textheight-70pt,keepaspectratio]{grac1.pdf}
	\label{fig:grac1}
\end{figure}
\begin{figure}[H]
	\centering
	\caption{Die Kondensatorspannung $U_C$ aus Tabelle 2.}
	\includegraphics[width=\linewidth-70pt,height=\textheight-70pt,keepaspectratio]{grac2.pdf}
	\label{fig:grac2}
\end{figure}
\input{../build/tabc.tex}

\subsection{Phasenverschiebung}
\begin{figure}[H]
	\centering
	\caption{Die Phasenverschiebung $\varphi$ aus Tabelle 3.}
	\includegraphics[width=\linewidth-70pt,height=\textheight-70pt,keepaspectratio]{grad1.pdf}
	\label{fig:grad1}
\end{figure}
\begin{figure}[H]
	\centering
	\caption{Die Phasenverschiebung $\varphi$ aus Tabelle 3.}
	\includegraphics[width=\linewidth-70pt,height=\textheight-70pt,keepaspectratio]{grad2.pdf}
	\label{fig:grad2}
\end{figure}
\input{../build/tabd.tex}

