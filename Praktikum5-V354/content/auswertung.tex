\section{Auswertung}
\label{sec:Auswertung}
In der Auswertung wurden die Graphen in den Abbildungen  bis  mit Matplotlib \cite{matplotlib} und NumPy \cite{numpy} angefertigt.\\


Folgende Werte waren geben und somit bereits vor der Messung bekannt:
\begin{displaymath}
R_1 = \SI{67.2(2)}{\ohm}
\end{displaymath}
\begin{displaymath}
R_2 = \SI{682(1)}{\ohm}
\end{displaymath}
\begin{displaymath}
L = \SI{16.78(9)e-3}{\henry}
\end{displaymath}
\begin{displaymath}
C = \SI{2.066(6)e-9}{\farad}
\end{displaymath}

\subsection{Bestimmung von Reff}
\begin{figure}[H]
	\centering
	\caption{$U_C$ im $RCL$-Kreis in Abhängigkeit der Zeit.}
	\includegraphics[width=\linewidth-70pt,height=\textheight-70pt,keepaspectratio]{graa.pdf}
	\label{fig:graa}
\end{figure}
\input{../build/taba.tex}
$\mu$ wird mithilfe einer nichtlinearen Ausgleichsrechnung mittels SciPy \cite{scipy} bestimmt. Es ergibt sich aus den Daten aus Tabelle 1:
\begin{displaymath}
\mu = \SI{3.87(5)e3}{\per\second}\text{.}
\end{displaymath}
Hiermit ergibt sich nach Formel ():
\begin{displaymath}
R_{eff} = \SI{130 \pm 1.9}{\ohm}\text{.}
\end{displaymath}
Der Vergleich mit $R_1$ ergibt, dass $R_{eff}$ ca. $\SI{60}{\ohm}$ größer ist als $R_1$. Dies wird durch dem Innerwiderstand $R_i$ der Spannungsquelle verursacht.

\subsection{Bestimmung von Rap}

\subsection{Bestimmung von q und Resonanzbreite}
\begin{figure}[H]
	\centering
	\caption{$A_C$ im angetriebenen $RCL$-Kreis in Abhängigkeit von $f_{Antrieb}$.}
	\includegraphics[width=\linewidth-70pt,height=\textheight-70pt,keepaspectratio]{grac1.pdf}
	\label{fig:grac1}
\end{figure}
$LC$ und $RC$ werden mithilfe einer nichtlinearen Ausgleichsrechnung mittels SciPy \cite{scipy} bestimmt. Es ergibt sich aus den Daten aus Tabelle 2: 
\begin{displaymath}
LC = \SI{3.626(13)e-11}{\second\squared}
\end{displaymath}
\begin{displaymath}
RC = \SI{1.505(12)e-6}{\ohm\squared\second}\text{.}
\end{displaymath}
Hiermit berechnet sich mit Formel ():
\begin{displaymath}
q = \num{4.002(33)}\text{.}
\end{displaymath}
Mit den gegebenen Werten $R_2$ und $L$ erhält man mit Formel ():
\begin{displaymath}
q = \num{4.179(14)}\text{.}
\end{displaymath}


\begin{figure}[H]
	\centering
	\caption{Lineare Betrachtung von $A_C$ im nahen Bereich um $f_{res}$.}
	\includegraphics[width=\linewidth-70pt,height=\textheight-70pt,keepaspectratio]{grac2.pdf}
	\label{fig:grac2}
\end{figure}
\input{../build/tabc.tex}
Für die Breite der Ressonanzkurve ergibt sich mit dem zuvor bestimmten $LC$ und $RC$, der Formel (14) und der Beziehung $f_+ - f_- = \frac{\omega_+ - \omega_-}{2\pi}$:
\begin{displaymath}
f_\text{+} - f_\text{-} = \SI{6.82(6)e3}{\per\second}\text{.}
\end{displaymath}
Mit den gegebenen Werten für $L$, $R_2$ und $C$ ergibt sich:
\begin{displaymath}
f_+ - f_- = \SI{6.82(6)e3}{\per\second}\text{.}
\end{displaymath}


\subsection{Phasenverschiebung}
\begin{figure}[H]
	\centering
	\caption{$\varphi$ im angetriebenen $RCL$-Kreis in Abhängigkeit von $f_{Antrieb}$ }
	\includegraphics[width=\linewidth-70pt,height=\textheight-70pt,keepaspectratio]{grad1.pdf}
	\label{fig:grad1}
\end{figure}
\begin{figure}[H]
	\centering
	\caption{Lineare Betrachtung von $\varphi$ im nahen Bereich um $f_{res}$}
	\includegraphics[width=\linewidth-70pt,height=\textheight-70pt,keepaspectratio]{grad2.pdf}
	\label{fig:grad2}
\end{figure}
\input{../build/tabd.tex}
$LC$ und $RC$ werden mithilfe einer nichtlinearen Ausgleichsrechnung mittels SciPy \cite{scipy} bestimmt. Es ergibt sich aus den Daten aus Tabelle 3:
\begin{displaymath}
LC = \SI{3.626(4)e-11}{\second\squared}
\end{displaymath}
\begin{displaymath}
RC = \SI{1.505(10)e-6}{\ohm\squared\second}\text{.}
\end{displaymath}
Mit Formel (11) ergibt sich:
\begin{displaymath}
f_{res} = \SI{26014(14)}{\per\second}
\end{displaymath}
und mit Formel (16) und der Beziehung $f = \frac{\omega}{2\pi}$:
\begin{displaymath}
f_1 = \SI{29938(29)}{\per\second}
\end{displaymath}
\begin{displaymath}
f_2 = \SI{23334(21)}{\per\second}\text{.}
\end{displaymath}
Mit den gegebenen Werten für $R_2$, $L$ und $C$ ergibt sich:
\begin{displaymath}
f_{res} = \SI{2664(8)e4}{\per\second}
\end{displaymath}
\begin{displaymath}
f_1 = \SI{29938(29)}{\per\second}
\end{displaymath}
\begin{displaymath}
f_2 = \SI{23334(21)}{\per\second}\text{.}
\end{displaymath}


