\section{Aufbau}
\label{sec:Aufbau}


\begin{figure}[H]
 \centering
 \includegraphics[width=\linewidth-200pt,height=\textheight-200pt,keepaspectratio]{content/Aufgabec.png}
 \caption{Messchaltung zur Bestimmung der Zeitkonstanten $\tau$ \cite{V353}.}
 \label{fig:Aufbaua}
\end{figure}
Die obige Schaltskizze zeigt einen Versuchsaufbau zur Messung der spezifischen
 Zeitkonstante $\tau$ eines $RC$-Kreises. Im Kern besteht sie aus einem $RC$-Glied, welches über einen
  Funktiongenerator einer periodischen Spannung unterliegt. Um den
   Spannungsverlauf am Kondensator zu untersuchen wird zudem ein
    Zweikanaloszilloskop am Kondensator zugeschaltet. Zunächst findet jedoch
     erst einmal ein Kanal Verwendung
