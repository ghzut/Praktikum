\section{Durchführung}
\label{sec:Durchführung}
\renewcommand{\labelenumi}{\alph{enumi})}
\begin{enumerate}
  \item Es soll die Zeitkonstante $\tau$ durch Beobachtung des Auf- oder Entladevorgangs
  des Kondensators bestimmt werden.Am Generator wird eine Rechteckspannung gewählt
  Anschließend wird das Bild des Oszilloskops auf einen
  geiegneten Bereich eingestellt, auf dem eine fast vollständige Auf- oder Entladung, sowie
  die Maximal- und die Nullspannung zu sehen sind. Zuletzt wird hiervon ein Bild für die
  Auswertung gespeichert.Für diese werden dem Bild einige Wertepaare entnommen und in einem
  halblogarithmischen Diagramm dargestellt. Für die Bestimmung von $\tau$ wird eine
  lineare Ausgleichsrechnung mit $\{(ln(U_C), t)\}$ durchgeführt, für dessen dessen Steigung nach (4) gilt:
  \begin{equation}
  a = -\frac{1}{RC}
  \end{equation}

  \item Es wird die Amplitude A von $U_C$ an einem RC-Glied, welches an einem Sinusgenerator
   angeschlossen ist, in Abhängigkeit von der Frequenz gemessen. Es soll über
   3 Zehnerpotenzen hinweg gemessen werden. Zur Auswertung wird $\{(A(f_{Antrieb}), f_{Antrieb})\}$
   in einem halblogarithmischen Diagramm aufgetragen. Anschließend werden $\tau$ und $U_0$ mithilfe einer nicht-linearen
    Ausgleichsrechnung ermittelt. Mithilfe dieser wird die Theoriekurve nach (7) erstellt und
    ebenfalls eingetragen. Es soll diskutiert werden ob eine systematische Abweichungen
    zwischen dem $\tau$ aus a) und dem aus b) existiert.



    \item Es wird die Phasenverschiebung zwischen $U_{Antrieb}$ und $U_C$ an einem
    RC-Glied in Abhängigkeit von $f_{Antrieb}$ gemessen.Hierzu wird ein Zweikanaloszilloskop
    einmal über den Ausgang am Kondensator und einmal direkt an den Sinusgenerator angeschlossen
     und so eingestellt, dass beide Graphen übereinander liegen.Es werden jeweils
      $\Delta T$ zwischen den Nullstellen und die zugehörige Frequenz notiert. Die jeweilige
       Phasendifferenz berechnet sich mit:
       \begin{equation}
         \varphi = 2 \pi \cdot \Delta T \cdot f_{Antrieb}
       \end{equation}
       Für die Auswertung wird mit der Menge der Messertpaare $\{(\varphi(f_{Antrieb}),f_{Antrieb})\}$
       analog wie in Teil b) verfahren.

       \item Es soll gezeigt werden, dass der RC-Kreis nach den Voraussetzungen aus 2.4 als
       Integrator arbeiten kann. Die Schaltung beleibt dieselbe, wie bei Aufgabenteil c).
       Anschließend wird eine hohe Frequenz am Generator eingestellt und das Oszilloskop so eingestellt, dass
       die zu integrierende und die integrierte Spannung dargestellt werden. Es werden wieder
       Bilder der Graphen erstellt.

       \item Zuletzt wird die Relativamplitude $A(\omega) / U_0$ in Abhängigkeit von $\varphi$ dargestellt.
       Dies folgt mit:
       \begin{equation}
         A(\varphi) = -U_0\cdot sin(\varphi)\sqrt{\frac{1}{sin^2(\varphi)}-1}
       \end{equation}
       Zusätzlich werden auch einige Messwertpaare $(A(\omega_i)/U_0,\varphi(\omega_i))$ in das Diagramm
        eingetragen.

\end{enumerate}
