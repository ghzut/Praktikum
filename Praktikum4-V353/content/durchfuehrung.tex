\section{Durchführung}
\label{sec:Durchführung}


\renewcommand{\labelenumi}{\alph{enumi})}
\begin{enumerate}
  \item Es soll die Zeitkonstante $\tau$ durch Beobachtung eines Auf- oder Entladevorgangs
  des Kondensators bestimmt werden. Die Schaltung wird gemäß Abb. 3 aufgebaut. Am Funktionsgenerator wird eine Rechteckspannung gewählt.
  Anschließend wird die Frequenz der Generatorspannung und des Oszilloskops auf einen
  geeigneten Bereich eingestellt, sodass der Graph eine fast vollständige Auf- oder Entladung darstellt. In unserem Fall
  liegt die Ausgangsspannungamplitude am Generator bei $\SI{14.34}{\volt}$,
   die eingestellte Generatorfrequenz bei $\SI{173.3}{\hertz}$. Zuletzt wird ein Bild des Graphen für die
  Auswertung gespeichert.% Für diese werden dem Graphen einige Wertepaare entnommen und in einem
  %halblogarithmischen Diagramm dargestellt. Zur Bestimmung von $\tau$ wird eine
  %lineare Ausgleichsrechnung mit $\{(t, ln(U_C))\}$ durchgeführt, für dessen Steigung nach (4) gilt:
  %\begin{equation}
  %a = -\frac{1}{RC}
  %\end{equation}
  %\begin{figure}[H]
  %	\centering
  %	\includegraphics[width=\linewidth-200pt,height=\textheight-200pt,keepaspectratio]{content/Aufgabea.png}
  %	\caption{Messchaltung zum Aufgabenteil a) \cite{V353}}
  %	\label{fig:Aufbaua}
  %\end{figure}

  \item Es wird die Amplitude $A$ von $U_C$ an einem $RC$-Glied in Abhängigkeit der Frequenz gemessen. Auch hier wird
  hier wird der Aufbau aus Teilaufgabe a) verwendet, jedoch wird am Funktionsgenerator eine Sinusspannung gewählt.
   Für eine aussagekräftige Messreihe wird der Frequenz im Bereich von
    $\SI{10}{\hertz}$ bis $\SI{100}{\kilo\hertz}$ variiert.% Zur Auswertung wird $\{(f_{Antrieb}, A(f_{Antrieb}))\}$
   %in einem halblogarithmischen Diagramm aufgetragen. Anschließend werden $\tau$ und $U_0$ mithilfe einer nichtlinearen
    %Ausgleichsrechnung ermittelt. Mithilfe dieser wird die Theoriekurve nach (7) erstellt und
    %ebenfalls eingetragen. Es soll diskutiert werden, ob eine systematische Abweichungen
    %zwischen dem $\tau$ aus a) und dem aus b) existiert.
	%\begin{figure}[H]
	%	\centering
	%	\includegraphics[width=\linewidth-200pt,height=\textheight-200pt,keepaspectratio]{content/Aufgabeb.png}
	%	\caption{Messchaltung zum Aufgabenteil b) \cite{V353}}
	%	\label{fig:Aufbaua}
	%\end{figure}


    \item Es wird die Phasenverschiebung zwischen $U_{\text{Antrieb}}$ und $U_C$ an einem
    RC-Glied in Abhängigkeit von $f_{\text{Antrieb}}$ gemessen. Die Spannung am
    Funktiongenerator wird wieder auf eine Sinusspannung gestellt.
     Zusätzlich wird das Oszilloskop über einen zweiten Kanal direkt an den Generator angeschlossen.
     Das Oszilloskop wird anschließend so eingestellt, dass beide Spannungsgraphen
      übereinander liegen. Es werden jeweils $\Delta t$ zwischen den Nullstellen
      und die zugehörige Frequenz notiert. Die jeweilige
       Phasendifferenz berechnet sich mit:
       \begin{equation}
         \varphi = 2 \pi \cdot \Delta t \cdot f_{\text{Antrieb}}
       \end{equation}
      % Für die Auswertung wird mit der Menge der Messertpaare $\{(f_{Antrieb}, \varphi(f_{Antrieb}))\}$
       %analog wie in Teil b) verfahren.
       %\begin{figure}[H]
       	%\centering
       	%\includegraphics[width=\linewidth-200pt,height=\textheight-200pt,keepaspectratio]{content/Aufgabec.png}
       	%\caption{Messchaltung zu den Aufgabenteile c) und d) \cite{V353}}
       	%\label{fig:Aufbaua}
       %\end{figure}

	%\item Es wird die Relativamplitude $A(\omega) / U_0$ in Abhängigkeit von $\varphi$ in einem Polarkoordinatensystem dargestellt.
	%Dies folgt aus Formel (7) und (8):
	%\begin{equation}
	%A(\varphi) = -U_0\cdot sin(\varphi)\sqrt{\frac{1}{sin^2(\varphi)}-1}\text{.}
	%\end{equation}
	%Zusätzlich werden auch einige Messwertpaare $(\varphi(\omega_i), A(\omega_i)/U_0)$ in das Diagramm
	%eingetragen.

       \item Zuletzt soll gezeigt werden, dass der $RC$-Kreis nach den Voraussetzungen aus 2.4 als
       Integrator arbeiten kann. Die Schaltung bleibt dieselbe, wie bei Aufgabenteil c).
       Anschließend wird eine hochfrequente Spannung am Generator gewählt und das Oszilloskop so eingestellt, dass
       die zu integrierende und die integrierte Spannung dargestellt werden. Zum Vergleich werden
       Bilder der Graphen erstellt.



\end{enumerate}
