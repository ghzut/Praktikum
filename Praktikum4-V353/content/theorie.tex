\section{Theorie}
\label{sec:Theorie}
\subsection{Die allgemeine Relaxationsgleichung}
Unter einer Relaxationserscheinung versteht man allgemein das Phänomen, wenn
 ein System aus seinem Ausgangszustand entfernt wird und es sich nicht osszilatorisch
 in diesen zurückbewegt.Die Änderungsrate der betrachteten Größe S verläuft dabei
 in der Regel proportional zur verliebenen Entfernung bis zum Ausgangszustand. Daher nähert sich
 das System diesem auch nur asymptotisch an, erreicht ihn aber nie wieder.Für diese
 Bewegung gilt dann:
 \begin{equation}
   \frac{dA}{dt} = c[A(t)-A(\infty)]
 \end{equation}
Womit man zur allgemeinen Bewegungsfleichung des Systems gelangt:
\begin{equation}
  A(t) = A(\infty)+[A(0)-A(\infty)]e^(ct)
\end{equation}
Damit A einen endlichen Endpunkt hat muss $ c<0$ gelten.

\subsection{Relaxationsverhalten bei einem RC-Gliedes}
