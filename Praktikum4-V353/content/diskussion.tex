\section{Diskussion}
\label{sec:Diskussion}\textbf{}
Bei den Graphen in Abbildung 6 und 7 liegen die Messwerte sehr nah an dem berechneten Fit. In den Abbildungen 8 und 9 jedoch lässt sich eine starke Abweichung von dem Fit bzw. der Theoriekurve feststellen. Beim Vergleich der in a), b) und c) bestimmten $\tau$ fällt auf, dass $\tau_c$ leicht höher ausfällt, als die von $\tau_a$ und $\tau_b$. Aufgrund der starken ungleichmäßigen Abweichung der Messwerte von der Theoriekurve liegt die Abweichung von $\tau_c$ jedoch im Bereich der Messunsicherheit. Ein Einfluss des Innenwiderstandes der Spannungsquelle ist nicht zu erkennen, da die ermittelten Werte von $\tau_a$, $\tau_b$ und $\tau_c$ kein systematischen Fehler erkennen lassen und der Innenwiderstand $R_i$ somit vermutlich sehr klein im Verhältnis zu $R$ ist. Bei der Betrachtung des Graphen in Abbildung 13 fällt auf, dass die Messwerte zwar gut auf der ermittelten Theoriekurve liegen, das daraus ermittelte $RC$ jedoch nicht annähernd zu den vorher bestimmten $RC$-Werten passt. Somit ist zu vermuten, dass die starke systematische Abweichung bei dem Graphen in Abbildung 9 sich auf ein anderes RC-Glied zurückführen lässt. 

\begin{figure}[H]
	\centering
	\caption{Phasendifferenz zwischen $U_{\text{Antrieb}}$ und $U_C$ aus Tabelle 4 in Abhängigkeit der Antriebsfrequenz mit ermittelten Fit.}
	\includegraphics[width=\linewidth-70pt,height=\textheight-70pt,keepaspectratio]{build/gradis.pdf}
	\label{fig:Dis}
\end{figure}
Eine nichtlineare Ausgleichsrechnung der Form $y = \arctan(-2\pi \cdot a \cdot x)$ mit SciPy \cite{scipy} liefert mit Formel (8):
\begin{displaymath}
\tau = RC = a = \SI{0.68 \pm 0.01}{\milli\second}\text{.}
%0,68 \text{ ms } \pm 0,01 \text{ ms.}
\end{displaymath}






	
