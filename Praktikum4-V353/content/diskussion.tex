\section{Diskussion}
\label{sec:Diskussion}\textbf{}
Bei den Graphen in Abbildung 3 und 4 liegen die Messwerte sehr nah an der ermittelten Theoriekurve. In den Abbildungen 5 und 6 lässt sich eine starke Abweichung von der Theoriekurve feststellen.
Beim Vergleich der mit a), b) und c) bestimmten $\tau$ fällt auf, dass das Ergebnis $\tau_c$ leicht höher ausfällt, als die von $\tau_a$ und $\tau_b$. Dies könnte als systematischer Fehler deklariert werden, entfällt jedoch bei Betrachtung der gleichen Versuchsaufbauten von b) und c). Daher scheinen die Fehler fast gänzlich im Bereich der Messunsicherheit zu liegen. Ein Einfluss des Innenwiderstandes der Spannungsquelle ist nicht zu erkennen. Die starke systematische Abweichung bei dem Graphenn in Abbildung 6 lässt sich vermutlich auf ein anderes RC-Glied zurückführen, da bei Betrachtung des Graphen in Abbildung 10 auffällt, dass die Messwete zwar gut auf der ermittelten Theoriekurve liegen, jedoch das daraus ermittelte $RC$ nicht annähernd zu den vorher bestimmten $RC$-Werten passt.
\begin{figure}[H]
	\centering
	\caption{}
	\includegraphics[width=\linewidth-70pt,height=\textheight-70pt,keepaspectratio]{build/gradis.pdf}
	\label{fig:Dis}
\end{figure}
Eine nichtlineare Ausgleichsrechnung mit SciPy \cite{scipy} liefert:
\begin{displaymath}
RC=0,68 \text{ ms } \pm 0,01 \text{ ms.}
\end{displaymath}






	
