\section{Auswertung}
\label{sec:Auswertung}


Die Graphen wurden sowohl mit Matplotlib \cite{matplotlib} als auch NumPy \cite{numpy} erstellt. Die
Fehlerrechnung wurde mithilfe von Uncertainties \cite{uncertainties} durchgeführt.

\subsection{Die gemessenen Daten}
Es ist zu beachten, dass die gemessenen Aufschläge ein statistisches Verhalten
aufweisen und daher mit einem Fehler von
\begin{equation}
\sigma = \sqrt{N}
\end{equation}
behaftet sind. Daraus wird die Intensität mit
\begin{equation}
  I = frac{N}{t} \pm \frac{\sigma}{t}
\end{equation}
 berechnet. Zusätzlich müssen die Treffer des Nulleffektes abgezogen werden,
welcher aufgrund von kosmischer Strahlung auftreten. Da die daraus folgenden
durchgehenden Intensitäten jedoch ungleich kleiner sind, werden die zugehörigen
 Fehler nicht berücksichtigt. Für die Gammastrahlung ergibt sich ein
 Nulleffekt von $\SI{1.001}{\per\second}$, für die Betastrahlung $\SI{0.514}{\per\second}$.

 \begin{table}
  \centering
  \caption{Die Materialeigenschaften der verwendeten Absorber.}
  \input{build/rohdaten.tex}
  \label{tab:rohdaten}
 \end{table}

\begin{table}
 \centering
 \caption{Die Absorptionsdaten der Gammastrahlung mit Kupfer als Absorber }
 \input{build/tabgammakupfer.tex}
 \label{tab:k}
\end{table}

\begin{table}
 \centering
 \caption{Die Absorptionsdaten der Gammastrahlung mit Kupfer als Absorber }
 \input{build/tabgammaeisen.tex}
 \label{tab:e}
\end{table}

\begin{table}
 \centering
 \caption{Die Absorptionsdaten der Betastrahlung mit Kupfer als Absorber }
 \input{build/tabgammabetaJ.tex}
 \label{tab:betaJ}
\end{table}

\subsection{Bestimmung der Absorptionskoeffizienten und der Ausgangsstrahlung mithilfe von Gamma-Absorptionskurven.}

\begin{figure}
 \centering
 \caption{Die halblogarithmische Darstellung der durchgehenden Gammastrahlintensität gegenüber der Schichtdicke des Kupferabsorbers.}
 \includegraphics[width=\linewidth-70pt,height=\textheight-70pt,keepaspectratio]{build/Kupfer.pdf}
 \label{fig:kupfer}
\end{figure}

\begin{figure}
 \centering
 \caption{Die halblogarithmische Darstellung der durchgehenden Gammastrahlintensität gegnüber der Schichtdicke des Eisenabsorbers.}
 \includegraphics[width=\linewidth-70pt,height=\textheight-70pt,keepaspectratio]{build/Eisen.pdf}
 \label{fig:eisen}
\end{figure}

\begin{table}
 \centering
 \caption{Die Absorptionsdaten der Betastrahlung mit Aluminium als Absorber }
 \input{build/tabgammabetaJ.tex}
 \label{tab:betaJ}
\end{table}

Zuerst wird das Absorptionsverhalten von Kupfer und Eisen unter der Gammastrahlung einer $^{137}$Cs-Quelle untersucht. Hierzu werden die Koeffizienten $\mu$ und $N(0)$ des Absorptionsgestzes für beide Metalle bestimmt. Mithilfe einer halblogarithmischen Darstellung der zeitlich normierten Treffer gegenüber der Metalldicke ergeben sich die Graphen in  den Abb. \ref{fig:kupfer} und \ref{fig:kupfer}. Damit folgen $\mu$ und $N(0)$ über eine Ausgleichsrechnung der Form
\begin{equation}
y = a x+ b \text{ mit } \mu = a \text{ und } N(0) = \exp(b)
\end{equation} Damit ergeben sich die experimentellen Ergebnisse in Tabelle \ref{tab:ergebnisse}.

\begin{table}
 \centering
 \caption{Die Ergebnisse der Gammastrahlungsabsorption.}
 \input{build/ergebnisse.tex}
 \label{tab:erg1}
\end{table}

\subsection{Vergleich mit den theoretischen Absorptionskoeffizienten}
Mithilfe der Formeln \ref{eq:HJHJ} und \ref{eq:HJHKJ} ergeben sich die theoretischen
Absorptionskoeffizienten. Auf Basis der Daten aus Tabelle \ref{tab:rohdaten} und einem
 $\epsilon$ von 1.295 \cite{V704} folgen die berechneten Koeffizienten in Tabelle \ref{fig:erg1}.
 Die Daten stammen für Kupfer von \cite{Kupfer}, für eisen von \cite{Eisen}.
Es ist zu erkennen das die berechnete Absorption der beiden Metalle in beiden
Fällen größer ausfällt, als die experimentell gemessene. Daher ist davon auszugehen,
dass der Compton-Effekt bei beiden Absorptionen die einzige messbar auftretende Wechselwirkung.
Zusätzlich ist bei der Eisenmessung von einem systematischen Fehler auszugehen,
da dieser weitaus stärker nach unten abweicht. Die Ursachen sind in der Diskussion zu erörtern.
\subsection{Bestimmung der Maximalenergie von $^{99}$Tc }


\begin{figure}
 \centering
 \caption{Die halblogarithmische Darstellung der Treffer pro Sekunde gegnüber der Schichtdicke des Aluminiumabsorbers bei Betastrahlung}
 \includegraphics[width=\linewidth-70pt,height=\textheight-70pt,keepaspectratio]{build/BetaJ.pdf}
 \label{fig:betaj}
\end{figure}

Zuletzt wird die Maximalenergie der Gammaquanten von $^{99}$Tc über deren Absorption von Aluminium bestimmt. Hierzu werden
die durchgehende Strahlintensität halblogarithmisch gegen die Massenbelegung aufgetragen.
Mit den Daten aus Tabelle \ref{tab:betaJ} und einer Dichte von $\SI{2.699}{\gram\per\cubic\centi\meter}$ folgt der Graph in Abb. \ref{fig:betaj}.Anschließend werden
die beiden großteils linearen Bereiche der Kurve durch linearen Funktionen approximiert. Der
 Schnittpunkt beider Geraden bildet das gesuchte $R_\text{max}$. Mithilfe von Formel \ref{eq:HJKH}
 ergibt sich schließlich die Maximalenergie $E_\text{max}$. Mit den bestimmten Daten folgen daher die Parameter in Tabelle \ref{tab:erg2}.
\begin{table}
 \centering
 \caption{Die Ergebnisse der Betastrahlungsabsorption.}
 \input{build/ergebnisse2.tex}
 \label{tab:erg2}
\end{table}
