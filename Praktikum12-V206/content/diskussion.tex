
\section{Diskussion}
\label{sec:Diskussion}
Die Auswertungsergebnisse deuten einige Punkte an, welche zu diskutieren sind.
Beim Vergleich der bestimmten, realen Gütewerte fällt auf, dass diese weit
unterhalb ihrer theoretisch möglichen liegen. Es zeigt sich, dass sich die
Verhältnisse Im Verlauf der Versuchsdurchführung annähern. So liegt der reale
Wert nach $\SI{16}{\minute}$ bei ca. $\SI{25}{\percent}$ des idealen, nach
$\SI{4}{\minute}$ noch bei ca. $\SI{5}{\percent}$. Dies liegt jedoch
hauptsächlich daran, dass die idealen Werte nach Formel \ref{eq:vid} bei
größeren Temperaturdifferenzen zwischen den Reservoirs stark sinken. In die Nähe
der idealen Werte kommen sie aber trotzdem nicht. Grund hierfür ist, dass bei der
idealen Güte von ein irreversiblem Prozess ausgegangen wird, dies ist in der
Realtät jedoch nicht erfüllt, da nicht vollständig isoliert werdn kann. Daher kommmt es zu einem Wärmeausgleich zwischen
beiden Reservoirs, welcher dem gewünschtem Wärmetransport entgegenwirkt. Ein
weiterer Grund liegt in der Isolierung des Systems. In der Theorie wird
ein geschlossenes System betrachtet, in der Realtät liegt dieses jedoch nicht
vor. Die verwendete Isolierung kann die Effekte der Umgebungstemperatur zwar
verringern, sie bleiben jedoch bestehen und verschlechtern die Güte. Bei
Betrachtung des Massendurchsätze ergeben sich geringe Werte von ca.
$0.2-\SI{0.3}{\gram}$. Mögliche Ursachen hierfür sind ein schlechter Wirkungsgrad
des Kompressors und ein schlechter Wert der Verdampfungswärme für Dichlordifluormethan.
Aufgrund mangelnder Literaturwerte bezüglich der Verdampfungswerte kann dies
jedoch nur vermutet werden. Der schlechte Wirkungsgrad wird zuletzt in den
bestimmten Leistungen deutlich. Die in Tabelle \ref{tab:tabn} bestimmten mechanischen Leistungen betragen alle ca.
$\SI{5}{\percent}$ der am Netzanschluss abgelesenen Leistungwerten. Hauptursachen
 hierfür sind der Kompressor sowie die mangelnde Isolierung.
