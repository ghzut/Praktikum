
\section{Theorie}
\label{sec:Theorie}
In einem geschlossenen System mit zwei Wärmeresservours unterschiedlicher Temperatur geht
  Wärmeenergie immer vom wärmeren zum kälteren Resservour über bis ein
  Temperaturausgleich zwischen beiden erfolgt ist. Um den Prozess
  umzukehren wird zusätzliche Energie benötigt. Ein Gerät, welches dies leistet,
  heißt Wärmepumpe. Für eine solche Wärmepumpe gilt nach dem 1. Hauptsatz der Wärmelehre Bedingungen:
  \begin{equation}
    Q1 = Q2 + A\label{eq:Q1}
  \end{equation}
mit der an das wärmere Resservour abgegeben Energie $Q1$ der dem kälteren
Resservour entnommenen Energie $Q2$ und der zusätzlich benötigten Arbeit $A$.
Mithilfe dieses Zusammenhangs folgt die Güteziffer:
\begin{equation}
  v = \frac{Q1}{A}\label{eq:v1}
  \end{equation}
  Für eine idealisierte Wärmepumpe folgt damit die maximal erreichbare Güte
  \begin{equation}
    v = \frac{T1}{T1-T2}\label{eq:vid}
  \end{equation}
  mit den Temperaturen $T1$ für das wärmere und $T2$ für das kältere Resservour.
  Für eine realistische Wärmepumpe ist die Güte daher kleiner als dieser Wert.
Die Wärmepumpe besteht aus aus zwei Reservours zweier unterschiedlicher
Temperaturen. Zum Wärmetransport wird ein reales Gas verwendet. Dieses verdampft,
wenn es Wärme im kälteren resservour aufnimmt und verflüssigt sich wieder, wenn es
die Wärme wieder abgibt.



Zur Bestimmung der realen Güteziffer gilt:
\begin{equation}
  \frac{\Delta Q1}{\Delta t} = \left( m_1 c_\text{w} + m_\text{k}c_\text{k}\right) \frac{\Delta T_1}{\Delta t}
\end{equation}
mit der Wärmekapazität $m_1 c_\text{w}$ des Wassers und der Wärmekapazität $m_k c_\text{k}$
der Verbindungsröhre und der des Eimers.
