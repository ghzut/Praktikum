
\section{Durchführung}
\label{sec:Durchführung}

Die Wasserbehälter, der im Aufbau \ref{sec:Aufbau} dargestellten Wärmepumpe, werden jeweils mit $\SI{3}{\liter}$ Wasser gefüllt. Dazu werden Messkolben verwendet. Die gefüllten Wasserbehälter werden dann an den, an der Wärmepumpe vorgesehenen, Stellen fixiert. Es ist dabei darauf zu achten, dass keine größere Lücken zwischen dem Wasserbehälter und dem, an der Wärmepumpe befestigten, Deckel existieren. Um eventuelle Lücken zu schließen, können Keile unter dem Wasserbehälter geschoben werden. Bevor die Wärmepumpe in Betrieb genommen wird, werden das erste Mal Messwerte entnommen. Nachdem die Wärmepumpe in Betrieb genommen wurde, wobei noch darauf zu achten ist, dass sich die Rührmotoren wirklich drehen, werden die nächsten Messwerte jeweils nach einem Zeitintervall von $\SI{1}{\minute}$ entnommen. Entnommen werden jeweils die Temperaturen $T_1$ und $T_2$ an den digitalen Thermometern, der Druck $p_\text{a}$ und $p_\text{b}$ an den Manometer, wobei auf den abgelesenen Wert $\SI{1}{\bar}$ addiert werden muss und die Leistungsaufnahme des Kompressors am Wattmeter. Die Messung wird fortgesetzt bis am Reservoir 1 eine Temperatur von $\SI{50}{\degreeCelsius}$ erreicht wird.