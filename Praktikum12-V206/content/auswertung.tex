\section{Auswertung}
\label{sec:Auswertung}
Die Graphen wurden sowohl mit Matplotlib als auch Numpy erstellt. Die
 Fehlerrechnung wurde mithilfe von Uncertainties durchgeführt.





 Zur Approximation des Temperaturverlaufes beim 1. Resservour wurde ein
 quadratischer Fit der Form $y = Ax+Bx+C$ verwendet. Eine nichtlineare
 Ausgleichsrechnung der Form $y = Ax+B+C$ liefert mit den im Minutentakt
 aufgenommenenen Temparaturdaten von Resservour 1 aus Tabelle JHJHJHJK folgende Parameter:

$A = 7.17334$

Mithilfe dieser folgen die Differentialquotienten $\text{d}T_1$ und $\text{d}T_2$
%Tabelle
Auf dieser Basis folgt die reale Güte mithilfe von Formel \cite{eq:Q1T1} und \cite{eq:v}:
%nochne Tablle
