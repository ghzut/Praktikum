\section{Auswertung}
\label{sec:Auswertung}
Die Graphen wurden sowohl mit Matplotlib\cite{matplotlib} als auch Numpy\cite{numpy} erstellt. Die
 Fehlerrechnung wurde mithilfe von Uncertainties\cite{uncertainties} durchgeführt.
 \begin{figure}
 	\centering
 	\caption{Die Temperaturverläufe des Wassers in den Wärmereservoirs 1 und 2 in Abhängigkeit der Zeit.}
 	\includegraphics[width=\linewidth-70pt,height=\textheight-70pt,keepaspectratio]{build/Temperaturen.pdf}
 	\label{fig:Graph1}
 \end{figure}
 \begin{figure}
 	\centering
 	\caption{Der Temperaturverlauf im ersten Reservoir und seine Approximation durch ein Polynom 2. Grades in Abhängigkeit der Zeit.}
 	\includegraphics[width=\linewidth-70pt,height=\textheight-70pt,keepaspectratio]{build/T1.pdf}
 	\label{fig:Graph1}
 \end{figure}
 \begin{figure}
 	\centering
 	\caption{Der Temperaturverlauf im zweiten Reservoir und seine Approximation durch ein Polynom 2. Grades in Abhängigkeit der Zeit.}
 	\includegraphics[width=\linewidth-70pt,height=\textheight-70pt,keepaspectratio]{build/T2.pdf}
 	\label{fig:Graph1}
 \end{figure}
 \begin{figure}
 	\centering
 	\caption{Der Verlauf der Dampfdruckkurve des Transportgases Dichluorfluormethan in Abhängigkeit der reziproken Temperatur.}
 	\includegraphics[width=\linewidth-70pt,height=\textheight-70pt,keepaspectratio]{build/Dampdruck.pdf}
 	\label{fig:Graph1}
 \end{figure}




 Zur Approximation des Temperaturverlaufes im 1. Resservour wurde ein
 quadratischer Fit der Form $y = At²+Bt+C$ verwendet. Eine nichtlineare
 Ausgleichsrechnung der Form $y = At²+Bt+C$ mittels scipy\cite{scipy} liefert mit den im Minutentakt
 aufgenommenenen Temparaturdaten von Resservour 1 aus Tabelle \ref{tab:daten}
 die Parameter:
 \begin{displaymath}
\begin{aligned}
 A = \num{-2(1)e-6}\\
 B = \num{3.0(1)e-2}\\
 C = \num{292.9(4)}
 \end{aligned}
 \end{displaymath}
 Für den Verlauf der Temperatur im zweiten Reservoir folgt dementsprechend:
 \begin{displaymath}
\begin{aligned}
 A = \num{6(2)e-6}\\
 B = \num{-3.1(3)e-2}\\
 C = \num{297.4(7)}
 \end{aligned}
 \end{displaymath}

 \begin{table}
   \centering
   \input{build/tabges.tex}\label{tab:daten}
 \end{table}

 \begin{table}
   \centering
   \input{build/tabv.tex}\label{tab:Guete}
 \end{table}

 \begin{table}
   \centering
   \input{build/taba.tex}\label{tab:Ableitungen}
 \end{table}

Durch Differentation folgen die Differenzenquotienten $\frac{\text{d}T_1}{\text{d}t}$ und $\frac{\text{d}T_2}{\text{d}t}$ der Form $y = Ax+B$ in Tabelle \ref{tab:dT}.

%Tabelle
Mithilfe von Formel \ref{eq:T1Q1} und \ref{eq:v} folgt die reale Güte an den in Tabelle \ref{tab:datv}:


Mithilfe einer linearen Ausgleichsrechnung der Form $y=Ax+b$ mittels scipy folgt
$L$ aus den Daten von $\frac{1}{T2}$ und $\ln(Pb)$ mithilfe von Formel \ref{eq:L}
 \begin{equation}
   L = R*\frac{\text{d}y}{\text{d}x}\label{eq:L}
 \end{equation}
\begin{table}
  \centering
  \input{build/tabm.tex}\label{tab:massen}
\end{table}

\begin{table}
  \centering
  \input{build/tabn.tex}\label{tab:Leistung}
\end{table}
