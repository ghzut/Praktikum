
\section{Theorie}
\label{sec:Theorie}

%Abb.1
Eine LC-Kette beschreibt eine Verkettung von n in Reihe geschalteten LC-Gliedern. Jeder dieser Glieder hat die
 Eigenschaften eines Tiefpasses, welcher Wechselspannungen mit geringen Frequenzen passieren lässt, hochfrequente jedoch verschwinden. Zum anderen kann die LC-Kette auch als System gekoppelter Schwingungssysteme betrachtet werden, welches viele Eigenschwingungen besitzt.


\subsection{Die stationäre LC-Kette}
1. Fall :Die einfache LC-Kette

%abb. 2
Aus den kirchhoffschen Regeln folgt die BewegungsDGL:
\begin{equation}
- \omega ^2 C U_n + \frac{1}{L} \left( -U_{n-1} + 2U_ n -U{n-1} \right) = 0
\end{equation}
Mit einem komplexen E-Ansatz erhält man:
\begin{equation}
\omega ^2 = \frac{2}{LC}(1-cos\theta)
\end{equation} 
Dieser Ausruck wird als Dispersionsrelation beschrieben und stellt die Abhängigkeit
