
\section{Theorie}
\label{sec:Theorie}

%Abb.1
Eine LC-Kette beschreibt eine Verkettung von n in Reihe geschalteten LC-Gliedern.
 Jeder dieser Glieder hat die Eigenschaften eines Tiefpasses, welcher
  Wechselspannungen mit geringen Frequenzen passieren lässt, hochfrequente jedoch
   verschwinden. Zum anderen kann die LC-Kette auch als System gekoppelter
    Schwingungssysteme betrachtet werden, welches viele Eigenschwingungen besitzt.


\subsection{Die stationäre LC-Kette}
1. Fall :Die einfache $LC$-Kette

%abb. 2
Aus den kirchhoffschen Regeln folgt die BewegungsDGL:
\begin{equation}
- \omega ^2 C U_n + \frac{1}{L} \left( -U_{n-1} + 2U_ n -U{n-1} \right) = 0
\end{equation}
Mit einem komplexen E-Ansatz erhält man:
\begin{equation}
\omega ^2 = \frac{1}{2LC\pi^2}(1-cos\theta)
\end{equation}
Dieser Ausruck wird als Dispersionsrelation beschrieben und stellt die Änderung
 der Phase pro Kettenglied in Abhängigkeit der Frequenz dar. Anhand der Formel lässt sich erkennen,
  dass der Frequenzbereich indem Schwingungen auftreten begrenzt ist. Für Frequenzen >
 $\frac{1}{\sqrt{2LC\pi^2}}$ bilden sich daher keine Schwingungen aus.\\\\

 2. Fall: Die $C_1C_2$-Kette:
 %Abb 4
 Aus den kirchhoffschen Regeln folgen die BewegungsDGLn:
 \begin{equation}
   -\omega^2 C_1 U_{2n+1} + \frac{1}{L} \left( -U_{2n} + 2U_{2n+1} - U_{2n+2} \right) = 0
 \end{equation}
 und
 \begin{equation}
   -\omega^2 C_2 U_{2n} + \frac{1}{L} \left( -U_{2n-1} + 2U_{2n+1} - U_{2n+1} \right) = 0
 \end{equation}

Es folgt für die auftretende Dispersion:
\begin{equation}
  f_{1/2}^2 = \frac{1}{4\pi^2L}\left(\frac{1}{C_1}+\frac{1}{C_2}\right) \pm \frac{1}{4\pi^2L}\sqrt{\left(\frac{1}{C_1}+\frac{1}{C_2} \right)^2 - \frac{4 sin^2\theta}{C_1C_2}}
\end{equation}
Es zeigt sich, dass die $C_1C_2$-Kette aufgrund der positiven und negativen Versionen der Wurzel
zwei Frequenzbereiche besitzt in denen Schwingungen auftreten. Die Verläufe der Unteren und Oberen
 Grenzfrequenz werden akustischer bzw. optischer Ast genannt. Nähert man den Verlauf mit $sin \theta \approx \theta$ erhält man eine Kurvenverlauf, welcher dann ca so aussieht.
%Biild einfügen zwei Äaste
