
\section{Theorie}
\label{sec:Theorie}

\subsection{Die allgemeinen Eigenschaften einer LC-Kette}
Eine LC-Kette ist eine Verkettung von N in Reihe geschalteten LC-Gliedern.
 Jedes dieser Glieder hat die Eigenschaften eines Tiefpasses und lässt
  Wechselspannungen mit geringen Frequenzen passieren, während es hochfrequente jedoch
   blockiert. Erhöht man die Anzahl der Kettenglieder und lässt
   diese $\to \infty$ laufen, erhält man eine unendliche LC-Kette. Letztere
    beschreibt ein Ersatzschaltbild einer elektrischen Leitung, im Fall der
     LC-Kette einer verlustfreien Leitung. Es zeigt sich, dass eine unendliche
      Kette einen spezifischen Gesamtwiderstand, den Wellenwiderstand $Z$, besitzt.
      Dieser ist reell und nur von der Generatorfrequenz abhängig. Für ihn gilt:
      \begin{equation}
        Z(f) = \sqrt{\frac{L}{C}} \cdot \frac{1}{2\pi \sqrt{1-0,25\omega² LC}}\text{.}
      \end{equation}
 Zum anderen kann die $LC$-Kette auch als System gekoppelter
    Schwingungen betrachtet werden, welches eine Vielzahl von Eigenschwingungen besitzt.
      Aus diesem Grund können Wellen und Wellenpakete auf ihr übertragen werden.
	Eine Welle besitzt eine Phasengeschwindigkeit, mit der sie sich im Medium ausbreitet. Für sie gilt im Fall der $LC$-Kette:
\begin{equation}
v_{ph} = \frac{\omega}{\theta}\text{.}
\end{equation}
	$\theta$ beschreibt dabei den Phasenunterschied pro Kettenglied. Auf ihn wird später näher eingegangen.
     Da ein Wellenpaket aus einer Menge unterschiedlicher Einzelwellen besteht
      und jede eine eigene Phasengeschwindigkeit besitzt, kommt es zu einer Verzerrung
  der Gestalt des Wellenpaketes. Daraus folgt, dass die auftretende Phasenverschiebung Frequenzabhängig ist.
   Dieser Effekt heißt Dispersion.
     \begin{figure}[H]
       \centering
       \includegraphics[width=\linewidth-200pt,height=\textheight-200pt,keepaspectratio]{content/Grafiken/LCKette.png}
       \caption{Die unendliche LC-Kette \cite{V356}.}
       \label{fig:LC-Kette}
     \end{figure}

\subsection{Die Eigenschaften einer unendlichen LC-Kette}

Im folgenden wird weiter auf die bereits gennanten Effekte eingegangen. Zunächst
 wird die einfache $LC$-Kette thematisiert. Diese besitzt nur Kondensatoren der
 gleichen Kapazität $C$.

Aus den kirchhoffschen Regeln folgt die Bewegungsgleichung:
\begin{equation}
- \omega ^2 C U_n + \frac{1}{L} \left( -U_{n-1} + 2U_ n -U{n-1} \right) = 0\text{.}
\end{equation}
Mit einem komplexen e-Ansatz gelangt man zu:
\begin{equation}
 f ^2 = \frac{1}{2LC\pi^2}(1-\cos\theta)
\end{equation}
Dieser Ausdruck wird als Dispersionsrelation bezeichnet und stellt die Frequenz in Abhängigkeit
 der Phasendifferenz pro Kettenglied dar. Anhand der Formel lässt sich erkennen,
  dass der Frequenzbereich indem Schwingungen auftreten begrenzt ist. Daher bilden sich
   für Frequenzen $f \geq \frac{1}{\sqrt{LC\pi^2}}$ keine Schwingungen aus.\\\\

 Nun wird die $LC_1C_2$-Kette näher betracht. Sie besitzt Kondensatoren mit zwei
  verschiedenen Kapazitäten $C_1$ und $C_2$, welche im Wechsel verbaut sind.
 Aus den kirchhoffschen Regeln folgen die Bewegungsgleichungen:
 \begin{equation}
   -\omega^2 C_1 U_{2n+1} + \frac{1}{L} \left( -U_{2n} + 2U_{2n+1} - U_{2n+2} \right) = 0\text{.}
 \end{equation}
 und
 \begin{equation}
   -\omega^2 C_2 U_{2n} + \frac{1}{L} \left( -U_{2n-1} + 2U_{2n+1} - U_{2n+1} \right) = 0\text{.}
 \end{equation}
Es folgt für die auftretende Dispersion:
\begin{equation}
  f_{1/2}^2 = \frac{1}{4\pi^2L}\left(\frac{1}{C_1}+\frac{1}{C_2}\right) \pm \frac{1}{4\pi^2L}\sqrt{\left(\frac{1}{C_1}+\frac{1}{C_2} \right)^2 - \frac{4 \sin^2\theta}{C_1C_2}}\text{.}
\end{equation}
Es zeigt sich, dass die $LC_1C_2$-Kette aufgrund der positiven und negativen Wurzel
zwei Frequenzbereiche besitzt, in denen Schwingungen auftreten. Die Verläufe der
 negativen und positiven Wurzel werden akustischer bzw. optischer Ast genannt.
 Der Kurvenverlauf hat die in \ref{fig:thetaf} dargestellte Form.
 \begin{figure}[H]
   \centering
   \includegraphics[width=\linewidth-100pt,height=\textheight-100pt,keepaspectratio]{content/Grafiken/Dispersionskurven.png}
   \caption{Die Frequenz in Abhängigkeit der Phasendifferenz \cite{V356}.}
   \label{fig:thetaf}
 \end{figure}
 Es zeigt sich, dass zwischen beiden Ästen eine Lücke existiert,
  in denen keine Schwingungen auftreten, da die untere Grenzfrequenz des
   optischen Astes größer ist als die obere Grenzfrequenz des akustischen Astes.
\subsection{stehende Wellen}
Breitet sich eine Wellenfront auf einem im Ort beschränkten Träger aus, trifft diese
nach endlicher Zeit auf den Rand des Trägers. An diesem kommt es zur Reflexion
 der Wellenfront und zur Überlagerung der ankommenden und der reflektierten
  Wellenfronten. Es bildet sich eine stehende Welle auf dem Träger aus. Diese besitzt
   Minima und Maxima welche örtlich und zeitlich konstant bleiben. Die Minima heißen Knoten,
    die Orte der Amplitudenmaxima heißen Bäuchen. Je höher die Frequenz der
     Welle ist, desto mehr Bäuche und Knoten bilden sich, da ihre Anzahl abhängig
      von der Wellenlänge $\lambda$ ist. Passt genau ein n-faches der halben Wellenlänge
       auf die Länge des Trägers, wird von der n-ten Eigenschwingung gesprochen. Es existieren
    folgende Spezialfälle:
    \begin{itemize}
      \item Besitzt der Träger zwei offene Enden, kommt es zu einer vollständigen
       Reflexion der Welle ohne einen Phasensprung. An beiden Enden bilden sich
        Bäuche aus. Die Schwingungsbäuche liegen bei $n \cdot \frac{\lambda}{2}$,
        die Knoten bei $2n+1 \cdot \frac{\lambda}{4}$.

\item Besitzt der Träger zwei feste Enden, kommt es zu einer vollständigen
 Reflexion der Welle mit einem Phasensprung von $\pi$. An beiden Enden bilden sich Knoten aus.
 Die Schwingungsbäuche liegen bei $2n+1 \cdot \frac{\lambda}{4}$,
 die Knoten bei $n \cdot \frac{\lambda}{2}$.
      \end{itemize}


\subsection{Die Eigenschaften einer endlichen LC-Kette}
Liegt eine LC-Kette endlicher Länge vor, bzw. wurden beide Abschlusswiderstände
 nicht nach Formel (1) der Frequenz des Wechselstromes entsprechend angepasst,
  kommt es zur Reflexion ankommender Wellen. Wenn dies geschieht bilden sich auf der
   $LC$-Kette stehende Wellen aus. Aus den kirchhoffschen Regeln folgt das Verhältnis:
  \begin{equation}
    \frac{U_{ref}}{U_{ein}} = \frac{R-Z}{R+Z}\text{.}
  \end{equation}
  Es gelten die Spezialfälle:\\
\begin{itemize}
  \item Die LC-Kette besitzt ein offenes Ende, $R$ beträgt also $\infty$. Die $LC$-Kette
   besitzt dann zwei offene Enden.\\

  \item Die LC-Kette ist kurzgeschlossen, $R$ beträgt daher 0. Die $LC$-Kette besitzt zwei
  feste Enden.\\

 \item Werden beide Abschlusswiderstände auf den Wellenwiderstand der $LC$-Kette eingestellt,
  kommt es zu keiner Reflexion. Die $LC$-Kette verhält sich dann wie eine unendliche Kette.
\end{itemize}
