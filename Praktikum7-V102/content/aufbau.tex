\section{Aufbau}
\label{sec:Aufbau}

\begin{figure}[H]
         \centering
         \includegraphics[width=\linewidth-150pt,height=\textheight-150pt,keepaspectratio]{content/Bilder/Aufbau.png}
         \caption{Messapperatur zur Bestimmung des Schubmoduls $G$ und des magnetischen Moments $m$ \cite{V102}.}
         \label{fig:Aufbau}
       \end{figure}

        Ein Draht des zu betrachtenden Materials wird oben mit einer Klemmschraube fixiert. Mithilfe eines
           Justierrades lässt er sich dann nurnoch drehen. Zur Besseren
            Betrachtung der auftretenden Drehschwingung wird ein Spiegel nach einer Länge $L_1$ eingesetzt. Der
             Spiegel wird über ein weiteres, kürzeres Stück Draht der Länge $L_2$ mit
             einer Haltekonstruktion verbunden, in welcher sich die bereits
              beschriebene Kugel befindet.
               Um die auftretende Schwingung zu dämpfen ist eine Dämpfkonstruktion
                unter der Kugelhalterung angebracht. Zur Vereinfachung der späteren Messung des magnetischen Moments ist ein Stabmagnet
               bereits in die Kugel integriert und ein Helmholtzspulenpaar um die Kugel aufgestellt. Dieses wird
                 mit einer Konstantstromquelle betrieben.

                 Nun zur genaueren Betrachtung der Zeitmessung mithilfe des Drehspiegels:

                 \begin{figure}[H]
                          \centering
                          \includegraphics[width=\linewidth-150pt,height=\textheight-150pt,keepaspectratio]{content/Bilder/Drehspiegel.png}
                          \caption{Aufbau zur genauen Bestimmung der Periodendauer mithilfe eines Drehspiegels \cite{V102}.}
                          \label{fig:Drehspiegel}
                        \end{figure}

Zur besseren Betrachtung des Drehwinkels $\varphi$ wird das Licht einer Glühbirne
 mithilfe eines Spaltes und einer Sammellinse fokussiert und so ausgerichtet,
  dass es auf den Drehspiegel trifft. Während $\varphi$ sich ändert fährt der
   Lichtstrahl einen Pfad mit zwei Umkehrpunkten ab. Auf diesem Pfad ist eine
    Photodiode montiert, welche ein Signal abgibt, wenn der Lichtstrahl
     diese überstreicht. Die Impulse werden über eine digitale Schaltung verarbeitet und
      gelangen zu einer elektronischen Uhr, mit welcher schließlich die Periodendauer $T$ gemessen wird.

      Die digitale Schaltung ist wie folgt aufgebaut:
