\section{Aufbau}
\label{sec:Aufbau}

\begin{figure}[H]
         \centering
         \includegraphics[width=\linewidth-150pt,height=\textheight-150pt,keepaspectratio]{content/Bilder/Aufbau.png}
         \caption{Messapperatur zur Bestimmung des Schubmoduls $G$ und des magnetischen Moments $m$ \cite{V102}.}
         \label{fig:Aufbau}
       \end{figure}

       Die Messapperatur zur Bestimmung von $G$ und $m$ setzt wie bereits in der
        Theorie erläutert auf Drehschwingungen, welche durch das Trägheitsmoment
        eines Gewichtes verursacht werden. Zur besseren Betrachtung des Drehwinkels
          $\varphi$ wird ein Spiegel verwendet. Ein Draht des zu betrachtenden
           Materials wird oben in eine Klemmschraube fixiert. Mithilfe eines
           Justierrades lässt er sich dann nurnoch drehen. Zur Besseren
            Betrachtung wird nach einer Länge $L_1$ ein Spiegel eingesetzt. Der
             Spiegel wird über ein weiteres, kürzeres Stück Draht der Länge $L_2$ mit
             der Haltekonstruktion verbunden, in welcher sich die bereits
              beschriebene Kugel befindet. Zur Vereinfachung ist der Stabmagnet
               bereits in der Kugel integriert. Für die spätere Messung von $m$
                ist ein Helmholtzspulenpaar um die Kugel aufgestellt. Es wird
                 mit einer Konstantstromquelle betrieben.

                 Nun zur genaueren Betrachtung der Zeitmessung mithilfe des Drehspiegels:

                 \begin{figure}[H]
                          \centering
                          \includegraphics[width=\linewidth-150pt,height=\textheight-150pt,keepaspectratio]{content/Bilder/Drehspiegel.png}
                          \caption{Aufbau zur genauen Bestimmung der Periodendauer mithilfe eines Drehspiegels \cite{V102}.}
                          \label{fig:Drehspiegel}
                        \end{figure}

Zur besseren Betrachtung des Drehwinkels $\varphi$ wird das Licht einer Glühbirne
 mithilfe eines Spaltes und einer Sammellinse fokussiert und so ausgerichtet,
  dass es auf den Drehspiegel trifft. Während $\varphi$ sich ändert fährt der
   Lichtstrahl einen Pfad mit zwei Umkehrpunkten ab. Auf diesem Pfad ist ein
    Lichtdetektor montiert, welcher ein signal abgibt, wenn der Lichtstrahl
     in trifft. Die Impulse werden über eine digitale Schaltung verarbeitet und
      gelangen zu einer elektronischen Uhr, welche schließlich die Periodendauer $T$ misst.

      Die digitale Schaltung ist wie folgt aufgebaut:
