
\section{Auswertung}
\label{sec:Auswertung}
Die Fehlerrechnung wurde mit Unterstützung von Uncertainties \cite{uncertainties} durchgeführt.
Folgende Werte waren vor der Messung bekannt.\\
Die Masse der Kugel beträgt:
\begin{displaymath}
	m_\text{K} = \SI{0.512(20)}{\kilogram}\text{,}
\end{displaymath}
der Durchmesser der Kugel:
\begin{displaymath}
	D_\text{K} = \SI{0.0508(4)}{\meter}\text{,}
\end{displaymath}
das Trägheitsmoment der Kugelhalterung:  
\begin{displaymath}
	\theta_\text{KH} = \SI{2.25(6)e-6}{\kilogram\meter\squared}\text{,}
\end{displaymath}
die Windungszahl der Helmholzspule:
\begin{displaymath}
	N = \num{390}
\end{displaymath}
und der Wert des Elastizitätsmoduls:  
\begin{displaymath}
	E = \SI{2.100(5)e11}{\pascal}\text{.}
\end{displaymath}




\subsection{Bestimmung des Schubmoduls G mithilfe einer Torsionsschwingung}
\input{build/tabDraht.tex}
Aus den Werten aus Tabelle 1 berechnet sich für den Mittelwert des Drahtdurchmessers:
\begin{displaymath}
	D_\text{D} = \SI{0.1723(20)e-3}{\meter}\text{.}
\end{displaymath}
\input{build/tabn.tex}
Aus den Werten aus Tabelle 2 ergibt sich für den Mittelwert der Periodendauer:
\begin{displaymath}
	T = \SI{20.0006(11)}{\second}\text{.}
\end{displaymath}
Mithilfe der Masse der Kugel und deren Durchmesser lässt sich mit Formel (8) das Trägheitsmoment der Kugel berechnen:
\begin{displaymath}
	\theta_{\text{K}} = \SI{0.132(6)e-3}{\kilogram\meter\squared}\text{.}
\end{displaymath}
Nun lässt sich der Schubmodul mit Formel (7) berechnen:
\begin{displaymath}
	G = \SI{8.9(6)e10}{\pascal}\text{.}
\end{displaymath}
Aus dem Schub- und Elastizitätsmoduls lassen sich der Kompressionsmodul und die poissonsche Querkontraktionszahl berechnen. Aus Formel (1) folgt:
\begin{displaymath}
	\mu = \num{0.17(7)}
\end{displaymath}
und aus Formel (2):
\begin{displaymath}
	Q = \SI{1.07(24)e11}{\pascal}\text{.}
\end{displaymath}

NUR HINWEIS\\
(Handelt sich vermutlich um Stahl \cite[624\psq]{TaschenbuchPhysik}) 210 GPa
\cite{EinführungMechanik}  	79,3 GPa






\subsection{Bestimmung des magnetischen Momentes des Permanentmagneten in der Kugel}
\begin{figure}[H]
	\centering
	\caption{Die magnetische Flussdichte $B$ gegen $T_m ^{-2}$ aufgetragen.}
	\includegraphics[width=\linewidth-70pt,height=\textheight-70pt,keepaspectratio]{build/helmholzspule.pdf}
	\label{fig:grad3}
\end{figure}
\input{build/tabh.tex}
Abbildung 3 wurde mit Matplotlib \cite{matplotlib} als auch NumPy \cite{numpy} erstellt.
Eine lineare Ausgleichsrechnung der Form $y = a \cdot x + b$ mittels SciPy \cite{scipy} der Wertepaare $\left(\frac{1}{T_\text{m}^2}, B\right)$ aus Tabelle 3 liefert nach Formel (12) für das magnetische Moment des Stabmagneten im innerem der Kugel:
\begin{displaymath}
	m' = \frac{4 \pi ^2 (\theta_k + \theta_{\text{KH}})}{a} = \SI{6.74(29)e-2}{\ampere\meter\squared}\text{,}
\end{displaymath}
und zusätzlich:
\begin{displaymath} 
	-\frac{D}{m'} = b =  \SI{-1.75(9)e-4}{\tesla}\text{.}
\end{displaymath}
Mit Formel (3) folgt:
\begin{displaymath} 
	G =  \SI{8.0(6)e10}{\pascal}\text{.}
\end{displaymath}
Hieraus folgt mit den Formeln (1) und (2):
\begin{displaymath}
	\mu = \num{0.32(10)}
\end{displaymath}
und
\begin{displaymath} 
	Q =  \SI{1.9(11)e11}{\pascal}\text{.}
\end{displaymath}

NUR HINWEIS\\
Werte passen besser trotz höherer Standartabweichung (systematischer Fehler bei erster Messung vielleicht unterschiedliche trägheitsmomente um verschiedene achsen vermutlich erste messung mist)





\subsection{Bestimmung der Horizontalkomponente des magnetischen Flusses des Erdmagnetfeldes}
\input{build/tabe.tex}
Aus den Werten aus Tabelle 4 ergibt sich für den Mittelwert der Periodendauer:
\begin{displaymath}
	T_\text{m} = \SI{20.0088(35)}{\second}\text{.}
\end{displaymath}
Die Horizontalkomponente des magnetischen Flusses des Erdmagnetfeldes kann durch einsetzen von $x=\frac{1}{T_\text{m}^2}$ in die Gerade der Form $y = a \cdot x +b$ mit den zuvor bestimmten Parametern $a$ und $b$ bestimmt werden. Nach Formel (12) folgt:
\begin{displaymath}
	B_\text{E} = y = \SI{ 2.2(8)e-5}{\tesla}\text{.}
\end{displaymath}

NUR HINWEIS\\
Vergleich mit theorie wert \SI{20}{\micro\tesla} \cite{Erdmagnetfeld}.
