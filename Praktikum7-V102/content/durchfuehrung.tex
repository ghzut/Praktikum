
\section{Durchführung}
\label{sec:Durchführung}

\begin{itemize}
   \item Als Vorbereitung werden die Längen $L_1$ und $L_2$ gemessen. Die Dicke des
   Drahtes wird an mindestens 5 Stellen mit einer Mikrometerschraube gemessen,
   eine davon muss unterhalb des Spiegels sein.
\end{itemize}
\renewcommand{\labelenumi}{\alph{enumi})}

\begin{enumerate}
  \item Zunächst wird die Periodendauer der Drehschwingung ohne Einfluss
   eines Magnetfeldes gemessen. Um dies zu erreichen bleiben die Helmholtzspulen
    ausgeschaltet und die Kugel wird so ausgerichtet, dass der Magnet parallel zum Draht
    liegt. Es wird näherungsweise angenommen, dass das Erdmagnetfeld parallel zur Erde
    verläuft. Der Lichtstrahl wird mithilfe des Justierrades so justiert,
     dass er ein Stück neben die Photodiode fällt und mit Spalt und Sammellinse
      fokussiert. Um im Geltungsbereich des hookschen Gesetzes zu bleiben muss
       die Schwingungsamplitude klein bleiben. Hierzu kann die Dämpfkonstruktion
        verwendet werden.Es werden die Zeiten von 10 Periodendauern aufgenommen.

        \item Nun wird die Periodendauer der Drehschwingung im
         homogenen Magnetfeld bestimmt werden. Dazu werden die Helmholtzspulen
          eingeschaltet und der Magnet parallel zu ihren Feldlinien ausgerichet.
           Auch hier ist auf kleine Auslenkungen zu achten.
          Die Stromstärke $I$, mit welcher die Helmholtzspulen versorgt werden,
           wird nun stückweise erhöht. In $\SI{0.1}{\ampere}$Schritten von
           $\SI{0.1}{\ampere}$ bis $\SI{1.0}{\ampere}$ werden nun jeweils 
Zeiten von 5 Periodendauern notiert.

      \item Schlussendlich wird die Periodendauer der Drehschwingung im Erdmagnetfeld gemessen.
       Hierzu wird die Kugel so gedreht, dass der Magnet parallel zum Erdmagnet
        ausgerichtet ist. Die Ausrichtung der Feldlinien kann an einem Kompass
         nachgesehen werden. Es ist wieder auf kleine Auslenkungen zu achten.
          Auch hier werden die Zeiten von 10 Periodendauern notiert.



\end{enumerate}
