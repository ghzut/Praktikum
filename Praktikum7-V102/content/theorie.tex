
\section{Theorie}
\label{sec:Theorie}

\subsection{das Einwirken von Kräften auf deformierbare Körper}
In der Mechanik wird bei Kräften auf einen Körper zwischen zwei Wirkungsweisen unterschieden.
 Es existieren Kräfte, welche auf jedes Volumenelement des Körpers wirken. Sie
  führen entweder zu einer Translation oder einer Rotation des Körpers.
  Andererseits gibt es auch die Kräfte, welche auf jedes Oberflächenelement des Körpers wirken.
    Sie führen zu einer Gestaltsänderung oder einer Volumenänderung des Körpers.
    Eine Kraft der letzten Kategorie wird als Spannung bezeichnet und bezieht sich im Normalfall
     auf die Kraft pro Fläche. Sie lässt sich in Komponenten zerlegen.
     Die Komponente, welche senkrecht auf der Oberfläche des Körpers steht
      wird als Normalspannung $\sigma$ bezeichnet,
       die paralele Komponente als Tangentialspannung $\tau$.
       Verformt sich der Körper wieder zurück in seine ursprüngliche Form,
        wird von einer elastischen Deformation gesprochen.
        Ist die auf den Körper einwirkende Spannung hinreichend klein,
         besteht zwischen den wirkenden Spannungskomponenten
          und den relativen Länge- und Volumenänderungen ein linearer
           Zusammenhang. Dieser wird Hooksches Gesetz genannt.



\subsection{Die elastischen Konstanten eines isotropen Körpers}
Unter einem Isotropen Körper werden Materialien verstanden,
 deren elastische Konstanten richtungsunabhängig sind. Zu ihnen gehören
  unter anderem Polykristalline Metalle. Theoretisch lässt sich ihr Verhalten
  mit 2 Konstanten beschreiben. Aus Bequemlichkeitsgründen werden jedoch 4 Konstanten verwendet:
  \begin{itemize}
    \item Der Schubmodul $G$ bezeichnet die Elastizität, bezogen auf eine Änderung
     der Gestalt. Seine Größe ist $\frac{\text{Kraft}}{\text{Fläche}}$.

    \item Der Kompressionsmodul $Q$ bezeichnet die Elastizität, bezogen auf
     eine Änderung des Volumens. Auch dessen Größe ist $\frac{\text{Kraft}}{\text{Fläche}}$.

    \item Der Elastizitätsmodul $E$ bezeichnet die relative Längenänderung in
     Spannungsrichtung eines Körpers unter Einfluss einer Normalspannung.

    \item Die Poissonsche Querkontraktionszahl $\mu$ bezeichnet die relative
     Längenänderung senkrecht zur Spannungsrichtung eines Körpers unter
      Einfluss einer Normalspannung.
\end{itemize}
Die 4 Größen sind nicht unabhängig voneinander und es bestehen folgende Beziehungen zwischen ihnen:
\begin{equation}
  E = 2G(\mu + 1)
\end{equation}
\begin{equation}
  E = 3(1-2\mu)Q
\end{equation}
Es lässt sich erkennen, dass $E$ und $G$ bestimmt werden müssen, während $Q$ und $\mu$
 anschließend mit ihnen berechnet werden können. Ein weiterer Effekt ist die
  elastische Nachwirkung. Sie tritt auf, wenn sich ein deformierbarer Körper
   unter Belastung nicht sofort in seinen Endzustand versetzt, oder sich bei
    Aufhebung der Spannung nicht sofort wieder in den Ursprungszustand zurückversetzt.

 \subsection{Bestimmung des Schubmoduls G mithilfe einer Torsion}
 Soll der Schubmodul eines Materials bestimmt werden, wird dieses in die Form eines Drahtes gebracht.
  Dieser wird an einer Seite festgehalten und an der anderen verdreht. Über das auftretende
   Drehmoment $M$ gelangt man anschließend an $G$ mithilfe der Richtgröße $D$. Für letztere gilt:
   \begin{equation}
     D = \frac{\pi GR⁴}{2L}
   \end{equation}
Aufgrund der elastischen Nachwirkungen kann es jedoch zu Fehlern kommen.
Um diesen entgegenzuwirken wird der Draht zu einer Drehschwingung angeregt, bei denen das Problem nicht auftritt.
 Hierzu wird eine schwingfähige Masse an den Körper gehängt, welche ein Trägheitsmoment $\theta$ besitzt.
Unter Vernachlässigung des Energieverlustes lautet dann die Bewegungsgleichung:
\begin{equation}
  D\varphi + \theta\frac{\text{d}²\varphi}{\text{d}t²} = 0
\end{equation}
Sie wird gelöst durch:
\begin{equation}
\varphi(t) = \varphi_0 \cos\frac{2\pi}{T}
\end{equation}
Der Winkel $\varphi(t)$ bezeichnet den zeitabhängigen Torsionswinkel,
 $\varphi_\text{0}$ den Winkel der Anfangstorsion. Die Periodendauer T lässt sich mit
\begin{equation}
  T = 2\pi\sqrt{\frac{\theta}{D}}
\end{equation}
ausdrücken. Für $G$ folgt dann:
\begin{equation}
  G = 8\pi \frac{L}{T²R⁴}(\theta_\text{K} + \theta_\text{H}) \text{ mit } \theta = \theta_\text{K} + \theta_\text{H}
\end{equation}
Die Trägheitsmomente $\theta_\text{K}$ und $\theta_\text{H}$ gehören zur Schwungmasse,
 bzw. zur Massenhalterung, mit welcher die Masse am Draht befestigt ist.
  Ist die Schwungmasse Kugelförmig folgt für $\theta_\text{K}$:
  \begin{equation}
    \theta_\text{K} = \frac{2}{5} m_\text{K} R_\text{K}^2
    \end{equation}
% Im Fall einer Kugelförmigen, angehängten Masse mit $R_\text{k} \gg R$ folgt für $G$ schließlich
%\begin{equation}
%G = \frac{16}{5} \pi \frac{m_\text{k}R^2_\text{k}L}{T^2R⁴}
%\end{equation}
%mit dem Radius der Kugel $R_\text{k}$, der Kugelmasse $m_\text{k}$, dem Stabradius $R$ und der Stablänge $L$.

\subsection{Messung des magnetischen Momentes eines Permanentmagneten}
Unter einem magnetischen Moment $m'$ wird das Produkt der Polstärke $p$
 und dem Abstandsvektor der beiden Pole $a$ in Richtung des Südpols verstanden.
  Wird der Stabmagnet nun in ein homogenes $B$-Feld gesetzt,
   erhält er ein Drehmoment $M_{\text{mag}}$
Zur Bestimmung des magnetischen Momentes wird wieder ein Torsionsschwingung wie
 beim Schubmodul verwendet, diesmal jedoch unter Einfluss eines $B$-Feldes.
  Aufgrund des $B$-Feldes kommt es zu einer veränderten Periodendauer $T_\text{m}$.  Wenn
  der Magnet im Inneren der Kugel untergebracht wird und parallel zur Feldrichtung
  liegt, gilt für den Drehwinkel \varphi die Bewegungsgleichung:
  \begin{equation}
    m'B\sin\varphi + D\varphi + \theta\frac{\text{d}^2\varphi}{\text{d}t²} = 0
  \end{equation}
  Für kleine Ablenkungen kann $\sin \varphi$ mit $\varphi$ genähert werden und die DGL wird durch
  \begin{equation}
    \varphi(t) = \varphi_\text{0} \cos\frac{2\pi}{T_\text{m}}t
  \end{equation}
  gelöst. Für die neue Periodendauer $T_\text{m'}$ gilt:
  \begin{equation}
    T_\text{m} = 2\pi\sqrt{\frac{\theta}{m'B + D}}\text{.}
  \end{equation}
Das magnetische Moment $m'$ berechnet sich schlussendlich mit:
\begin{equation}
  m' = \frac{4\pi²\theta}{BT_\text{m'}^2} - \frac{D}{B}
\end{equation}
