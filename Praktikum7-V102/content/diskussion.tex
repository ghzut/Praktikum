
\section{Diskussion}
\label{sec:Diskussion}
Im Folgenden wird das Material anhand der verschiedenen Konstanten bestimmt.
 Für eine eindeutige Aussage reichen jedoch die Ergebnisse von $E$ und $G$. Die beiden anderen Konstanten lassen
 sich mit ihnen ermitteln.
Aufgrund des gegebenen Wertes des Elastizitätsmoduls wird vermutet, dass es sich um einen Stahldraht handelt.
Der gegebene Wert von $E = \SI{210 \pm 0.5}{\giga\pascal}$ passt zum Literaturwert von Stahl. Dieser liegt ebenfalls bei
 ca. $\SI{210}{\giga\pascal}$\cite[624\psq]{TaschenbuchPhysik}.
Beim Vergleich der in 5.1 und 5.2 ermittelten Werte des Schubmoduls, fällt auf
 ,dass beide Erwartungswerte nicht in der jeweils anderen Sigma-Umgebung liegen.
 Da es sich beim Material des Drahtes dem Elastizitätsmoduls nach um Stahl handelt
, scheint der in 5.2 bestimmte Erwartungswert besser zu sein. Dieser liegt mit $\SI{80.0 \pm 6.0}{\giga\pascal}$ in Nähe
  des Literaturwertes des Schubmoduls von Stahl, welcher ca. $\SI{79.3}{\giga\pascal}$\cite{EinführungMechanik} beträgt.
   %Der gegebene Wert des Elastizitätsmoduls von ca. $\SI{210}{\giga\pascal}$
    %bestätigt diese Vermutung, da der Literaturwert von $E$ bei Stahl ebenfalls
     %bei ca. $\SI{210}{\giga\pascal}$\cite[624\psq]{TaschenbuchPhysik} liegt.
      Dieses Ergebnis erscheint zwar recht eindeutig, aufgrund der hohen
      Unsicherheit des bestimmten $G$ ist sie jedoch nicht sonderlich aussagekräftig.
      Das Ergebnis von 5.1 ist hingegen ca. $G =\SI{89.0 \pm 6.0}{\giga\pascal}$.
      Dieses weicht stark vom Literaturwert von $G$ von Stahl ab. Es zeigt aber auch
      keine Tendenz, welche auf ein anderes isotropes Material schließen lassen würde.
      Da alle Konstanten in 5.1 kleiner als ihre Gegenstücke aus 5.2 sind, lässt dies auf einen systematischen Fehler schließen.
     Es existieren mehrere mögliche Fehlerquellen, welche die schlechteren
      Ergebnisse aus 5.1 erklären können. Eine mögliche Fehlerquelle bildet das
       Trägheitsmoment der Kugel. Diese ist schon aufgrund des integrierten
        Magneten keine reine Vollkugel und eine Änderung der Drehachse führt daher
         automatisch zu einem veränderten Trägheitsmoment. Eine weitere
          Fehlerquelle liegt im Verlassen des Bereiches, indem das Hooksche
           Gesetz gültig ist. Falls dieser Bereich aufgrund einer zu hohen
            Auslenkung verlassen wurde, kommt es ebenfalls zu Fehlern.
            Das in 5.3 bestimmte Ergebnis der Stärke des Erdmagnetfeldes mit $\SI{43 \pm 17}{\micro\tesla}$ zeigt Abweichungen von ca. $\SI{100}{\percent}$
            im Vergleich zum Literaturwert. Dieser liegt bei ca. $\SI{20}{\micro\tesla}$\cite{Erdmagnetfeld}. Dies lässt sich nur damit erklären, dass ein grober systematischer Fehler vorliegt,
             welcher sich jedoch nicht in den elastischen Konstanten widerspiegelt.
              Falls die gegebene Anzahl der Spulenwicklungen jedoch für beide Spulen
               zusammen gelte, wäre auch der bestimmte Erwartungswert im Rahmen der Messunsicherheit.
 Auch hier ist der Wert jedoch aufgrund der hohen Unsicherheit nicht aussagekräftig.

%NUR HINWEIS\\

%(Handelt sich vermutlich um Stahl \cite[624\psq]{TaschenbuchPhysik}) 210 GPa
%\cite{EinführungMechanik}  	79,3 GPa

%NUR HINWEIS\\teil 2
%Werte passen besser trotz höherer Standartabweichung (systematischer Fehler bei erster Messung vielleicht unterschiedliche trägheitsmomente um verschiedene achsen vermutlich erste messung mist)

%NUR HINWEIS\\
%Vergleich mit theorie wert \SI{20}{\micro\tesla} \cite{Erdmagnetfeld}.
