
\section{Diskussion}
\label{sec:Diskussion}


Beim Vergleich der in 5.1 und 5.2 ermittelten Werte des Schubmoduls, fällt auf
 das beide Erwartungswerte nicht in der jeweils anderen sigma-Umgebung liegen.
 Da es sich beim Material des Drahtes allerdings um ein isotropes Metall handeln
  soll, scheint der in 5.2 bestimmte Wert besser zu sein. Dieser liegt in Nähe
  des Literaturwertes des Schubmoduls von Stahl, welcher ca. $\SI{79.3}{\giga\pascal}$\cite{EinführungMechanik} beträgt.
   Der gegebene Wert des Elastizitätsmoduls von ca. $\SI{210}{\giga\pascal}$
    bestätigt diese Vermutung, da der Literaturwert von $E$ bei Stahl ebenfalls
     bei ca. $\SI{210}{\giga\pascal}$\cite[624\psq]{TaschenbuchPhysik} liegt.
     Auch die Erwartungswerte der errechneten Größen $Q$ und $\mu$ passen mit
      $\SI{190}{\giga\pascal}$ und bzw. 0,32 im Rahmen ihrer Fehlers zu den Literaturwerten von Stahl.
      Diese liegen bei $Q = \SI{170}{\giga\pascal}$\cite{QundMu} bzw 0,28\cite{QundMu}.
      Die Ergebnisse von 5.1 sind hingegen ca. $G =\SI{89}{\giga\pascal}$, $Q = \SI{107}{\giga\pascal}$ und $\mu = 0,17$.
      Selbst in Anbetracht der teilweise noch größeren Unsicherheiten weichen
      diese stark von den Literaturwerten von Stahl ab. Sie zeigen aber auch
      keine Tendenz, welche auf ein anderes Metall schließen lassen würde.
     Es gibt existieren mehrere mögliche Fehlerquellen, welche die schlechteren
      Ergebnisse aus 5.1 erklären können. Eine mögliche Fehlerquelle bildet das
       Trägheitsmoment der Kugel. Diese ist schon aufgrund des integrierten
        Magneten keine reine Vollkugel und eine Änderung der Drehachse führt daher
         automatisch zu einem veränderten Trägheitsmoment. Eine weitere
          Fehlerquelle liegt im Verlassen des Bereiches, indem das Hooksche
           Gesetz gültig ist. Falls dieser Bereich aufgrund einer zu hohen
            Auslenkung verlassen wurde, kommt es ebenfalls zu Fehlern.

%NUR HINWEIS\\

%(Handelt sich vermutlich um Stahl \cite[624\psq]{TaschenbuchPhysik}) 210 GPa
%\cite{EinführungMechanik}  	79,3 GPa

NUR HINWEIS\\teil 2
Werte passen besser trotz höherer Standartabweichung (systematischer Fehler bei erster Messung vielleicht unterschiedliche trägheitsmomente um verschiedene achsen vermutlich erste messung mist)

NUR HINWEIS\\
Vergleich mit theorie wert \SI{20}{\micro\tesla} \cite{Erdmagnetfeld}.
