
\section{Diskussion}
\label{sec:Diskussion}

Die Schwankungen in den Graphen in den Abbildungen 5 und 6 lassen sich durch die Bildung von stehenden Wellen erklären, da der Wellenwiderstand nur für die Frequenz \\$f=0$ eingestellt wurde. Die errechneten Werte weichen von den aus dem Graphen abgelesenen Werten leicht (unter \SI{6}{\percent}) ab. Diese Abweichung kann durch die hohe Unsicherheit in der Wahl der Stelle an der die Grenzfrequenz erreicht wird begründet werden. 

Die Messwerte mit den zugehörigen Phasenverschiebungen in dem Graphen in Abbildung 7 zeigen keine systematischen Abweichungen zu der Theoriekurve und lassen sich allein mit der Messunsicherheit begründen. Jedoch ist zu beachten, dass die Phasenverschiebungen bestimmt wurden unter der Annahme, dass bei $f=0$ keine Phasenverschiebung auftritt und die Phasenverschiebung zwischen der Spannung vor und hinter der $LC$-Kette bei der jeweils nächst höheren gemessenen Frequenz um $\pi$ zunimmt.

Die Messwerte mit den zugehörigen Phasenverschiebungen in dem Graphen in Abbildung 8 zeigen eine kleine systematische Abweichung nach unten (durchschnittlich \SI{1.1}{\percent}). Diese Abweichung könnte durch eine Abweichung des gegebenen Wertes für $C_2$ zustande kommen. Auch hier ist zu beachten, dass die Phasenverschiebungen unter denselben Annahmen wie zuvor bestimmt wurden. Für die zwei Wertepaare im optischem Zweig wurde jedoch angenommen, dass die Phasenverschiebung bei $f=\SI{66.3}{\kilo\hertz}$ verschwinde und für kleinere Frequenzen um $\pi$ zunimmt und 3 Frequenzen übergangen wurden. Das angenommen wird, dass 3 Frequenzen übergangen wurden an denen die Phasenverschiebung zwischen der Spannung vor und hinter der $LC_1C_2$-Kette das Vielfache von $\pi$ beträgt macht diese zwei Wertepaare nicht aussagekräftig.

Die Eigenschwingungen sollten bei den Frequenzen auftreten, bei denen die Phasenverschiebung zwischen der Spannung vor und hinter der $LC$-Kette das Vielfache von $\pi$ beträgt. Vergleicht man die ermittelten Frequenzen aus Versuchsteil b) und c) bestätigt sich dies. Jedoch lässt sich eine in b) ermittelte Frequenz ($\SI{45.0}{\kilo\hertz}$) keiner Frequenz aus c) zuordnen. Dies lässt vermuten, dass eine Frequenz in c) übergangen wurde.
Bei der Betrachtung des Graphen in Abbildung 9 lässt sich keine systematischen Abweichungen erkennen und liegen somit im Rahmen der Messungenauigkeit.

Die Messwerte in der Tabelle 4 wurden mit 14 statt 16 $LC$-Gliedern mit anderen Kapazitäten und Induktivitäten ($L = \SI{1.217}{\milli\henry}$ und $C = \SI{20.13}{\nano\farad}$) von der anderen Gruppe aufgenommen.
Die SI-Präfixe der Einheiten von $U_1$ und $U_2$ waren nicht bekannt.

In den Graphen in den Abbildungen 10 und 11 weichen die Werte von dem Fit ab. Diese Abweichung wird vermutlich durch eine hohe Messunsicherheit verursacht.

In dem letztem Graphen weichen die Werte nicht von dem Fit ab.







	

