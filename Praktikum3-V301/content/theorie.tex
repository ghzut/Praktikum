\section{Theorie}
\label{sec:Theorie}

Ein Gerät, welches eine konstante Leistung über
einen endlichen Zeitraum erzeugen kann, beschreibt eine Spannungsquelle. Es wird von einer Leerlaufspannung $ U_0 $
an den Ausgangsklemmen gesprochen, falls der Spannungsquelle kein Strom entnommen wird.
Sobald es zu einer Leistungsabnahme durch einen äußeren Widerstand $ R_a $ kommt,
sinkt die an den Klemmen gemessene Spannung, die "Klemmenspannung", unter den von $ U_0 $. Erklärt wird
dies durch einen Eigenwiderstand der Spannungsquelle. In der Theorie wird die reale
Spannungsquelle durch eine ideale Spannungsquelle in Reihe mit einem Widerstand $ R_i$
ersetzt.
\\
\\
{\huge Abbildung 1 aus Skript einfügen}
\\
\\
Aus dem zweiten kirchhoffschen Gesetz folgt dann gemäß Abb. 1
\\
\\
$ U_0 = I R_i + I R_a $ bzw. $ U_k = I R_a = U_0-IR_i$
\\
\\
Zur Messung der Leerlaufspannung wird deshalb ein hochohmiges Voltmeter verwendet,
sodass der Strom $sI$ gegen $0$ läuft und somit $ U_0 \approx U_k$ gilt.

