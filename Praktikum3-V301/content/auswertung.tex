\section{Auswertung}
\label{sec:Auswertung}

\subsection{Gleichspannungsquelle}
\begin{figure}
	\centering
	\includegraphics[width=\linewidth,height=\textheight,keepaspectratio]{Gleichstrom.png}
	\caption{Gleichstrom.}
	\label{fig:Gleichstrom}
\end{figure}
\begin{table}
	\centering
	\caption{Ergebnisse der Gleichstrommessung ohne Gegenspannung}
	\label{tab:Gleichstrom}
	\sisetup{table-format=1.2}
	\begin{tabular}{S S }
		\toprule
		{$I/$mA} & {$U_k/$V} \\
		\midrule
		23.0 & 1.20 \\
		25.0 & 1.20 \\
		27.0 & 1.15 \\
		30.0 & 1.10 \\
		33.0 & 1.10 \\
		36.5 & 0.95 \\
		41.0 & 0.90 \\
		46.5 & 0.80 \\
		55.0 & 0.65 \\
		69.5 & 0.40 \\
		93.0 & 0.10 \\
		\bottomrule
	\end{tabular}
\end{table}





\subsection{Sinusspannungsquelle}

\begin{figure}
  \centering
  \includegraphics[width=\linewidth,height=\textheight,keepaspectratio]{Sinus.png}
  \caption{Sinus.}
  \label{fig:Sinus}
\end{figure}
\begin{table}
	\centering
	\caption{Ergebnisse der Sinusspannungsmessung}
	\label{tab:Sinus}
	\sisetup{table-format=1.2}
	\begin{tabular}{S S }
		\toprule
		{$I/$mA} & {$U_k/$V} \\
		\midrule
		0.16 & 0.99 \\
		0.17 & 0.98 \\
		0.18 & 0.97 \\
		0.20 & 0.96 \\
		0.23 & 0.95 \\
		0.26 & 0.92 \\
		0.32 & 0.88 \\
		0.42 & 0.82 \\
		0.56 & 0.72 \\
		0.83 & 0.55 \\
		1.15 & 0.33 \\
		\bottomrule
	\end{tabular}
\end{table}


\subsection{Rechteckspannungsquelle}

\begin{figure}
	\centering
	\includegraphics[width=\linewidth,height=\textheight,keepaspectratio]{Rechteck.png}
	\caption{Rechteck.}
	\label{fig:Rechteck}
\end{figure}
\begin{table}
	\centering
	\caption{Rechteck}
	\label{tab:Rechteck}
	\sisetup{table-format=1.2}
	\begin{tabular}{S S }
		\toprule
		{$I/mA$} & {$U/V$} \\
		\midrule
		0.8 & 0.52 \\
		0.9 & 0.51 \\
		1.0 & 0.5 \\
		1.2 & 0.49 \\
		1.5 & 0.47 \\
		1.8 & 0.45 \\
		2.1 & 0.42 \\
		2.5 & 0.39 \\
		3.2 & 0.31 \\
		4.3 & 0.25 \\
		5.1 & 0.19 \\
		\bottomrule
	\end{tabular}
\end{table}


\subsection{ Mit Gegenspannung}
\begin{figure}
	\centering
	\includegraphics[width=\linewidth,height=\textheight,keepaspectratio]{GleichstromR.png}
	\caption{GleichstromR.}
	\label{fig:GleichstromR}
\end{figure}
\begin{table}
	\centering
	\caption{Ergebnisse der Gleichspannungsmessung mit Gegenspannung}
	\label{tab:GleichstromR}
	\sisetup{table-format=1.2}
	\begin{tabular}{S S }
		\toprule
		{$I/$mA} & {$U_k/$V} \\
		\midrule
		32.0 & 2.10 \\
		33.5 & 2.20 \\
		36.5 & 2.25 \\
		40.0 & 2.30 \\
		44.0 & 2.38 \\
		48.0 & 2.45 \\
		55.5 & 2.58 \\
		64.0 & 2.70 \\
		74.0 & 2.90 \\
		89.0 & 3.25 \\
		130.0 & 3.70 \\
		\bottomrule
	\end{tabular}
\end{table}


Soweit so gut ... .
