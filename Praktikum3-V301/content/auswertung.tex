\section{Auswertung}
\label{sec:Auswertung}



\subsection{Die U/I Graphen der verschiedenen Spannungsquellen}

\subsubsection{Gleichspannungsquelle}
\begin{figure}[H]
	\centering
	\caption{}
	\includegraphics[width=\linewidth-150pt,height=\textheight-150pt,keepaspectratio]{Gleichstrom.png}
	\label{fig:Gleichstrom}
\end{figure}
\begin{table}
	\centering
	\caption{Ergebnisse der Gleichstrommessung ohne Gegenspannung}
	\label{tab:Gleichstrom}
	\sisetup{table-format=1.2}
	\begin{tabular}{S S }
		\toprule
		{$I/$mA} & {$U_k/$V} \\
		\midrule
		23.0 & 1.20 \\
		25.0 & 1.20 \\
		27.0 & 1.15 \\
		30.0 & 1.10 \\
		33.0 & 1.10 \\
		36.5 & 0.95 \\
		41.0 & 0.90 \\
		46.5 & 0.80 \\
		55.0 & 0.65 \\
		69.5 & 0.40 \\
		93.0 & 0.10 \\
		\bottomrule
	\end{tabular}
\end{table}


\newpage
\subsubsection{Gleichspannungsquelle mit Gegenspannung}

\begin{figure}[H]
	\centering
	\caption{}
	\includegraphics[width=\linewidth-150pt,height=\textheight-150pt,keepaspectratio]{GleichstromR.png}
	\label{fig:GleichstromR}
\end{figure}
\begin{table}
	\centering
	\caption{Ergebnisse der Gleichspannungsmessung mit Gegenspannung}
	\label{tab:GleichstromR}
	\sisetup{table-format=1.2}
	\begin{tabular}{S S }
		\toprule
		{$I/$mA} & {$U_k/$V} \\
		\midrule
		32.0 & 2.10 \\
		33.5 & 2.20 \\
		36.5 & 2.25 \\
		40.0 & 2.30 \\
		44.0 & 2.38 \\
		48.0 & 2.45 \\
		55.5 & 2.58 \\
		64.0 & 2.70 \\
		74.0 & 2.90 \\
		89.0 & 3.25 \\
		130.0 & 3.70 \\
		\bottomrule
	\end{tabular}
\end{table}


\newpage
\subsubsection{Sinus-Spannungsquelle}

\begin{figure}[H]
  \centering
  \caption{}
  \includegraphics[width=\linewidth-150pt,height=\textheight-150pt,keepaspectratio]{Sinus.png}
  \label{fig:Sinus}
\end{figure}
\begin{table}
	\centering
	\caption{Ergebnisse der Sinusspannungsmessung}
	\label{tab:Sinus}
	\sisetup{table-format=1.2}
	\begin{tabular}{S S }
		\toprule
		{$I/$mA} & {$U_k/$V} \\
		\midrule
		0.16 & 0.99 \\
		0.17 & 0.98 \\
		0.18 & 0.97 \\
		0.20 & 0.96 \\
		0.23 & 0.95 \\
		0.26 & 0.92 \\
		0.32 & 0.88 \\
		0.42 & 0.82 \\
		0.56 & 0.72 \\
		0.83 & 0.55 \\
		1.15 & 0.33 \\
		\bottomrule
	\end{tabular}
\end{table}


\newpage
\subsubsection{Rechteck-Spannungsquelle}

\begin{figure}[H]
	\centering
	\caption{}
	\includegraphics[width=\linewidth-150pt,height=\textheight-150pt,keepaspectratio]{Rechteck.png}
	\label{fig:Rechteck}
\end{figure}
\begin{table}
	\centering
	\caption{Rechteck}
	\label{tab:Rechteck}
	\sisetup{table-format=1.2}
	\begin{tabular}{S S }
		\toprule
		{$I/mA$} & {$U/V$} \\
		\midrule
		0.8 & 0.52 \\
		0.9 & 0.51 \\
		1.0 & 0.5 \\
		1.2 & 0.49 \\
		1.5 & 0.47 \\
		1.8 & 0.45 \\
		2.1 & 0.42 \\
		2.5 & 0.39 \\
		3.2 & 0.31 \\
		4.3 & 0.25 \\
		5.1 & 0.19 \\
		\bottomrule
	\end{tabular}
\end{table}

\newpage
\subsection{Die lineare Ausgleichsrechnung zu den Messwerten}
Für die Ausgleichsrechnung wurden die folgenden Formeln verwendet:

\begin{equation}
a=\frac{ \overline{xy} - \overline{x} \cdot \overline{y}}{\overline{x^2} - \overline{x}^2}
\end{equation}
\begin{equation}
b=\frac{\overline{x^2} \cdot \overline{y} - \overline{x} \cdot \overline{xy}}{\overline{x^2} - \overline{x}^2}
\end{equation}
\begin{equation}
\sigma_y=\sqrt{\frac{1}{N-2}\sum\limits_{i=1}^{N}(y_i - a x_i - b)^2}
\end{equation}
\begin{equation}
\sigma_a=\sqrt{\frac{\sigma_y^2}{N ( \overline{x^2} - \overline{x} ^2 )}}
\end{equation}
\begin{equation}
\sigma_b=\sqrt{\frac{\sigma_y^2 \cdot \overline{x}}{N ( \overline{x^2} - \overline{x} ^2 )}}
\end{equation}



























































\newpage
\subsection{Der systematische Fehler einer unmittelbaren $U_0$ Bestimmung mit einem Voltmeter}
Der systematische Fehler einer unmittelbaren Leerlaufspannungsermittlung folgt aus der Näherung, das
der Innenwiderstand des Voltmeters unendlich betrage. Da dies bei einem realen Voltmeter nicht der
Fall ist kommt es zu einem systematischen Fehler.
Aus den Formeln ?????????????????? ergibt sich sofort
\begin{equation}
U_0 = \frac{U_k}{R_v}(R_v + R_i)
\end{equation}
sowie dem absoluten systematischen Fehler
\begin{equation}
\Delta U = \frac{U_k}{R_v}R_i \text.
\end{equation}
Es ergibt sich ein systematischer Fehler von $2,64 \cdot 10^-6 V$





\subsection{der systematische Fehler einer falschen Platzierung des Voltmeters}
Auch die falsche Positionierung des Voltmeters im Stromkreis führt zu einem
systematischen Fehler.Fügt man das Voltmeter wie in Abbildung 2 an der Stelle H
ein, also hinter dem Amperemeter, müsste man dessen kleinen, aber durchaus existenten
Innenwiderstand miteinbeziehen.






\begin{figure}[H]
	\centering
	\caption{}
	\includegraphics[width=\linewidth-150pt,height=\textheight-150pt,keepaspectratio]{GleichstromRe.png}
	\label{fig:GleichstromLeistung}
\end{figure}
