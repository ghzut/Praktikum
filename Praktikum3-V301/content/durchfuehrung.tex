\section{Durchführung}
\label{sec:Durchführung}

\renewcommand{\labelenumi}{\alph{enumi})}
\begin{enumerate}
    \item Es wird die Leerlauflaufspannung einer Monozelle mit einem Spannungsmesser ermittelt.
    Es wird der Eingangswiderstand des Voltmeters notiert.

    \item Es wird die Klemmenspannung $U_k$ in Abhängigkeit des Belastungsstroms I mithilfe
    der Schaltung aus Abb. 2 gemessen.
    Hierzu wird der Belastungswiderstand $R_a$ im Bereich von $ 0-50 \Omega$ variiert.

    \item Es wird eine Gegenspannung wie in Abb. 3 an die Monozelle angelegt, welche ca. 2V
    grösser als $U_0$ ist. Der Strom fliesst nun in umgekehrter Richtung und es gilt:\\
    \begin{equation}
	    U_k = U_0 + IR_i
    \end{equation}
	    
    
    Es wird wiederum $U_k$ in Abhängigkeit von I gemessen.

    \item Es soll die Messreihe aus b) nochmals mit dem Sinus bzw. dem
    Rechteckausgang eines RC-Generators durchgeführt werden. Für den Variationsbereich
    von $R_a$ soll gelten:\\
    1 V Sinusausgang: $R_a \in [0,1 - 5 k\Omega]$\\
    1 V Rechteckausgang: $R_a \in [20 -250 \Omega]$\\
    Es ist zu beachten das die Messgeräte nur für einen engen Frequenzbereich geeicht sind.
\end{enumerate}
