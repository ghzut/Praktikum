\section{Diskussion}
\label{sec:Diskussion}


Zunächst werden die $U_0/I $-Kurven der einzelnen Spannungsquellen sowie ihre Auswertung diskutiert.

Bei der Betrachtung der $U_0 /I$-Kurve der Gleichspannungsquelle ohne Gegenspannung
fällt auf das die Messpunkte alle relativ nahe zur dieser liegen, was auf eine gute
Messung schliessen lässt. Der einzelne Ausreisser bei (36,5/0,95) kann als Messfehler
eingestuft werden.
 Im Detail liegt die Standardabweichung der Ausgleichsgeradensteigung bei ca. 3\%. Das durch den
 y-Achsenabschnitt bestimmte $U_0$ besitzt einen Fehler von ca. 1,3\%, wobei ein möglicher
 systematischer Fehler vernachlässigt wird.
 Die y-Standardabweichung der Messwerte zur Ausgleichsgraden liegt nur bei ca. 1,9\%.

Die Messung der Gleichstromspannungsquelle mit Gegenspannung bestätigt die $U_0$ und $R_i$ Ergebnisse
der ersten, obrigen Messreihe. Sowohl $U_0$ als auch $R_i$ liegen bei beiden Messreihen
 relativ nahe beieinander.Die Messreihe selbst kann als genügend bezeichnet werden.
 Zwar liegen alle Standardabweichungen leicht über denen der ersten Messung,
 sind aber im Rahmen.
 Der letzten beiden Messwerte können als Messfehler klassifiziert werden.
 Somit können die Messwerte als aussagekräftig bezeichnet werden.
\\
 Bei der Rechteckspannungsquelle zeigt der Graph nochmals dasselbe
Bild, wieder recht gute Werte,wieder keinerlei Tendenzen, wieder ein Ausreisser, der
zu vernachlässigen ist. Die Sinusmessung scheint zuletzt die beste zu sein, da hier auch
kein Ausreisser mehr zu erkennen ist. Im ganzen scheinen die Graphen also auf eine
gute Messung schliessen.
