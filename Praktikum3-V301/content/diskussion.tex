\section{Diskussion}
\label{sec:Diskussion}


Zunächst werden die $U_0/I $ Kurven der einzelnen Spannungsquellen sowie ihre Auswertung diskutiert.

Bei der Betrachtung der U/I Kurve der Gleichspannungsquelle ohne Gegenspannung
fällt auf das die Messpunkte alle relativ nahe zur dieser liegen, was auf eine gute
Messung schliessen lässt. Es lassen sich auch keine
Tendenzen zu einer Seite erkennen. Dies lässt folgern das keine systematischen
aufgetreten sind. Der einzelne Ausreisser bei (36,5/0,95) kann als Zufallsfehler
eingestuft werden.
 Im Detail liegt die Abweichung der Ausgleichsgeradensteigung bei ca 3\%, das durch den
 Y-Achsenabschnitt bestimmte $U_0$ besitzt einen Fehler von ca 1,3\%. Auch
 die Y-Standardabweichung der Messwerte zur Ausgleichsgraden liegt nur bei ca 1,9\%.






Dies spiegelt sich auch in den ERgebnissen wieder Bei der Messung mit Gegenspannung zeigt sich ein ähnliches Bild.
 Auch hier lassen sich keine Tendenzen erkennen, die Differenzen der Messwerte zeugen
jedoch von einer noch genaueren Messung. Der letzte Messwert lässt sich wieder als
Ausreisser deuten. Bei der Rechteckspannungsquelle zeigt der Graph nochmals dasselbe
Bild, wieder recht gute Werte,wieder keinerlei Tendenzen, wieder ein Ausreisser, der
zu vernachlässigen ist. Die Sinusmessung scheint zuletzt die beste zu sein, da hier auch
kein Ausreisser mehr zu erkennen ist. Im ganzen scheinen die Graphen also auf eine
gute Messung schliessen.
