\section{Diskussion}
\label{sec:Diskussion}


Zunächst werden die $U_0/I $-Kurven der einzelnen Spannungsquellen sowie ihre Auswertung diskutiert.

Die Lineare Regression der $U_0 /I$ der Messergebnisse zur Gleichspannungsquelle ohne Gegenspannung lässt eine gute Messung vermuten,
 betrachtet man den einzelnen Ausreißer bei (36,5| 0,95) als Messfehler.
 Im Detail liegt die Standardabweichung der Ausgleichsgeradensteigung bei ca. $\SI{3}{\%}$. Das durch den
 y-Achsenabschnitt bestimmte $U_0$ besitzt eine Standardabweichung von ca. $\SI{1,3}{\%}$, wobei ein möglicher
 systematischer Fehler vernachlässigt wird.
 Die y-Standardabweichung der Messwerte zur Ausgleichsgeraden liegt nur bei ca. $\SI{1,9}{\%}$,
  was die Gerade aussagekräftig macht.


Die Messung der Gleichstromspannungsquelle mit Gegenspannung bekräftigt die Ergebnisse von $U_0$ und $R_i$ aus
der Messreihe von Tabelle 2. Zumindest bei den bestimmten $R_i$ kommt es zu Überschneidungen der zugehörigen
$\sigma$ Umgebungen.
 Die Standardabweichungen der zweiten Messreihe liegen zwar leicht über denen der ersten Messung,
 sind aber im Rahmen.
 Die großen Abweichungen der letzten beiden Messwerte können als Messfehler klassifiziert werden.
 Eine mögliche Fehlerquelle für die Ausreißer liegt beim in der Schaltung verbauten Taster,
  welcher während der Messung lang gedrückt werden musste. Eine leichte Druckabnahme auf den Schalter
  konnte somit zu einer Verfälschung der gemessenen Werte führen.
  Im Gegenzug zu $R_i$ zeigt sich, dass die aus den beiden Ausgleichsrechnungen bestimmten Werte für $U_0$
 nicht in den angegebenen $\sigma$-Umgebungen zueinander liegen.
 Dies lässt sich durch systematische Fehler der Messgeräte bzw. der Schaltungskomponenten erklären.

Auch die Lineare Regression der Messergebnisse der Rechteckspannung lässt eine gute Messung vermuten.
 Da  der Schalter sowohl bei Rechteck- als auch bei der folgenden Sinusspannung nicht mehr gedrückt werden musste,
entfällt diese Fehlerquelle. Dies zeigt sich auch leicht in den bestimmten Unsicherheiten,
welche im allgemeinen etwas kleiner sind. So liegt die Unsicherheit des neuen $R_i$ bei
unter $\SI{2,5}{\%}$, die der Effektivspannung sogar bei unter $\SI{1}{\%}$. Auch hier ist der Ausreißer wahrscheinlich
ein einfacher Messfehler.

Zum Schluss liefert die Messung mit der Sinusspannungsquelle die besten Ergebnissen.
Die Standardabweichung von $R_i$ liegt bei unter $\SI{1}{\%}$, die von $U_0$ sogar bei unter $\SI{0,5}{\%}$.
 Es existieren auch keine größeren Ausreißer.

Der in 4.3 bestimmte systematische Fehler ist mit $\SI{2,64}{\mu\volt}$ in Relation zu den bestimmten $U_0$ viel zu gering
um einen wichtigen Unterschied in Rechnungen zu erzeugen. Somit ist die Näherung, welche bei direkten Messung der Leerlaufspannung mit einem Voltmeter durchgeführt wurde,
 vollkommen gerechtfertigt. Die durch $R_i$ hervorgerufene Unsicherheit ist sogar nochmals
 zwei Größenordnungen kleiner, womit auch sie vernachlässigt werden kann.

 Zu einem anderen Ergebnis kommt man beim systematischen Fehler in 4.4, welcher durch falsche Positionierung des Voltmeters hervorgerufen wird.
Hier ist der Innenwiderstand eines üblichen Amperemeters zwar klein, würde aber nacher vollständig vom ermittelten Innenwiderstand $R_i$ abgezogen werden müssen.
 Bei einem typischen Wert von $\SI{1,8}{\ohm}$ bei einer Reichweite von $\SI{400}{\milli\ampere}$ [2] würde dies sich deutlich in dem bestimmten $R_i$ bemerkbar machen.

Die Leistungkurve weist im allgemeinen wie bereits beschrieben keine Anzeichen eines systematischen Fehlers auf. Allein im Bereich bis zum Theoriemaximum
könnte eine leichte Tendenz nach unten vorliegen. Die Messergebnisse liegen daher hauptsächlich
 im Bereich der Messungenauigkeit. Der hintere Bereich zeigt keinerlei Tendenzen.
 Eine Messung mit mehr Messproben könnte diese Unklarheit lösen.

\section{Literatur}
\begin{enumerate}[{[}1{]}]
	\item TU Dortmund. Versuchsanleitung zum Versuch 301 - EMK und Innenwiderstand von Spannungsquellen. 2014.\\
  http://129.217.224.2/HOMEPAGE/PHYSIKER/BACHELOR/AP/SKRIPT/V301.pdf (abgerufen am 28.11.2016)


	\item Newcombe, Chuck: Can you live with the burden? (o.J.).\\ http://www.fluke.com/fluke/uses/comunidad/fluke-
  news-plus/articlecategories/electrical/burdenvoltage (abgerufen am 15.11.2016)
\end{enumerate}
