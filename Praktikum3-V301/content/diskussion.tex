\section{Diskussion}
\label{sec:Diskussion}


Zunächst werden die $U_0/I $-Kurven der einzelnen Spannungsquellen sowie ihre Auswertung diskutiert.

Bei der Betrachtung der $U_0 /I$-Kurve der Gleichspannungsquelle ohne Gegenspannung
fällt auf, dass die Messpunkte alle relativ nahe zur dieser liegen, was auf eine gute
Messung schließen lässt. Der einzelne Ausreißer bei (36,5| 0,95) kann als Messfehler
eingestuft werden.
 Im Detail liegt die Standardabweichung der Ausgleichsgeradensteigung bei ca. 3\%. Das durch den
 y-Achsenabschnitt bestimmte $U_0$ besitzt eine Standartabweichung von ca. 1,3\%, wobei ein möglicher
 systematischer Fehler vernachlässigt wird.
 Die y-Standardabweichung der Messwerte zur Ausgleichsgeraden liegt nur bei ca. 1,9\%,
  was die Gerade aussagekräftig macht.


Die Messung der Gleichstromspannungsquelle mit Gegenspannung bestätigt die $U_0$ und $R_i$ Ergebnisse
der Messreihe aus Tabelle 2. Sowohl $U_0$ als auch $R_i$ liegen bei beiden Messreihen
 relativ nahe beieinander.  Die Messreihe selbst ist ausreichend gut gelungen.
 Zwar liegen alle Standardabweichungen leicht über denen der ersten Messung,
 sind aber im Rahmen.
 Die großen Abweichungen der letzten beiden Messwerte können als Messfehler klassifiziert werden.
 Eine mögliche Fehlerquelle für die Ausreißer liegt beim in der Schaltung verbauten Taster,
  welcher während der Messung lang gedrückt werden muss. Eine leichte Druckabnahme auf den Schalter
  kann zu einer Verfälschung der gemessenen Werte führen.
 Somit können die bestimmten $U_0$ und $R_i$ als aussagekräftig bezeichnet werden.
 Jedoch fällt auf, dass die aus den beiden Ausgleichsrechnungen bestimmten Werte für $U_0$
 nicht in den angegebenen $1\sigma$-Umgebungen zueinander liegen.
 Dies lässt sich durch systematische Fehler der Messgeräte bzw. der Schaltungskomponenten erklären.

Auch die Messung bei der Rechteckspannungsquelle zeigt ein ähnlich gutes Bild.
 Da  der Schalter sowohl bei Rechteck- als auch bei Sinusspannung nicht mehr gedrückt werden musste,
entfällt diese Fehlerquelle. Dies zeigt sich auch leicht in den bestimmten Unsicherheiten,
welche im allgemeinen etwas kleiner sind. So liegt die Unsicherheit des neuen $R_i$ bei
unter 2,5\%, die der Effektivspannung sogar bei unter 1\%. Auch hier ist der Ausreißer wahrscheinlich
ein einfacher Messfehler.

Zum Schluss liefert die Messung mit der Sinusspannungsquelle die besten Ergebnissen.
Die Standardabweichung von $R_i$ liegt bei unter 1\%, die von $U_0$ sogar bei unter 0,5\%.
 Es existieren auch keine größeren Ausreißer.
 Im Ganzen schließen die Graphen auf aussagekräftige Messreihen.

Der in 4.3 bestimmte systematische Fehler ist mit $2,64\text{ }\mu$V in Relation zu den bestimmten $U_0$ viel zu gering
um einen wichtigen Unterschied in Rechnungen zu erzeugen. Somit ist die Näherung in Durchführungsteil a)
 vollkommen gerechtfertigt. Die durch $R_i$ hervorgerufene Unsicherheit ist sogar nochmals
 zwei Größenordnungen kleiner, womit auch sie vernachlässigt werden kann.

 Zu einem anderen Ergebnis kommt man beim systematischen Fehler in 4.4, welcher durch falsche Positionierung des Voltmeters hervorgerufen wird.
Hier ist der Innenwiderstand eines üblichen Amperemeters zwar klein, würde aber nacher vollständig vom ermittelten Innenwiderstand $R_i$ abgezogen werden müssen. Bei einem typischen Wert von $1,8\text{ }\Omega$ bei einer Reichweite von $400$ mA [2] würde dies sich deutlich in dem bestimmten $R_i$ bemerkbar machen.

Die Leistungkurve weist im allgemeinen wie bereits beschrieben keine eindeutige Tendenz auf. Allein im Bereich bis zum Theoriemaximum
könnte eine leichte Tendenz nach unten vorliegen. Die Messergebnisse liegen daher hauptsächlich
 im Bereich der Messungenauigkeit. Allerdings fehlt es der Messreihe an Aussagekraft, da vorallem im Bereich
  hinter dem Theoriemaximum starke Abweichungen auftreten und die Theoriekurve regelmässig
  unterschritten oder überschritten wird. Daher kann für den hinteren Teil keine derartige Aussage getroffen werden.

\section{Literatur}
\begin{enumerate}[{[}1{]}]
	\item TU Dortmund. Versuchsanleitung zum Versuch 301 - EMK und Innenwiderstand von Spannungsquellen. 2014.
	
	\item Newcombe, Chuck: Can you live with the burden? (o.J.).\\ http://www.fluke.com/fluke/uses/comunidad/fluke-news-plus/articlecategories/electrical/burdenvoltage (abgerufen am 15.11.2016)
\end{enumerate}



