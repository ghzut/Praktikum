\section{Diskussion}
\label{sec:Diskussion}


Zunächst werden die $U_0/I $-Kurven der einzelnen Spannungsquellen sowie ihre Auswertung diskutiert.

Bei der Betrachtung der $U_0 /I$-Kurve der Gleichspannungsquelle ohne Gegenspannung
fällt auf das die Messpunkte alle relativ nahe zur dieser liegen, was auf eine gute
Messung schliessen lässt. Der einzelne Ausreisser bei (36,5|0,95) kann als Messfehler
eingestuft werden.
 Im Detail liegt die Standardabweichung der Ausgleichsgeradensteigung bei ca. 3\%. Das durch den
 y-Achsenabschnitt bestimmte $U_0$ besitzt eine Standartabweichung von ca. 1,3\%, wobei ein möglicher
 systematischer Fehler vernachlässigt wird.
 Die y-Standardabweichung der Messwerte zur Ausgleichsgraden liegt nur bei ca. 1,9\%,
  was die Gerade aussagekräftig macht.

Die Messung der Gleichstromspannungsquelle mit Gegenspannung bestätigt die $U_0$ und $R_i$ Ergebnisse
der Messreihe aus Tabelle 2. Sowohl $U_0$ als auch $R_i$ liegen bei beiden Messreihen
 relativ nahe beieinander.Die Messreihe selbst ist ausreichend gut gelungen.
 Zwar liegen alle Standardabweichungen leicht über denen der ersten Messung,
 sind aber im Rahmen.
 Die großen Abweichungen der letzten beiden Messwerte können als Messfehler klassifiziert werden.
 Eine mögliche Fehlerquelle für die Ausreißer liegt am, in der Schaltung verbauten Taster,
  welcher während der Messung lang gedrückt werden muss. Eine leichte Druckabnahme auf den Schalter
  führt hierbei zu einer Verfälschung der gemessenen Werte.
 Somit können die bestimmten $U_0$ und $R_i$ als aussagekräftig bezeichnet werden.
 Jedoch fällt auf, dass die aus den beiden Ausgleichsrechnungen bestimmten Werte für $U_0$
 nicht in den angegebenen $1\sigma$-Umbgebungen zueinander liegen.
 Dies lässt sich durch systematische Fehler der Messgeräte bzw. der Schaltungskomponenten erklären.


Auch die Messung bei der Rechteckspannungsquelle zeigt ein ähnlich gutes Bild.
 Da  der Schalter sowohl bei Rechteck- als auch bei Sinusspannung nichtmehr gedrückt werden musste,
entfällt diese Fehlerquelle. Die zeigt sich auch leicht in den bestimmten Unsicherheiten,
welche im allgemeinen etwas kleiner sind. So liegt die Unsicherheit des neuen $R_i$ bei
unter 2,5\%, die der Effektivspannung sogar bei unter 1\%. Auch hier ist der Ausreißer wahrscheinlich
ein einfacher Messfehler.

Zum Schluss liefert die Messung mit der Sinusspannungsquelle die besten Ergebnissen.
Die Standardabweichung von $R_i$ liegt bei unter 1\%, die von $_0$ sogar bei unter 0,5\%.
 Es existieren auch keine größeren Ausreißer.
 Im gGanzen schließen die Graphen auf aussagekräftige Messreihen.
