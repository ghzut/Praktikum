\section{Auswertung}
\label{sec:Auswertung}


Die Graphen wurden sowohl mit Matplotlib \cite{matplotlib} als auch NumPy \cite{numpy} erstellt. Die
Fehlerrechnung wurde mithilfe von Uncertainties \cite{uncertainties} durchgeführt.
Der verwendete Einzelspalt besitzt eine Breite von $\SI{0.075}{\milli\meter}$. Der erste Doppelspalt
 hat eine Spaltbreite von $\SI{0.25}{\milli\meter}$ bei einem Spaltabstand von $\SI{0.15}{\milli\meter}$.
 Der erste Doppelspalt
 hat eine Spaltbreite von $\SI{0.1}{\milli\meter}$ bei einem Spaltabstand von $\SI{0.4}{\milli\meter}$.
%Die Konstanten $k$, $\hslash$, $e_0$, $m_0$, $u_0$ und $N_\text{A}$ sind vom NIST \cite{nistgov}.
\subsection{Die gemessenen Daten}
\begin{table}
 \centering
 \caption{Die gemessenen Daten am Einzelspalt.}
 \input{build/tabeinzelspalt.tex}
 \label{tab:einzel}
\end{table}

\begin{table}
 \centering
 \caption{Die gemessenen Daten am ersten Doppelspalt.}
 \input{build/tabds1.tex}
 \label{tab:ds12}
 \input{build/tabds12.tex}
\end{table}

\begin{table}
 \centering
 \caption{Die gemessenen Daten am zweiten Doppelspalt.}
 \input{build/tabds2.tex}
 \label{tab:ds2}
 \input{build/tabds22.tex}
\end{table}

\subsection{Bestimmung der Spaltbreite eines Einzelspaltes aus seinem Beugungsbild}
Zunächst wird die Spaltbreite des verwendeten Einzelspaltes aus dem entstandenen Beugungsbildes bestimmt.
Auf Basis von Formel \eqref{eq:1Spalt} folgt die nichtlineare Ausgleichsrechnung der Form:
\begin{equation}
  y(x) = A \left( \frac{\sin(B (x+C))}{(x+C)}\right)^2 \text{,}
\end{equation}
wobei sich die Spaltbreite $b$ durch
\begin{equation}
  b = \frac{B \lambda}{\pi}
  \end{equation}
  berechnet. Um eine Ausgleichsrechnung dieser Form durchführen zu können wird $\sin(\varphi)$mit $\tan(\varphi)$ genähert.
  Aufgrund der $\Delta x$ im der Größenordnung weniger Millimeter und einer Entfernung
   zwischen Spalt und Laser von $\SI{126.5}{\centi\meter}$ ist die Näherung möglich. Mit den Daten aus Tabelle \ref{tab:einzel} folgen die Parameter
  $A = \SI{87000(439)}{}$, $B = \SI{383(2)}{}$, $C = \SI{-0.60(2)}{\milli\meter}$ und
   die Spaltbreite $b = \SI{0.0774(4)}{\milli\meter}$. Der Parameter $C$ ist nach oben beschriebenen Maß bereits in $\si{\milli\meter}$ zurückberechnet.
\begin{figure}
 \centering
 \caption{Das bestimmte Intensitätenbeugungsbild des Einzelspaltes mit der Stromstärke gegenüber dem Abstand zur Mitte aufgetragen.}
 \includegraphics[width=\linewidth-70pt,height=\textheight-70pt,keepaspectratio]{build/einzelspalt.pdf}
 \label{fig:einzer}
\end{figure}

\subsection{Betrachtung der beiden vermessenen Doppelspaltbeugungsbilder}
\begin{figure}
 \centering
 \caption{Das bestimmte Intensitätenbeugungsbild des ersten Doppelspaltes mit der Stromstärke gegenüber dem Abstand zur Mitte aufgetragen.}
 \includegraphics[width=\linewidth-70pt,height=\textheight-70pt,keepaspectratio]{build/ds1.pdf}
 \label{fig:dseins}
\end{figure}

\begin{figure}
 \centering
 \caption{Das bestimmte Intensitätenbeugungsbild des zweiten Doppelspaltes mit der Stromstärke gegenüber dem Abstand zur Mitte aufgetragen.}
 \includegraphics[width=\linewidth-70pt,height=\textheight-70pt,keepaspectratio]{build/ds2.pdf}
 \label{fig:dszwei}
\end{figure}
Nun werden die experimentellen Messwerte der beiden Doppelspalte der Tabellen \ref{tab:ds12} bzw. \ref{tab:ds2} mit ihren zugehörigen Theoriegraphen nach Formel \eqref{Jochen.3} verglichen.
Um eine bessere Übereinstimmung mit den Werten zu erreichen wird der beim Einzelspalt bestimmte Korrekturwert beim Graphen des ersten Doppelspalts berücksichtigt.
Der Zweite Doppelspalt zeigt in dieser Hinsicht keine Verbesserung, weshalb er dort weggelassen wird.
Es ist erkennbar, dass die Messwerte die Theoriekurven in beiden Fällen ungefähr abbilden. Der Graph des ersten Doppelspalts weist weniger
 dichte und im allgemeinen auch kleinere Peaks auf als der Zweite.
Der Kurvenverlauf des Ersten weist zudem weitaus bessere Übereinstimmung mit den Werten auf als die des Zweiten. Die dargestellten Einhüllenden,
welche nach Formel \eqref{eq:1Spalt}
Die Ursachen hierfür müssen in der Diskussion geklärt werden.
