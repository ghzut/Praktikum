
\section{Theorie}
\label{sec:Theorie}

Beugung ist eine typische Welleneigenschaft. Man spricht von Beugung an einem Hindernis, wenn ein Welle beim Passieren ihre Ausbreitungsrichtung verändert. Gleichzeitig kommt es zur Interferenz also zur Überlagerung von den Amplituden der Welle. Die Intensität der Welle hinter dem Hindernis in einem bestimmten Abstand wird Beugungsbild genannt. Eine mögliche Erklärung für Beugung liefert das HYIGENSCHE PRINZIP. Dieses besagt, das eine Welle sich in jedem Punkt der Welle wie eine Kugelwelle ausbreitet und die Superposition dieser die Welle ergeben. Die Amplitude einer auslaufenden Kugelwelle ist proportional zu $\exp(i (k r -w t+\phi))$, wobei $k$ die Wellenzahl, $r$ der Abstand zum Ausgangspunkt der Kugelwelle, $w$ die Kreisfrequenz der Welle und $t$ die Zeit ist. 

\subsection{Beugung von kohärenten Wellen an einer Blende}

Im Folgendem wird zur Vereinfachung davon ausgegangen, dass eine kohärente Wellenfront homogener Intensität auf die Öffnung der Blende trifft, also dass die Welle an jeder Stelle der Öffnung die selbe Phase $\phi$ und Amplitude $A$ besitzt. Nach dem HYIGENSCHEn PRINZIP ergibt sich dann durch Summation über alle Amplituden der Kugelwellen für die Gesamtamplitude hinter der Öffnung
\begin{equation}
	A_n \propto \Sum\limits_i \exp(i (k r_i -w t+\phi))=
\end{equation}