
\section{Diskussion}
\label{sec:Diskussion}
Die Auswertung hat einige Ergebnisse erbracht, welche nun noch zu diskutieren
sind. Zunächst folgt der Einzelspalt. Für diesen wurde eine Spaltbreite von
$b = \SI{0.0774(4)}{\milli\meter}$. Ein Vergleich mit der auf
der Blende angegeben Spaltbreite ergibt einen relativen Fehler von $\SI{3.3(6)}{\percent}$.
Daher ist die bestimmte Spaltbreite größer als zu erwarten wäre. Dies lässt
 auf einen systematischen Fehler schließen. Des weiteren ist lässt sich am ermittelten
  Verschiebungsparameter $c$ erkennen, dass die experimentelle Kurve leicht
  verschoben gegenüber der Theoriekurve ist. Dies lässt sich darauf zurückführen,
  dass der Laser händisch auf den Detektor fokussiert werden musste und dieser
   daher nicht exakt mittig auf diesen gerichtet war. Zudem war der Laserstrahl breiter
    als die Detektoröffnung, was eine genaue Ausrichtung noch erschwert hat.

    Die experimentellen Messdaten der Intensitätsverläufe der Doppelspaltbeugungen
     zeigen nur grobe Übereinstimmungen mit den theoretischen Verläufen. Dies lässt
      sich auf die geringe Zahl der bestimmten Messwerte zurückführen, was
      vor allem im stark schwankenden Bereich um das Zentrum zu bemerken ist.
      Da der Zweite Doppelspalt den stärkeren Schwankungen unterliegt, sind die
       Messungenauigkeiten dementsprechend stärker zu erkennen. Es ist zu vermuten,
        dass der Laser nach der zweiten Messreihe leicht verstellt wurde und der
        bestimmte Korrekturparameter deswegen Verbesserung beim Graphen des zweiten Doppelspalts erzielt. 
