
\section{Durchführung}
\label{sec:Durchführung}
Zuerst werden Durchmesser und Masse von zwei unterschiedlich großen Glaskugeln
gemessen um daraus die jeweiligen Dichten zu ermitteln. Im Anschluss wird das
Viskosimeter mithilfe der Libelle ausgerichtet, sodass es gerade steht und die Fallröhre anschließend
mit destiliertem Wasser aufgefüllt. Um eine Verfälschung der Messszeiten durch
Luftblasen zu vermeiden, werden diese vorher mit einem Glasstab entfernt.
Anschließend wird zunächst die kleine Kugel unter Zimmertemperatur in die
Fallröhre eingesetzt und letztere wasserdicht verschlossen. Danach wird die Fallzeit
gemessen, welche die Kugel von der Oberen bis zur unteren Markierung benötigt.
Dieser Vorgang wird Zehn mal wiederholt und danach auch noch mit der großen Kugel
wiederholt. Um eine neue Messung lässt sich das Viskosimeter an der Befestigung
frei drehen. Im folgenden wird das Wasser mithilfe des Wasserbades erwärmt. Es
werden die Fallzeiten zu zehn verschiedenen Temperaturen notiert, jeweils mit
zwei Einzelmessungen, bis $\SI{70}{\degreeCelsius}$ erreicht sind.
