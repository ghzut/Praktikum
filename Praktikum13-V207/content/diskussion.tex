
\section{Diskussion}
\label{sec:Diskussion}
Zunächst folgt ein Vergleich der Koeffizienten der dynamischen Viskosität. Der
bestimmte Wert des Vorfaktors $A$ ist mit $\SI{3.8(0.4)}{\micro\pascal\second}$ ca.
dreimal so groß wie der zugehörige Literaturwert. Dieser liegt bei ca.
$\SI{0.96}{\micro\pascal\second}$ \cite{etalit}. Die bestimmte Achsenabschnitt liegt mit
$\SI{166(3)e1}{\kelvin}$ hingegen ca. $\SI{25}{\percent}$ unter dem Literaturwert
von $\SI{2036}{\kelvin}$ \cite{etalit}. Eine mögliche Ursache hierfür ist, dass die am Thermostat
abgelesene Temperatur höher ist, falls sich das Fallrohr nicht vollständig erwärmen
konnte. Eine weitere mögliche Fehlerquelle liegt in der zuvor bestimmten Viskosität
bei Normaltemperatur. Sie liegt bei ca. $\SI{1138}{\micro\pascal\second}$ und
entspricht damit mehr einer Temperatur von $\SI{15}{\degree\Celsius}$ \cite{vislit} als einer von
$\SI{20}{\degreeCelsius}$. Dies kann entweder durch einen Messfehler bei der
Dichtebestimmung oder einer falsch angegeben Apparatekonstante für die kleine Kugel
liegen. Eine Verfälschung der Ergebnisse durch Luftblasen ist nicht zu erwarten.
Die bestimmten Reynoldzahlen liegen alle weit unterhalb des kritischen
Bereiches. Daher ist es wahrscheinlich, dass die Strömungen laminar sind.
