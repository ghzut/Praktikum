
\section{Theorie}
\label{sec:Theorie}
\subsection{Theorie}
Wird ein Körper durch eine Flüssigkeit bewegt, so kommt es zu einer Interaktion
der Körperoberfläche mit den umliegenden Teilchen der Flüssigkeit, kurz Reibung.
 Es wirkt daher eine Reibungskraft $\vec{F_\text{R}}$ auf diesen. Sie ist abhängig von
Geschwindigkeit und Berührungsfläche des Körpers. Liegen in der Flüssigkeit
keine Wirbelströmungen vor, ist die Strömung laminar und $\vec{F_\text{R}}$ lässt sich
bei einer Kugelgeometrie mit der stokesschen Reibungskraft
\begin{equation}
  \vec{F_R} = -6 \pi \eta r \vec{v}\label{stokeR}
  \end{equation}
  beschreiben. Hierbei bezeichnet $\eta$ die Viskosität des Stoffes und $\vec{v}$ den Geschwindigkeitsvektor des Körper gegenüber der Flüssigkeit. Die Viskosität ist
  stark Temperaturabhängig und lässt sich für Wasser durch die Andradesche
  Gleichung
  \begin{equation}
    \eta(T) = A \exp\left(\frac{B}{T}\right)\label{andra}
    \end{equation}
darstellen. Die Konstanten $A$ und $B$ sind materialspezifisch.
Bei der Viskositätsbestimmung nach Höppler wird eine Kugel in die zu untersuchenden
Flüssigkeit gesenkt. Es wirkt nun die Reibungskraft der Flüssigkeit $\vec{F_\text{R}}$ und der Auftrieb $\vec{F_\text{A}}$ gegen die
Schwerkraft $\vec{F_\text{g}}$ an. Nach einer kurzen Zeit stellt sich ein Kräftegleichgewicht ein und
 die Kugel sinkt mit konstanter Geschwindigkeit. Mit der bestimmten Zeit $t$, welche
  die Kugel für die Fallstrecke benötigt folgt die Viskosität durch:
  \begin{equation}
    \eta = K(\rho_\text{K} - \rho_\text{Fl})\cdot t\label{visko}
    \end{equation}
mit der Kugeldichte $\rho_\text{K}$, der Flüssigkeitsdichte $\rho_\text{Fl}$ und der
Apparatekonstante $K$, in welcher die Kugelgeometrie und die Fallstrecke enthalten sind.
Schlussendlich beschreibt die Reynoldsche Zahl $Re$ das Verhältnis zwischen
zwischen Trägheits- und Viskositätskräften, welche auf einen Körper der
charakteristischen Länge $d$ wirken. Sie berechnet sich durch:
\begin{equation}
  Re = \frac{\rho v d}{\eta}\label{Re}
  \end{equation}
  Mithilfe der Reynoldszahl lässt sich bestimmen ob Strömungen in einer
  Flüssigkeit laminar sind. Liegt sie unterhalb eines kritischen Wertes $Re_{krit}$, ist
  sie laminar, liegt sie darüber ist sie turbulent. Für Wasser liegt dieser bei ca. $2300$ \cite{Rekrit}
Es ist jedoch zu beachten, dass die bestimmte Reynoldszahl durch die charakteristische Länge nicht eindeutig ist und auch
$Re_{krit}$ nur empirisch bestimmt ist. Liegt der bestimmte Wert jedoch weit unterhalb
 von $Re_{krit}$, kann trotzdem von einer laminaren Strömung ausgegangen werden.
