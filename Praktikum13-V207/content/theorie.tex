
\section{Theorie}
\label{sec:Theorie}
\subsection{Die Reibungskr}
Wird ein körper durch eine Flüssigkeit bewegt, so kommt es zu einer Interaktion
der Körperoberfläche mit den umliegenden Teilchen der Flüssigkeit, kurz Reibung.
 Es wirkt daher eine Reibungskraft $\vec{F_R}$ auf diesen. Sie ist abhängig von
Geschwindigkeit und Berührungsfläche des Körpers. Liegen in der Flüssigkeit
keine Wirbelströmungen vor ist die Strömung laminar und $\vec{F_R}$ lässt sich
bei einer Kugelgeometrie mit der stokeschen Reibung
\begin{equation}
  \vec{F_R} = 6 \pi \eta \nu r
  \end{equation}
  beschreiben. Hierbei bezeichnet $\eta$ die Viskosität des Stoffes. Sie ist
  stark Temperaturabhängig und lässt sich für Wasser durch die Andradesche
  Gleichung
  \begin{equation}
    \eta(T) = A \exp(\frac{B}{T})
    \end{equation}
darstellen. Die Konstanten $A$ und $B$ sind materialspezifische.
Schlussendlich definiert die Reynoldsche Zahl $Re$ das Verhältnis zwischen
zwischen Trägheits- und Viskositätskräften, welche auf einen Körper der Länge $d$
wirken. Sie berechnet sich durch:
\begin{equation}
  Re = \frac{\rho v d}{\eta}
  \end{equation}
