\section{Auswertung}
\label{sec:Auswertung}

Die Graphen wurden sowohl mit Matplotlib \cite{matplotlib} als auch NumPy \cite{numpy} erstellt. Die
 Fehlerrechnung wurde mithilfe von Uncertainties \cite{uncertainties} durchgeführt.



\subsection{Berechnung der Dichte der Glaskugeln}
\begin{table}
	\centering
	\caption{Die gemessenen Werte für den Durchmesser $D_\text{kl}$ und die Masse $M_\text{kl}$ der kleinen Glaskugel.}
	\input{build/klKW.tex}
\end{table}
Mit den gemessenen Werten, für die Masse und den Durchmesser, der kleinen Glaskugel aus Tabelle \ref{tab:klKW} errechnet sich die Masse der kleinen Glaskugel zu
\begin{displaymath}
	m_\text{kl} = \SI{4.453(3)}{\gram}
\end{displaymath}
und der Durchmesser zu
\begin{displaymath}
	D_\text{kl} = \SI{15.63(0)}{\milli\meter}\text{.}
\end{displaymath}
Nun lässt sich die Dichte der kleinen Glaskugel bestimmen. Es ergibt sich
\begin{displaymath}
	\rho_\text{kl} = \frac{m_\text{kl}}{\frac{4}{3}\pi \left(\frac{D_\text{kl}}{2}\right)^3} = \SI{2228(2)}{\kilo\gram\per\meter\tothe{3}}\text{.}
\end{displaymath}
\begin{table}
	\centering
	\caption{Die gemessenen Werte für den Durchmesser $D_\text{gr}$ und die Masse $M_\text{gr}$ der großen Glaskugel.}
	\input{build/grKW.tex}
\end{table}
Mit den gemessenen Werten, für die Masse und den Durchmesser, der großen Glaskugel aus Tabelle \ref{tab:grKW} errechnet sich die Masse der großen Glaskugel zu
\begin{displaymath}
m_\text{gr} = \SI{4.960(6)}{\gram}
\end{displaymath}
und der Durchmesser zu
\begin{displaymath}
D_\text{gr} = \SI{15.803(3)}{\milli\meter}\text{.}
\end{displaymath}
Nun lässt sich die Dichte der großen Glaskugel bestimmen. Es ergibt sich
\begin{displaymath}
\rho_\text{gr} = \frac{m_\text{kl}}{\frac{4}{3}\pi \left(\frac{D_\text{kl}}{2}\right)^3} = \SI{2400(3)}{\kilo\gram\per\meter\tothe{3}}\text{.}
\end{displaymath}



\subsection{Berechnung der Apparatekonstante für die große Glaskugel}
\begin{table}
	\centering
	\caption{Die bestimmte Viskosität $\eta$ bei verschiedenen Temperaturen $T$.}
	\input{build/table1.tex}
\end{table}
Mit den Werten aus Tabelle \ref{tab:table1} berechnet sich die Fallzeit der kleinen Glaskugel zu
\begin{displaymath}
	t_\text{kl} = \SI{12.12(4)}{\second}
\end{displaymath}
und der großen Glaskugel zu
\begin{displaymath}
	t_\text{gr} = \SI{69.9(2)}{\second}\text{.}
\end{displaymath}
Da die Viskosität bei den beiden Messreihen mit verschieden Kugeln die gleiche ist, folgt aus der Formel \eqref{??}, mit dem Literaturwert für die Dichte des Wasser $\rho_\text{Wasser}$ unter Normalbedingungen von $\SI{998}{\kilo\gram\per\meter\tothe{3}}$ \cite{eta} und der gegebenen Apparatekonstante der kleinen Glaskugel $K_\text{kl}$ von $\SI{76.40e-9}{\pascal\meter\tothe{3}\per\kilo\gram}$
\begin{displaymath}
	K_\text{gr} = \frac{K_\text{kl} \left( \rho_\text{kl} - \rho_\text{Wasser}\right) t_\text{kl}}{\left( \rho_\text{gr} - \rho_\text{Wasser}\right) t_\text{gr}} = \SI{11.60(6)e-9}{\pascal\meter\tothe{3}\per\kilo\gram}\text{.}
\end{displaymath}

\subsection{Bestimmung der Parameter A und B der Andradeschen Gleichung}


\begin{table}
	\centering
	\caption{Die gemessene Fallzeit $t_1$ und $t_2$ der großen Glaskugel bei verschiedenen Temperaturen $T$ und die daraus berechnete Fallzeit $t$ und Viskosität $\eta$.}
	\input{build/table2.tex}
\end{table}
\begin{figure}
	\centering
	\caption{Der natürliche Logarithmus von der Viskosität $\eta$ gegen die reziproke Temperatur $1/T$ aufgetragen.}
	\includegraphics[width=\linewidth-70pt,height=\textheight-70pt,keepaspectratio]{build/tT.pdf}
	\label{fig:Graph}
\end{figure}
Mithilfe der zuvor bestimmten Werte (die Apparatekonstante $K_\text{gr}$ und die Dichte $\rho_\text{gr}$) berechnet sich mit dem Literaturwert für die Dichte des Wasser von $\SI{998}{\kilo\gram\per\meter\tothe{3}}$ \cite{eta} die Viskosität des Wassers $\eta$ in Tabelle \ref{tab:table2} sich nach Formel \eqref{??} aus den Werten für die Fallzeit $t$ aus Tabelle \ref{tab:table2}. In dem Graphen in Abbildung \ref{fig:Graph} ist der natürliche Logarithmus von der Viskosität gegen die reziproke Temperatur aufgetragen. 
Der Fit in Abbildung \ref{fig:Graph2} besitzt die Form $y=a x + b$. Eine lineare Ausgleichsrechnung der Form $y=a x + b$ mittels SciPy \cite{scipy} liefert mit den Wertepaaren aus Tabelle \ref{tab:table2} nach Formel \eqref{??}
\begin{displaymath}
	B = a = \SI{166(3)e1}{\kelvin}
\end{displaymath}
und
\begin{displaymath}
	A = \exp(b) = \SI{3.8(4)}{\micro\pascal}\text{.}
\end{displaymath}

\subsection{Bestimmung von Reynoldschen Zahlen}
Mithilfe der Formel \eqref{??} lässt sich nun die Reynoldsche Zahl $Re$ bei verschiedenen Temperaturen für die Glaskugeln bestimmen. Bei Raumtemperatur von ca. $\SI{20}{\degreeCelsius}$ berechnet sich die Reynoldsche Zahl bei der kleinen Kugel zu
\begin{displaymath}
	Re_1 = \num{253(2)}\text{,}
\end{displaymath}
 bei der großen Kugel zu
\begin{displaymath}
	Re_2 = \num{47.7(2)}
\end{displaymath}
und bei $\SI{20}{\degreeCelsius}$ bei der großen Kugel zu
\begin{displaymath}
Re_2 = \num{26(4)e1}
\end{displaymath}
