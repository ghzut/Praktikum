
\section{Diskussion}
\label{sec:Diskussion}
Die Auswertung liefert Ergebnisse die nun diskutiert werden sollen. Die ermittelte Wellenlänge liegt mit $\lambda = \SI{662.0(3)}{\nano\meter}$ im Bereich des roten sichtbaren Lichtes \cite{sichtbaresSpektrum}, was während der Durchführung auch beobachtet werden konnte. Die ermittelte Wellenlänge weicht um $\SI{2}{\percent}$ von dem angegeben Wert für die Wellenlänge des Lasers von $\SI{650}{\nano\meter}$ nach oben ab. Dies könnte durch den Ablesefehler oder auch durch das Übergehen einiger Maxima, wodurch nach Formel \eqref{lambda} auch die Wellenlänge größer ausfallen würde, erklärt werden. Der ermittelte Brechungsindex $n= \SI{1.00030(1)}{}$ liegt mit seiner $\sigma$-Umgebung im Bereich des Literaturwertes von $n_\text{lit}=\SI{1.000 292}{}$ \cite{nLuft} und widerspricht somit nicht dem Literaturwert.