
\section{Durchführung}
\label{sec:Durchführung}
Als erstes muss der optische Teil der Apperatur justiert. Dazu wird der justierbare
Spiegel so eingestellt, dass beide Lichtstrahlen parallel liegen und zentral auf das Photoelement fallen.
Zunächst wird die Wellenänge des Lasers über sein Interferenzbild bestimmt. Hierzu
wird der verschiebbare Spiegel über den Motor langsam verschoben. An diesem wird eine Geschwindigkeit von 1 eingestellt.
Zudem wird die aktuelle Einstellung der Mikrometerschraube notiert.
Der Spiegel wird nun solange verschoben bis der Detektor ca. 1000 Maxima gezählt hat. Es
wird die neue Einstellung der Mikrometerschraube , sowie die gemessene Zahl der Maxima notiert. Dieser Vorgang wird Zehn mal wiederholt.
Für den zweiten Versuchsteil wird nun der Luftdruck in der Messzelle variiert. Dazu
wird in der Messzelle mit der Vakuumpumpe ein Unterdruck von ca. $\SI{-0.7}{\bar}$ erzeugt.
Anschließend wird der Druck wieder langsam auf Normaldruck erhöht. Dabei wird wieder die Anzahl der auftretenden Maxima notiert.
