\section{Auswertung}
\label{sec:Auswertung}


Die Graphen wurden sowohl mit Matplotlib \cite{matplotlib} als auch NumPy \cite{numpy} erstellt. Die
Fehlerrechnung wurde mithilfe von Uncertainties \cite{uncertainties} durchgeführt.


\subsection{Bestimmung der Zerfallskurve und Halbwertszeit von Indium}
 \begin{table}
  \centering
  \caption{FEHLT NOCH.}
  \input{build/Indium.tex}
 \end{table}
\begin{figure}
	\centering
	\caption{Der Logarithmus der gemessenen durchschnittlichen Zählraten $N$ bei Indium in einem Zeitintervall von $t-240\si{\second}$ bis $t$ gegen die vergangene Zeit $t$ aufgetragen.}
	\includegraphics[width=\linewidth-70pt,height=\textheight-70pt,keepaspectratio]{build/IndiumLog.pdf}
	\label{fig:Indium}
\end{figure}

\subsection{Bestimmung der Halbwertszeiten von Rhodium}
Es sollen sowohl die Halbwertszeit von $^{104} Rh$ als auch $^{104i} Rh$ mithilfe von zwei Ausgleichsrechnungen bestimmt werden. Dabei handelt es sich bei $^{104i} Rh$ um einen zu $^{104} Rh$ isomeren Kern.
 \begin{table}
	\centering
	\caption{FEHLT NOCH.}
	\input{build/Rhodium1.tex}
\end{table}
\begin{table}
	\centering
	\caption{FEHLT NOCH.}
	\input{build/Rhodium2.tex}
\end{table}
\begin{figure}
	\centering
	\caption{Der Logarithmus der gemessenen durchschnittlichen Zählraten $N$ bei Rhodium in einem Zeitintervall von $t-240\si{\second}$ bis $t$ gegen die vergangene Zeit $t$ aufgetragen.}
	\includegraphics[width=\linewidth-70pt,height=\textheight-70pt,keepaspectratio]{build/RhodiumLog.pdf}
	\label{fig:Rhodium}
\end{figure}


