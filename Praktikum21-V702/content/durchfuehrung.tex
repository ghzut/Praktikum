
\section{Durchführung}
\label{sec:Durchführung}
Um Fehler aufgrund von kosmischer Strahlung zu minimieren wird zunächst ein Leerlauf ohne Isotop durchgeführt.
 Um ein möglichst aussagekräftiges Ergebnis zu erhalten wird hierzu eine Messzeit von $\SI{1000}{\second}$ eingestellt.
Dann wird zunächst wird die Dämpfung der Gammastrahlung untersucht. Hierzu wird
das Isotop im Messaufbau platziert und eine Kupferplatte zwischen Isotop und Sensor gestellt.
 Anschließend wird eine Messzeit, bei welcher mehr als 10000 Treffer gemessen werden. Es werden
 Dicke der Platte, Messzeit und Trefferzahl notiert. Dies wird für Kupfer und Eisenplatten in jeweils 10 Dicken wiederholt.
 Anschließend wird die Betastrahlung gemessen. Der Messvorgang ist derselbe wie
 bei der Gammastrahlung, jedoch werden dieses mal nur Aluminiumplatten als Absorber
 verwendet und die Trefferzahlen sollten nicht kleiner als 500 sein. Diese Maßnahme
 ist der geringeren Trefferintensitäten geschuldet. 
