
\section{Diskussion}
\label{sec:Diskussion}
\begin{table}
	\centering
	\caption{Die in der Auswertung bestimmten Werte mit zugehörigen Literaturwerten und die relative Abweichung von diesen.}
	\input{build/Ergebnisse.tex}
	\label{tab:disss}
\end{table}
Die Auswertung hat einige Ergebnisse ergeben, welche nun zu diskutieren sind. Zunächst wird die Halbwertszeit von $^{116}$In betrachtet. Aus  Tabelle \ref{tab:disss} folgt, dass der experimentelle Wert leicht oberhalb der $\sigma$-Umgebung des Literaturwertes liegt. In Abb. \ref{fig:Indium} ist zu erkennen, dass ca. die Hälfte der einzelnen Messwerte mitsamt ihrer $\sigma$-Umgebung außerhalb der Ausgleichsgeraden liegen. Dies lässt vermuten, dass die Standartabweichung vermutlich größer ist als die Angegebene.
% Dies ist zum Teil durch den stochastischen Zerfalssprozess zu erklären, welcher bei Messabständen von $\SI{240}{\second}$ noch erkennbar ist.
 Die bestimmten Halbwertszeiten der beiden Rhodiumisomere liegen beide im Bereich der Literaturwerte. Es ist jedoch zu beachten, dass diese auf Basis selbst gewählter Parameters bestimmt wurden. Im Hinblick auf die geringen Fehler bezüglich der Literaturwerte erscheinen die bestimmten Zerfälle nach $\SI{15}{\second}$ jedoch auch realistisch.  

% Eine mögliche Ursache liegt in der Radioaktivität. Die bestimmten Halbwertzeiten der beiden $^{104}$Rh Isomere liegen hingegen beide im Bereich der Literaturwerte. Da
%diese jedoch auf Basis selbst gesetzter Parameter ermittelt worden sind, kann die Richtigkeit der
